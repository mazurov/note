\section{Systematic Uncertainties}
\label{sec:syst}

This analysis measures the fraction of $\Upsilon(nS)$ particles originating
from $\chi_b$ decays, most systematic uncertainties cancel in the ratio and
only residual effects need to be taken into account. Systematic uncertainties
can be grouped according to various sources, related respectively to:
\begin{itemize}
\item the fit model of $\Upsilon$ and $\chi_b$ invariant masses, 
\item the determination of the photon reconstruction efficiency, and 
\item the unknown initial polarization of $\chi_b$ and $\Upsilon$ particles. 
\end{itemize}
These uncertainties will be detailed in the following. 

\subsection{Uncertainties related to the fit model}

The uncertainty related to the modeling of the $\Upsilon$ invariant mass
distribution has been estimated by following previous
studies~\ref{Aaij:2013yaa}. An uncertainty of 0.7\% has been assigned to the
yields of $\Upsilon(nS)$ mesons.

In the fit model of the $chi_b$ invariant masses, several sources need to be
taken into account. Firstly, the relative proportion of spin-1 and spin-2
states, which is kept fixed in the fit to values close to 0.5, predicted by
theory, is varied from 0.4 to 0.7. 

Table~\ref{tab:syst:lambda_ups1s} reports the relative
variation in percent of the $\chi_b$ yields as function of $\lambda$, the
relative proportion of the two $\chi_b$ states, for all examined decays, in
each bin of $\Upsilon(nS)$ transverse momentum. We take as systematic error in
each $p_T$ bin the maximum variation of the $\chi_b$ yields with respect to the
nominal fit.

Another source of systematic is due to the variation of the $\chi_b$ masses as
function of $p_T(\Upsilon)$, observed in Section~\ref{relevant_section_here}.
We repeat the fits by taking the minimum and maximum values of the $\chi_b$
masses and take the maximum difference in the yields as systematic uncertainty.
The resulting uncertainties are reported in
Tables~\ref{tab:syst:m1p}-\ref{tab:syst:m3p}.


Other uncertainties due to parameters taken from PDG (e.g. mass differences)
are assumed to be negligible.

\subsubsection{Uncertainties related to the Data-MonteCarlo resolution difference}

In the $\chi_b$ fits, the resolution of the Crystal Ball functions, determined
from simulation, has been scaled by a factor 1.16 in order to account for data ---
MonteCarlo differences. This factor was obtained by fitting a histogram, 
that store the ratio between data and MonteCarlo resolution, by constant function.
The result of fit is shown in Figure~\ref{fig:syst:ratio_data_mc_sigma}.

\begin{figure}[H]
  \setlength{\unitlength}{1mm}
  \centering
  \begin{picture}(75, 60)
  \put(0,0){
    \includegraphics*[width=75mm, height=60mm]{syst/sigma_b1_1p}
  }
  \put(50,0){$p_T^{\Y1S} \left[\gevc\right]$}
  \put(0,20){\begin{sideways} $\sigma_{\chiboneOneP} \left[\gevcc\right]$ \end{sideways}}
  
  \put(45,50){\includegraphics*[width=4mm, height=2mm]{blue}}
  \put(45,45){\includegraphics*[width=4mm, height=2mm]{red}}
  \put(49,50){\sqs=7\tev}
  \put(49,45){\sqs=8\tev}



  \end{picture}
  \label{fig:syst:data_sigma}
  \caption{\small \chiboneOneP yield resolution in $\chi_b \to \Y1S \gamma$ decay}
\end{figure}
\begin{figure}[H]
  \setlength{\unitlength}{1mm}
  \centering
  \begin{picture}(150, 60)
  \put(0,0){
    \includegraphics*[width=75mm, height=60mm]{syst/sigma_data_mc_2011}
  }
  \put(75,0){
    \includegraphics*[width=75mm, height=60mm]{syst/sigma_data_mc_2012}
  }

  \put(50,0){$p_T^{\Y1S} \left[\gevc\right]$}
  \put(125,0){$p_T^{\Y1S} \left[\gevc\right]$}
  
  \put(0,10){\begin{sideways} $\sigma_{\chiboneOneP}^{data}/\sigma_{\chiboneOneP}^{MC} \left[\gevcc\right]$ \end{sideways}}
  \put(75,10){\begin{sideways} $\sigma_{\chiboneOneP}^{data}/\sigma_{\chiboneOneP}^{MC} \left[\gevcc\right]$ \end{sideways}}

  \put(45,50){\sqs=7\tev}
  \put(120,50){\sqs=8\tev}




  \end{picture}
  \caption{\small The ratio of \chiboneOneP yield resolution in data to \chiboneOneP
  yield resolution in MonteCarlo in $\chi_b \to \Y1S \gamma$ decay. The black
  line on the plot shows the result of histogram fit by constant function.
  }
  \label{fig:syst:ratio_data_mc_sigma}
\end{figure}

We estimate the systematic uncertainty due to this assumption by repeating the
fits with scaling the $\sigma$ parameter with maximum and minimum values of
the scaling factor obtained from the fit in Figure~\ref{fig:syst:ratio_data_mc_sigma}.
Results are shown in Table~\ref{tab:syst:sigma_ups1s}.



\subsection{Photon reconstruction efficiency}
The photon reconstruction efficiency, taken from simulation, needs not to be
the same as in real data. The detailed comparison between MonteCarlo and data,
presented in Section~\ref{sec:mc:datavsmc}, shows that the differences
are small. We assign a systematic uncertainty based on previous studies of
photon reconstruction efficiencies. These studies compare the $B^+ \rightarrow
J/\psi K^*+$ and $B^+ \rightarrow J/\psi K^+$ yields in data and Monte Carlo in
order to determine the neutral pion, hence the photon  reconstruction
efficiency. A systematic uncertainty of 3\% is assigned to this effect.




\begin{table}[H]
\centering
\caption{\small $\chi_b$ yields systematic uncertainties (\%) related to the Data-MonteCarlo resolution difference in the fit model for $\chi_b(1,2,3P) \to \OneS \gamma$ decays}
\subtable[$6 < p_T^{\Y1S} < 10 \gevc$] {
\scalebox{0.65}{
\begin{tabular}{lrrrrrrrrrrrr}\toprule
 & \multicolumn{12}{c}{$\Upsilon(1S)$ transverse momentum intervals, \gevc}\\
 & \multicolumn{6}{c}{6 -- 8} & \multicolumn{6}{c}{8 -- 10}\\
\cmidrule(r){2-7}\cmidrule(r){8-13}
 & \multicolumn{3}{c}{\sqs = 7\tev} & \multicolumn{3}{c}{\sqs = 8\tev} & \multicolumn{3}{c}{\sqs = 7\tev} & \multicolumn{3}{c}{\sqs = 8\tev}\\
\cmidrule(r){2-4}\cmidrule(r){5-7}\cmidrule(r){8-10}\cmidrule(r){11-13}
 & $N_{\chi_{b}(1P)}$ & $N_{\chi_{b}(2P)}$ & $N_{\chi_{b}(3P)}$ & $N_{\chi_{b}(1P)}$ & $N_{\chi_{b}(2P)}$ & $N_{\chi_{b}(3P)}$ & $N_{\chi_{b}(1P)}$ & $N_{\chi_{b}(2P)}$ & $N_{\chi_{b}(3P)}$ & $N_{\chi_{b}(1P)}$ & $N_{\chi_{b}(2P)}$ & $N_{\chi_{b}(3P)}$\\
\midrule
$\frac{\sigma_{\chiboneOneP}^{data}}{\sigma_{\chiboneOneP}^{MC}} = 1.13$ & 2.6 & 1.8 & --- & 2.5 & 2.4 & --- & 2.3 & 1.6 & --- & 2.4 & 1.7 & ---\\
$\frac{\sigma_{\chiboneOneP}^{data}}{\sigma_{\chiboneOneP}^{MC}} = 1.20$ & -3.5 & -2.2 & --- & -3.6 & -1.5 & --- & -3.1 & -2.0 & --- & -3.3 & -2.2 & ---\\
\bottomrule
\end{tabular}
} % scalebox

} % subtable
\subtable[$10 < p_T^{\Y1S} < 18 \gevc$] {
\scalebox{0.65}{
\begin{tabular}{lrrrrrrrrrrrr}\toprule
 & \multicolumn{12}{c}{$\Upsilon(1S)$ transverse momentum intervals, \gevc}\\
 & \multicolumn{6}{c}{10 -- 14} & \multicolumn{6}{c}{14 -- 18}\\
\cmidrule(r){2-7}\cmidrule(r){8-13}
 & \multicolumn{3}{c}{\sqs = 7\tev} & \multicolumn{3}{c}{\sqs = 8\tev} & \multicolumn{3}{c}{\sqs = 7\tev} & \multicolumn{3}{c}{\sqs = 8\tev}\\
\cmidrule(r){2-4}\cmidrule(r){5-7}\cmidrule(r){8-10}\cmidrule(r){11-13}
 & $N_{\chi_{b}(1P)}$ & $N_{\chi_{b}(2P)}$ & $N_{\chi_{b}(3P)}$ & $N_{\chi_{b}(1P)}$ & $N_{\chi_{b}(2P)}$ & $N_{\chi_{b}(3P)}$ & $N_{\chi_{b}(1P)}$ & $N_{\chi_{b}(2P)}$ & $N_{\chi_{b}(3P)}$ & $N_{\chi_{b}(1P)}$ & $N_{\chi_{b}(2P)}$ & $N_{\chi_{b}(3P)}$\\
\midrule
$\frac{\sigma_{\chiboneOneP}^{data}}{\sigma_{\chiboneOneP}^{MC}} = 1.13$ & 2.4 & 1.9 & 5.4 & 2.2 & 1.2 & 5.8 & 1.4 & 2.5 & 1.7 & 1.5 & 2.1 & 3.0\\
$\frac{\sigma_{\chiboneOneP}^{data}}{\sigma_{\chiboneOneP}^{MC}} = 1.20$ & -3.1 & -2.5 & -7.4 & -3.0 & -1.5 & -8.8 & -1.9 & -3.3 & -2.5 & -2.0 & -2.8 & -3.6\\
\bottomrule
\end{tabular}
} % scalebox

} % subtable
\subtable[$18 < p_T^{\Y1S} < 40 \gevc$] {
\scalebox{0.65}{
\begin{tabular}{lrrrrrrrrrrrr}\toprule
 & \multicolumn{12}{c}{$\Upsilon(1S)$ transverse momentum intervals, \gevc}\\
 & \multicolumn{6}{c}{18 -- 22} & \multicolumn{6}{c}{22 -- 40}\\
\cmidrule(r){2-7}\cmidrule(r){8-13}
 & \multicolumn{3}{c}{\sqs = 7\tev} & \multicolumn{3}{c}{\sqs = 8\tev} & \multicolumn{3}{c}{\sqs = 7\tev} & \multicolumn{3}{c}{\sqs = 8\tev}\\
\cmidrule(r){2-4}\cmidrule(r){5-7}\cmidrule(r){8-10}\cmidrule(r){11-13}
 & $N_{\chi_{b}(1P)}$ & $N_{\chi_{b}(2P)}$ & $N_{\chi_{b}(3P)}$ & $N_{\chi_{b}(1P)}$ & $N_{\chi_{b}(2P)}$ & $N_{\chi_{b}(3P)}$ & $N_{\chi_{b}(1P)}$ & $N_{\chi_{b}(2P)}$ & $N_{\chi_{b}(3P)}$ & $N_{\chi_{b}(1P)}$ & $N_{\chi_{b}(2P)}$ & $N_{\chi_{b}(3P)}$\\
\midrule
$\frac{\sigma_{\chiboneOneP}^{data}}{\sigma_{\chiboneOneP}^{MC}} = 1.13$ & 1.2 & 1.7 & 1.8 & 1.3 & 1.8 & 1.7 & 1.2 & 1.8 & 2.0 & -4.1 & 10.4 & 21.5\\
$\frac{\sigma_{\chiboneOneP}^{data}}{\sigma_{\chiboneOneP}^{MC}} = 1.20$ & -1.6 & -2.3 & -2.2 & -1.7 & -2.4 & -3.9 & -1.6 & -2.2 & -2.6 & -0.9 & 2.4 & -10.3\\
\bottomrule
\end{tabular}
} % scalebox

} % subtable
\label{tab:syst:sigma_ups1s}
\end{table}

\begin{table}[H]
\centering
\caption{\small $\chi_b$ yields systematic uncertainties (\%)  related to $\lambda$ values in the fit model for  $\chi_b(1,2,3P) \to \OneS \gamma$ decays}
\subtable[$6 < p_T^{\Y1S} < 10 \gevc$] {
\scalebox{0.65}{
\begin{tabular}{lrrrrrrrrrrrr}\toprule
 & \multicolumn{12}{c}{$\Upsilon(1S)$ transverse momentum intervals, \gevc}\\
 & \multicolumn{6}{c}{6 -- 8} & \multicolumn{6}{c}{8 -- 10}\\
\cmidrule(r){2-7}\cmidrule(r){8-13}
 & \multicolumn{3}{c}{\sqs = 7\tev} & \multicolumn{3}{c}{\sqs = 8\tev} & \multicolumn{3}{c}{\sqs = 7\tev} & \multicolumn{3}{c}{\sqs = 8\tev}\\
\cmidrule(r){2-4}\cmidrule(r){5-7}\cmidrule(r){8-10}\cmidrule(r){11-13}
 & $N_{\chi_{b}(1P)}$ & $N_{\chi_{b}(2P)}$ & $N_{\chi_{b}(3P)}$ & $N_{\chi_{b}(1P)}$ & $N_{\chi_{b}(2P)}$ & $N_{\chi_{b}(3P)}$ & $N_{\chi_{b}(1P)}$ & $N_{\chi_{b}(2P)}$ & $N_{\chi_{b}(3P)}$ & $N_{\chi_{b}(1P)}$ & $N_{\chi_{b}(2P)}$ & $N_{\chi_{b}(3P)}$\\
\midrule
$\lambda=0.0$ & 19.2 & -5.6 & --- & 16.6 & -13.5 & --- & 26.5 & -7.5 & --- & 22.4 & -5.9 & ---\\
$\lambda=0.1$ & 14.0 & -4.5 & --- & 11.8 & -10.7 & --- & 20.1 & -6.3 & --- & 16.7 & -5.0 & ---\\
$\lambda=0.2$ & 9.4 & -4.0 & --- & 7.6 & -8.2 & --- & 14.0 & -5.0 & --- & 11.3 & -4.0 & ---\\

\rule{0pt}{4ex}$\lambda=0.3$ & 5.4 & -2.6 & --- & 4.1 & -5.4 & --- & 8.5 & -3.3 & --- & 6.6 & -2.6 & ---\\
$\lambda=0.4$ & 2.2 & -1.3 & --- & 1.4 & -2.5 & --- & 3.7 & -1.6 & --- & 2.8 & -1.2 & ---\\
$\lambda=0.5$ & 0.0 & 0.0 & --- & 0.0 & 0.0 & --- & 0.0 & 0.0 & --- & 0.0 & 0.0 & ---\\
$\lambda=0.6$ & -1.1 & 1.3 & --- & -0.5 & 3.4 & --- & -2.4 & 1.4 & --- & -1.6 & 1.0 & ---\\
$\lambda=0.7$ & -0.9 & 2.5 & --- & 0.5 & 3.4 & --- & -3.6 & 2.6 & --- & -1.9 & 1.8 & ---\\

\rule{0pt}{4ex}$\lambda=0.8$ & 0.4 & 3.3 & --- & 2.5 & 3.1 & --- & -3.5 & 3.4 & --- & -1.1 & 2.3 & ---\\
$\lambda=0.9$ & 2.8 & 3.7 & --- & 5.5 & 0.8 & --- & -2.4 & 3.9 & --- & 1.0 & 2.3 & ---\\
$\lambda=1.0$ & 6.3 & 3.9 & --- & 9.1 & -2.5 & --- & -0.1 & 4.1 & --- & 3.9 & 2.2 & ---\\
\bottomrule
\end{tabular}
} % scalebox

} % subtable
\subtable[$10 < p_T^{\Y1S} < 18 \gevc$] {
\scalebox{0.65}{
\begin{tabular}{lrrrrrrrrrrrr}\toprule
 & \multicolumn{12}{c}{$\Upsilon(1S)$ transverse momentum intervals, \gevc}\\
 & \multicolumn{6}{c}{10 -- 14} & \multicolumn{6}{c}{14 -- 18}\\
\cmidrule(r){2-7}\cmidrule(r){8-13}
 & \multicolumn{3}{c}{\sqs = 7\tev} & \multicolumn{3}{c}{\sqs = 8\tev} & \multicolumn{3}{c}{\sqs = 7\tev} & \multicolumn{3}{c}{\sqs = 8\tev}\\
\cmidrule(r){2-4}\cmidrule(r){5-7}\cmidrule(r){8-10}\cmidrule(r){11-13}
 & $N_{\chi_{b}(1P)}$ & $N_{\chi_{b}(2P)}$ & $N_{\chi_{b}(3P)}$ & $N_{\chi_{b}(1P)}$ & $N_{\chi_{b}(2P)}$ & $N_{\chi_{b}(3P)}$ & $N_{\chi_{b}(1P)}$ & $N_{\chi_{b}(2P)}$ & $N_{\chi_{b}(3P)}$ & $N_{\chi_{b}(1P)}$ & $N_{\chi_{b}(2P)}$ & $N_{\chi_{b}(3P)}$\\
\midrule
$\lambda=0.0$ & 23.7 & -1.1 & 59.2 & 21.6 & -12.5 & 62.0 & 10.9 & 0.5 & 1.7 & 11.6 & -0.1 & -14.6\\
$\lambda=0.1$ & 17.6 & -1.6 & 46.8 & 15.8 & -10.6 & 49.0 & 7.5 & 0.0 & 2.5 & 8.0 & -0.5 & -10.6\\
$\lambda=0.2$ & 11.9 & -1.7 & 33.0 & 10.3 & -8.5 & 31.7 & 4.6 & -0.2 & 2.9 & 5.1 & -0.6 & -6.7\\

\rule{0pt}{4ex}$\lambda=0.3$ & 6.8 & -1.3 & 19.2 & 5.7 & -5.3 & 19.6 & 2.4 & -0.2 & 2.6 & 2.7 & -0.5 & -3.5\\
$\lambda=0.4$ & 2.8 & -0.6 & 8.9 & 2.2 & -2.6 & 6.0 & 0.9 & -0.2 & 1.6 & 1.0 & -0.3 & -1.3\\
$\lambda=0.5$ & 0.0 & 0.0 & 0.0 & 0.0 & 0.0 & 0.0 & 0.0 & 0.0 & 0.0 & 0.0 & 0.0 & 0.0\\
$\lambda=0.6$ & -1.4 & 0.3 & -6.2 & -2.4 & 2.6 & -14.9 & -0.3 & 0.4 & -2.6 & -0.3 & 0.5 & 0.5\\
$\lambda=0.7$ & -1.5 & 0.3 & -9.9 & -2.1 & 3.3 & -17.1 & 0.1 & 1.0 & -5.4 & 0.1 & 1.1 & -0.2\\

\rule{0pt}{4ex}$\lambda=0.8$ & -0.5 & -0.1 & -12.1 & -0.4 & 3.0 & -15.5 & 1.0 & 2.0 & -8.5 & 1.1 & 2.0 & -1.1\\
$\lambda=0.9$ & 1.6 & -0.9 & -12.7 & 3.1 & 3.0 & -6.1 & 2.5 & 3.4 & -10.9 & 2.7 & 3.2 & -2.5\\
$\lambda=1.0$ & 4.6 & -1.9 & -12.5 & 6.3 & 2.2 & -4.6 & 4.3 & 5.7 & -9.5 & 4.9 & 4.6 & -4.1\\
\bottomrule
\end{tabular}
} % scalebox

} % subtable
\subtable[$18 < p_T^{\Y1S} < 40 \gevc$] {
\scalebox{0.65}{
\begin{tabular}{lrrrrrrrrrrrr}\toprule
 & \multicolumn{12}{c}{$\Upsilon(1S)$ transverse momentum intervals, \gevc}\\
 & \multicolumn{6}{c}{18 -- 22} & \multicolumn{6}{c}{22 -- 40}\\
\cmidrule(r){2-7}\cmidrule(r){8-13}
 & \multicolumn{3}{c}{\sqs = 7\tev} & \multicolumn{3}{c}{\sqs = 8\tev} & \multicolumn{3}{c}{\sqs = 7\tev} & \multicolumn{3}{c}{\sqs = 8\tev}\\
\cmidrule(r){2-4}\cmidrule(r){5-7}\cmidrule(r){8-10}\cmidrule(r){11-13}
 & $N_{\chi_{b}(1P)}$ & $N_{\chi_{b}(2P)}$ & $N_{\chi_{b}(3P)}$ & $N_{\chi_{b}(1P)}$ & $N_{\chi_{b}(2P)}$ & $N_{\chi_{b}(3P)}$ & $N_{\chi_{b}(1P)}$ & $N_{\chi_{b}(2P)}$ & $N_{\chi_{b}(3P)}$ & $N_{\chi_{b}(1P)}$ & $N_{\chi_{b}(2P)}$ & $N_{\chi_{b}(3P)}$\\
\midrule
$\lambda=0.0$ & 11.2 & 1.4 & -12.3 & 9.3 & -1.6 & -6.5 & 7.4 & 1.8 & -0.8 & 3.5 & 4.5 & 11.7\\
$\lambda=0.1$ & 7.6 & 0.7 & -9.5 & 6.1 & -1.7 & -4.4 & 4.7 & 0.8 & -0.6 & -0.6 & 3.0 & 19.0\\
$\lambda=0.2$ & 4.7 & 0.3 & -6.7 & 3.6 & -1.5 & -2.5 & 2.6 & 0.1 & -0.5 & -2.5 & 6.7 & 15.9\\

\rule{0pt}{4ex}$\lambda=0.3$ & 2.5 & -0.1 & -4.3 & 1.7 & -1.1 & -0.9 & 1.1 & -0.2 & -0.3 & -4.4 & 7.9 & 17.5\\
$\lambda=0.4$ & 0.9 & -0.2 & -2.1 & 0.5 & -0.6 & -0.5 & 0.3 & -0.1 & -0.1 & -5.4 & 8.8 & 18.7\\
$\lambda=0.5$ & 0.0 & 0.0 & 0.0 & -0.0 & -0.0 & -0.7 & 0.0 & 0.0 & 0.0 & 0.0 & 0.0 & 0.0\\
$\lambda=0.6$ & -0.2 & 0.3 & 1.9 & 0.1 & 0.6 & -1.9 & 0.2 & 0.7 & 0.4 & -4.7 & 9.8 & 20.5\\
$\lambda=0.7$ & 0.2 & 0.8 & 3.5 & 0.9 & 1.4 & -3.9 & 0.9 & 1.5 & 0.8 & -2.9 & 9.9 & 20.9\\

\rule{0pt}{4ex}$\lambda=0.8$ & 1.3 & 1.6 & 5.0 & 2.2 & 2.7 & -5.9 & 2.1 & 2.9 & 1.5 & -0.6 & 9.9 & 20.9\\
$\lambda=0.9$ & 3.0 & 2.6 & 6.4 & 4.0 & 4.4 & -7.6 & 3.8 & 4.5 & 2.3 & 2.3 & 10.2 & 20.8\\
$\lambda=1.0$ & 5.3 & 4.0 & 8.0 & 6.4 & 6.5 & -9.0 & 6.0 & 6.6 & 3.5 & 5.6 & 11.1 & 20.7\\
\bottomrule
\end{tabular}
} % scalebox

} % subtable
\label{tab:syst:lambda_ups1s}
\end{table}

\begin{table}[H]
\centering
\caption{\small $\chi_b$ yields systematic uncertainties (\%) related to \chiboneThreeP mass uncertainty in the fit model for $\chi_b(1,2,3P) \to \OneS \gamma$ decays}
\subtable[$6 < p_T^{\Y1S} < 10 \gevc$] {
\scalebox{0.6}{
\begin{tabular}{lrrrrrrrrrrrr}\toprule
 & \multicolumn{12}{c}{$\Upsilon(1S)$ transverse momentum intervals, \gevc}\\
 & \multicolumn{6}{c}{6 -- 8} & \multicolumn{6}{c}{8 -- 10}\\
\cmidrule(r){2-7}\cmidrule(r){8-13}
 & \multicolumn{3}{c}{\sqs = 7\tev} & \multicolumn{3}{c}{\sqs = 8\tev} & \multicolumn{3}{c}{\sqs = 7\tev} & \multicolumn{3}{c}{\sqs = 8\tev}\\
\cmidrule(r){2-4}\cmidrule(r){5-7}\cmidrule(r){8-10}\cmidrule(r){11-13}
 & $N_{\chi_{b}(1P)}$ & $N_{\chi_{b}(2P)}$ & $N_{\chi_{b}(3P)}$ & $N_{\chi_{b}(1P)}$ & $N_{\chi_{b}(2P)}$ & $N_{\chi_{b}(3P)}$ & $N_{\chi_{b}(1P)}$ & $N_{\chi_{b}(2P)}$ & $N_{\chi_{b}(3P)}$ & $N_{\chi_{b}(1P)}$ & $N_{\chi_{b}(2P)}$ & $N_{\chi_{b}(3P)}$\\
\midrule
$m_{\chiboneThreeP} = 10,502 \mevcc$ & 0.0 & 0.0 & --- & 0.0 & 0.0 & --- & 0.0 & 0.0 & --- & 0.0 & 0.0 & ---\\
$m_{\chiboneThreeP} = 10,518 \mevcc$ & 0.0 & 0.0 & --- & -0.0 & 0.0 & --- & 0.0 & 0.0 & --- & 0.0 & 0.0 & ---\\
\bottomrule
\end{tabular}
} % scalebox

} % subtable
\subtable[$10 < p_T^{\Y1S} < 18 \gevc$] {
\scalebox{0.6}{
\begin{tabular}{lrrrrrrrrrrrr}\toprule
 & \multicolumn{12}{c}{$\Upsilon(1S)$ transverse momentum intervals, \gevc}\\
 & \multicolumn{6}{c}{10 -- 14} & \multicolumn{6}{c}{14 -- 18}\\
\cmidrule(r){2-7}\cmidrule(r){8-13}
 & \multicolumn{3}{c}{\sqs = 7\tev} & \multicolumn{3}{c}{\sqs = 8\tev} & \multicolumn{3}{c}{\sqs = 7\tev} & \multicolumn{3}{c}{\sqs = 8\tev}\\
\cmidrule(r){2-4}\cmidrule(r){5-7}\cmidrule(r){8-10}\cmidrule(r){11-13}
 & $N_{\chi_{b}(1P)}$ & $N_{\chi_{b}(2P)}$ & $N_{\chi_{b}(3P)}$ & $N_{\chi_{b}(1P)}$ & $N_{\chi_{b}(2P)}$ & $N_{\chi_{b}(3P)}$ & $N_{\chi_{b}(1P)}$ & $N_{\chi_{b}(2P)}$ & $N_{\chi_{b}(3P)}$ & $N_{\chi_{b}(1P)}$ & $N_{\chi_{b}(2P)}$ & $N_{\chi_{b}(3P)}$\\
\midrule
$m_{\chiboneThreeP} = 10,502 \mevcc$ & 0.0 & -0.7 & -6.6 & 0.1 & -0.2 & -0.2 & 0.1 & -1.2 & -7.2 & -0.0 & 0.1 & 2.9\\
$m_{\chiboneThreeP} = 10,518 \mevcc$ & 0.2 & 2.0 & 20.5 & -2.4 & 3.7 & -8.0 & -0.5 & 3.4 & 20.3 & 0.0 & -0.2 & -4.4\\
\bottomrule
\end{tabular}
} % scalebox

} % subtable
\subtable[$18 < p_T^{\Y1S} < 40 \gevc$] {
\scalebox{0.6}{
\begin{tabular}{lrrrrrrrrrrrr}\toprule
 & \multicolumn{12}{c}{$\Upsilon(1S)$ transverse momentum intervals, \gevc}\\
 & \multicolumn{6}{c}{18 -- 22} & \multicolumn{6}{c}{22 -- 40}\\
\cmidrule(r){2-7}\cmidrule(r){8-13}
 & \multicolumn{3}{c}{\sqs = 7\tev} & \multicolumn{3}{c}{\sqs = 8\tev} & \multicolumn{3}{c}{\sqs = 7\tev} & \multicolumn{3}{c}{\sqs = 8\tev}\\
\cmidrule(r){2-4}\cmidrule(r){5-7}\cmidrule(r){8-10}\cmidrule(r){11-13}
 & $N_{\chi_{b}(1P)}$ & $N_{\chi_{b}(2P)}$ & $N_{\chi_{b}(3P)}$ & $N_{\chi_{b}(1P)}$ & $N_{\chi_{b}(2P)}$ & $N_{\chi_{b}(3P)}$ & $N_{\chi_{b}(1P)}$ & $N_{\chi_{b}(2P)}$ & $N_{\chi_{b}(3P)}$ & $N_{\chi_{b}(1P)}$ & $N_{\chi_{b}(2P)}$ & $N_{\chi_{b}(3P)}$\\
\midrule
$m_{\chiboneThreeP} = 10,502 \mevcc$ & -0.2 & 1.1 & 8.5 & 0.0 & -0.5 & -4.7 & -0.1 & 0.3 & 2.1 & -0.0 & -0.1 & 1.2\\
$m_{\chiboneThreeP} = 10,518 \mevcc$ & 0.3 & -1.7 & -13.9 & -0.2 & 1.2 & 9.2 & 0.1 & -0.5 & -2.9 & -5.7 & 9.1 & 13.9\\
\bottomrule
\end{tabular}
} % scalebox

} % subtable
\label{tab:syst:m3p_ups1s}
\end{table}

\begin{table}[H]
\centering
\caption{\small Systematic uncertainties (\%) related to $\lambda$ values in $\chi_b(2,3P) \to \TwoS \gamma$ decays}
\subtable[$18 < p_T^{\Y2S} < 24 \gevc$] {
\scalebox{0.65}{
\begin{tabular}{lrrrrrrrrrrrr}\toprule
 & \multicolumn{12}{c}{$\Upsilon(2S)$ transverse momentum intervals, \gevc}\\
 & \multicolumn{4}{c}{18 -- 22} & \multicolumn{4}{c}{18 -- 24} & \multicolumn{4}{c}{22 -- 24}\\
\cmidrule(r){2-5}\cmidrule(r){6-9}\cmidrule(r){10-13}
 & \multicolumn{2}{c}{\sqs = 7\tev} & \multicolumn{2}{c}{\sqs = 8\tev} & \multicolumn{2}{c}{\sqs = 7\tev} & \multicolumn{2}{c}{\sqs = 8\tev} & \multicolumn{2}{c}{\sqs = 7\tev} & \multicolumn{2}{c}{\sqs = 8\tev}\\
\cmidrule(r){2-3}\cmidrule(r){4-5}\cmidrule(r){6-7}\cmidrule(r){8-9}\cmidrule(r){10-11}\cmidrule(r){12-13}
 & $N_{\chi_{b}(2P)}$ & $N_{\chi_{b}(3P)}$ & $N_{\chi_{b}(2P)}$ & $N_{\chi_{b}(3P)}$ & $N_{\chi_{b}(2P)}$ & $N_{\chi_{b}(3P)}$ & $N_{\chi_{b}(2P)}$ & $N_{\chi_{b}(3P)}$ & $N_{\chi_{b}(2P)}$ & $N_{\chi_{b}(3P)}$ & $N_{\chi_{b}(2P)}$ & $N_{\chi_{b}(3P)}$\\
\midrule
$\lambda=0.0$ & 22.0 & --- & 13.9 & --- & --- & 19.1 & --- & 13.4 & 5.5 & --- & 4.3 & ---\\
$\lambda=0.1$ & 15.9 & --- & 9.4 & --- & --- & 14.1 & --- & 9.5 & 2.2 & --- & 1.3 & ---\\
$\lambda=0.2$ & 10.2 & --- & 5.9 & --- & --- & 9.3 & --- & 5.6 & -0.1 & --- & -0.6 & ---\\

\rule{0pt}{4ex}$\lambda=0.3$ & 5.7 & --- & 3.2 & --- & --- & 5.5 & --- & 3.0 & -1.4 & --- & -1.5 & ---\\
$\lambda=0.4$ & 2.3 & --- & 1.1 & --- & --- & 2.2 & --- & 1.1 & -1.3 & --- & -1.2 & ---\\
$\lambda=0.5$ & 0.0 & --- & 0.0 & --- & --- & 0.0 & --- & -0.1 & -0.0 & --- & 0.1 & ---\\
$\lambda=0.6$ & -0.9 & --- & 0.0 & --- & --- & -1.0 & --- & 0.2 & 2.5 & --- & 2.4 & ---\\
$\lambda=0.7$ & -0.7 & --- & 1.0 & --- & --- & -1.3 & --- & 2.1 & 6.4 & --- & 5.9 & ---\\

\rule{0pt}{4ex}$\lambda=0.8$ & 0.6 & --- & 3.1 & --- & --- & -0.3 & --- & 4.0 & 11.3 & --- & 10.3 & ---\\
$\lambda=0.9$ & 2.9 & --- & 6.1 & --- & --- & 1.8 & --- & 6.9 & 17.5 & --- & 15.8 & ---\\
$\lambda=1.0$ & 6.2 & --- & 9.9 & --- & --- & 4.3 & --- & 10.8 & 24.5 & --- & 22.0 & ---\\
\bottomrule
\end{tabular}
} % scalebox

} % subtable
\subtable[$24 < p_T^{\Y2S} < 40 \gevc$] {
\scalebox{0.65}{
\begin{tabular}{lrrrr}\toprule
 & \multicolumn{4}{c}{$\Upsilon(2S)$ transverse momentum intervals, \gevc}\\
 & \multicolumn{4}{c}{24 -- 40}\\
\cmidrule(r){2-5}
 & \multicolumn{2}{c}{\sqs = 7\tev} & \multicolumn{2}{c}{\sqs = 8\tev}\\
\cmidrule(r){2-3}\cmidrule(r){4-5}
 & $N_{\chi_{b}(2P)}$ & $N_{\chi_{b}(3P)}$ & $N_{\chi_{b}(2P)}$ & $N_{\chi_{b}(3P)}$\\
\midrule
$\lambda=0.0$ & 12.6 & 4.9 & 14.2 & -8.9\\
$\lambda=0.1$ & 8.3 & 3.3 & 9.8 & -8.9\\
$\lambda=0.2$ & 4.7 & 1.9 & 6.0 & -8.0\\

\rule{0pt}{4ex}$\lambda=0.3$ & 2.1 & 0.8 & 3.2 & -6.3\\
$\lambda=0.4$ & 0.5 & 0.1 & 1.2 & -3.5\\
$\lambda=0.5$ & 0.0 & 0.0 & 0.0 & 0.0\\
$\lambda=0.6$ & 0.9 & -0.1 & -0.2 & 4.4\\
$\lambda=0.7$ & 2.7 & 0.4 & 0.3 & 9.5\\

\rule{0pt}{4ex}$\lambda=0.8$ & 5.7 & 1.3 & 1.7 & 15.1\\
$\lambda=0.9$ & 9.6 & 2.6 & 4.0 & 21.2\\
$\lambda=1.0$ & 14.5 & 4.0 & 7.1 & 27.6\\
\bottomrule
\end{tabular}
} % scalebox

} % subtable
\label{tab:syst:lambda_ups2s}
\end{table}

\begin{table}[H]
\centering
\caption{\small Systematic uncertainties (\%) related to \chiboneThreeP mass uncertainty in $\chi_b(2,3P) \to \TwoS \gamma$ decays}
\subtable[$18 < p_T^{\Y2S} < 24 \gevc$] {
\scalebox{0.6}{
\begin{tabular}{lrrrrrrrrrrrr}\toprule
 & \multicolumn{12}{c}{$\Upsilon(2S)$ transverse momentum intervals, \gevc}\\
 & \multicolumn{4}{c}{18 -- 22} & \multicolumn{4}{c}{18 -- 24} & \multicolumn{4}{c}{22 -- 24}\\
\cmidrule(r){2-5}\cmidrule(r){6-9}\cmidrule(r){10-13}
 & \multicolumn{2}{c}{\sqs = 7\tev} & \multicolumn{2}{c}{\sqs = 8\tev} & \multicolumn{2}{c}{\sqs = 7\tev} & \multicolumn{2}{c}{\sqs = 8\tev} & \multicolumn{2}{c}{\sqs = 7\tev} & \multicolumn{2}{c}{\sqs = 8\tev}\\
\cmidrule(r){2-3}\cmidrule(r){4-5}\cmidrule(r){6-7}\cmidrule(r){8-9}\cmidrule(r){10-11}\cmidrule(r){12-13}
 & $N_{\chi_{b}(2P)}$ & $N_{\chi_{b}(3P)}$ & $N_{\chi_{b}(2P)}$ & $N_{\chi_{b}(3P)}$ & $N_{\chi_{b}(2P)}$ & $N_{\chi_{b}(3P)}$ & $N_{\chi_{b}(2P)}$ & $N_{\chi_{b}(3P)}$ & $N_{\chi_{b}(2P)}$ & $N_{\chi_{b}(3P)}$ & $N_{\chi_{b}(2P)}$ & $N_{\chi_{b}(3P)}$\\
\midrule
$m_{\chiboneThreeP} = 10,502 \mevcc$ & 0.0 & --- & -0.1 & --- & --- & -4.2 & --- & 0.0 & -0.3 & --- & 0.0 & ---\\
$m_{\chiboneThreeP} = 10,518 \mevcc$ & 0.1 & --- & 0.2 & --- & --- & 10.5 & --- & 5.3 & 0.9 & --- & -0.3 & ---\\
\bottomrule
\end{tabular}
} % scalebox

} % subtable
\subtable[$24 < p_T^{\Y2S} < 40 \gevc$] {
\scalebox{0.6}{
\begin{tabular}{lrrrr}\toprule
 & \multicolumn{4}{c}{$\Upsilon(2S)$ transverse momentum intervals, \gevc}\\
 & \multicolumn{4}{c}{24 -- 40}\\
\cmidrule(r){2-5}
 & \multicolumn{2}{c}{\sqs = 7\tev} & \multicolumn{2}{c}{\sqs = 8\tev}\\
\cmidrule(r){2-3}\cmidrule(r){4-5}
 & $N_{\chi_{b}(2P)}$ & $N_{\chi_{b}(3P)}$ & $N_{\chi_{b}(2P)}$ & $N_{\chi_{b}(3P)}$\\
\midrule
$m_{\chiboneThreeP} = 10,502 \mevcc$ & 0.0 & 0.0 & 0.1 & 13.5\\
$m_{\chiboneThreeP} = 10,518 \mevcc$ & 0.5 & 11.2 & -0.1 & -16.6\\
\bottomrule
\end{tabular}
} % scalebox

} % subtable
\label{tab:syst:m3p_ups2s}
\end{table}

\begin{table}[H]
\caption{\small Systematic uncertainties (\%) related to $\lambda$ values in $\chi_b(3P) \to \ThreeS \gamma$ decays}
\centering
\scalebox{1}{
\begin{tabular}{lrr}\toprule
 & \multicolumn{2}{c}{$\Upsilon(1S)$ transverse momentum intervals, \gevc}\\
 & \multicolumn{2}{c}{27 -- 40}\\
\cmidrule(r){2-3}
 & \sqs = 7\tev & \sqs = 8\tev\\
\cmidrule(r){2-2}\cmidrule(r){3-3}
 & $N_{\chi_{b}(3P)}$ & $N_{\chi_{b}(3P)}$\\
\midrule
$\lambda=0.0$ & 2.9 & -10.5\\
$\lambda=0.1$ & -0.1 & 100.0\\
$\lambda=0.2$ & -2.4 & -14.6\\

\rule{0pt}{4ex}$\lambda=0.3$ & -1.9 & -15.8\\
$\lambda=0.4$ & -4.8 & -16.5\\
$\lambda=0.5$ & -5.0 & -8.9\\
$\lambda=0.6$ & -5.2 & -16.8\\
$\lambda=0.7$ & -4.5 & -7.5\\

\rule{0pt}{4ex}$\lambda=0.8$ & -2.9 & -6.2\\
$\lambda=0.9$ & -0.5 & -6.1\\
$\lambda=1.0$ & 2.9 & -10.6\\
\bottomrule
\end{tabular}
} % scalebox
\label{tab:syst:lambda_ups3s}
\end{table}

\begin{table}[H]
\caption{\small Systematic uncertainties (\%) related to \chiboneThreeP mass uncertanty in $\chi_b(3P) \to \ThreeS \gamma$ decays}
\centering
\scalebox{1}{
\begin{tabular}{lrr}\toprule
 & \multicolumn{2}{c}{$\Upsilon(1S)$ transverse momentum intervals, \gevc}\\
 & \multicolumn{2}{c}{27 -- 40}\\
\cmidrule(r){2-3}
 & \sqs = 7\tev & \sqs = 8\tev\\
\cmidrule(r){2-2}\cmidrule(r){3-3}
 & $N_{\chi_{b}(3P)}$ & $N_{\chi_{b}(3P)}$\\
\midrule
$m_{\chiboneThreeP} = 10{,}502 \mevcc$ & 83.4 & 13.7\\
$m_{\chiboneThreeP} = 10{,}518 \mevcc$ & -3.1 & -26.7\\
\bottomrule
\end{tabular}
} % scalebox
\label{tab:syst:m3p_ups3s}
\end{table}

% \begin{table}[H]
\centering
\caption{\small Systematic uncertainties (\%) related to $\lambda$ values in $\chi_b(1,2,3P) \to \OneS \gamma$ decays}
\subtable[$6 < p_T^{\Y1S} < 12 \gevc$] {
\scalebox{0.5}{
\begin{tabular}{lrrrrrrrrrrrrrrrrrr}\toprule
 & \multicolumn{18}{c}{$\Upsilon(1S)$ transverse momentum range, \gevc}\\
 & \multicolumn{6}{c}{6 --- 8} & \multicolumn{6}{c}{8 --- 10} & \multicolumn{6}{c}{10 --- 12}\\
\cmidrule(r){2-7}\cmidrule(r){8-13}\cmidrule(r){14-19}
 & \multicolumn{3}{c}{\sqs=7\tev} & \multicolumn{3}{c}{\sqs=8\tev} & \multicolumn{3}{c}{\sqs=7\tev} & \multicolumn{3}{c}{\sqs=8\tev} & \multicolumn{3}{c}{\sqs=7\tev} & \multicolumn{3}{c}{\sqs=8\tev}\\
\cmidrule(r){2-4}\cmidrule(r){5-7}\cmidrule(r){8-10}\cmidrule(r){11-13}\cmidrule(r){14-16}\cmidrule(r){17-19}
 & $N_{\chi_{b}(1P)}$ & $N_{\chi_{b}(2P)}$ & $N_{\chi_{b}(3P)}$ & $N_{\chi_{b}(1P)}$ & $N_{\chi_{b}(2P)}$ & $N_{\chi_{b}(3P)}$ & $N_{\chi_{b}(1P)}$ & $N_{\chi_{b}(2P)}$ & $N_{\chi_{b}(3P)}$ & $N_{\chi_{b}(1P)}$ & $N_{\chi_{b}(2P)}$ & $N_{\chi_{b}(3P)}$ & $N_{\chi_{b}(1P)}$ & $N_{\chi_{b}(2P)}$ & $N_{\chi_{b}(3P)}$ & $N_{\chi_{b}(1P)}$ & $N_{\chi_{b}(2P)}$ & $N_{\chi_{b}(3P)}$\\
\midrule
$\lambda=0.0$ & 19.0 & -6.0 & --- & 17.0 & -13.0 & --- & 27.0 & -7.0 & --- & 22.0 & -6.0 & --- & 27.0 & -7.0 & --- & 20.0 & -27.0 & ---\\
$\lambda=0.1$ & 14.0 & -5.0 & --- & 12.0 & -11.0 & --- & 20.0 & -6.0 & --- & 17.0 & -5.0 & --- & 20.0 & -7.0 & --- & 14.0 & -23.0 & ---\\
$\lambda=0.2$ & 9.0 & -4.0 & --- & 8.0 & -8.0 & --- & 14.0 & -5.0 & --- & 11.0 & -4.0 & --- & 13.0 & -8.0 & --- & 4.0 & -18.0 & ---\\

\rule{0pt}{4ex}$\lambda=0.3$ & 5.0 & -3.0 & --- & 4.0 & -5.0 & --- & 8.0 & -3.0 & --- & 7.0 & -3.0 & --- & 7.0 & -8.0 & --- & 1.0 & -14.0 & ---\\
$\lambda=0.4$ & 2.0 & -1.0 & --- & 1.0 & -3.0 & --- & 4.0 & -2.0 & --- & 3.0 & -1.0 & --- & 2.0 & -8.0 & --- & 0.0 & -10.0 & ---\\
$\lambda=0.5$ & 0.0 & 0.0 & --- & 0.0 & 0.0 & --- & 0.0 & 0.0 & --- & 0.0 & 0.0 & --- & -2.0 & -7.0 & --- & -3.0 & -7.0 & ---\\
$\lambda=0.6$ & -1.0 & 1.0 & --- & -1.0 & 3.0 & --- & -2.0 & 1.0 & --- & -2.0 & 1.0 & --- & -5.0 & -7.0 & --- & -2.0 & -7.0 & ---\\
$\lambda=0.7$ & -1.0 & 3.0 & --- & 0.0 & 3.0 & --- & -4.0 & 3.0 & --- & -2.0 & 2.0 & --- & -6.0 & -7.0 & --- & -2.0 & -5.0 & ---\\

\rule{0pt}{4ex}$\lambda=0.8$ & 0.0 & 3.0 & --- & 3.0 & 3.0 & --- & -4.0 & 3.0 & --- & -1.0 & 2.0 & --- & -6.0 & -7.0 & --- & -1.0 & -6.0 & ---\\
$\lambda=0.9$ & 3.0 & 4.0 & --- & 5.0 & 1.0 & --- & -2.0 & 4.0 & --- & 1.0 & 2.0 & --- & -5.0 & -7.0 & --- & 2.0 & -7.0 & ---\\
$\lambda=1.0$ & 6.0 & 4.0 & --- & 9.0 & -3.0 & --- & -0.0 & 4.0 & --- & 4.0 & 2.0 & --- & -2.0 & -7.0 & --- & 5.0 & -9.0 & ---\\
\bottomrule
\end{tabular}
} % scalebox

} % subtable
\subtable[$12 < p_T^{\Y1S} < 18 \gevc$] {
\scalebox{0.5}{
\begin{tabular}{lrrrrrrrrrrrrrrrrrr}\toprule
 & \multicolumn{18}{c}{$\Upsilon(1S)$ transverse momentum range, \gevc}\\
 & \multicolumn{6}{c}{12 --- 14} & \multicolumn{6}{c}{10 --- 14} & \multicolumn{6}{c}{14 --- 18}\\
\cmidrule(r){2-7}\cmidrule(r){8-13}\cmidrule(r){14-19}
 & \multicolumn{3}{c}{\sqs=7\tev} & \multicolumn{3}{c}{\sqs=8\tev} & \multicolumn{3}{c}{\sqs=7\tev} & \multicolumn{3}{c}{\sqs=8\tev} & \multicolumn{3}{c}{\sqs=7\tev} & \multicolumn{3}{c}{\sqs=8\tev}\\
\cmidrule(r){2-4}\cmidrule(r){5-7}\cmidrule(r){8-10}\cmidrule(r){11-13}\cmidrule(r){14-16}\cmidrule(r){17-19}
 & $N_{\chi_{b}(1P)}$ & $N_{\chi_{b}(2P)}$ & $N_{\chi_{b}(3P)}$ & $N_{\chi_{b}(1P)}$ & $N_{\chi_{b}(2P)}$ & $N_{\chi_{b}(3P)}$ & $N_{\chi_{b}(1P)}$ & $N_{\chi_{b}(2P)}$ & $N_{\chi_{b}(3P)}$ & $N_{\chi_{b}(1P)}$ & $N_{\chi_{b}(2P)}$ & $N_{\chi_{b}(3P)}$ & $N_{\chi_{b}(1P)}$ & $N_{\chi_{b}(2P)}$ & $N_{\chi_{b}(3P)}$ & $N_{\chi_{b}(1P)}$ & $N_{\chi_{b}(2P)}$ & $N_{\chi_{b}(3P)}$\\
\midrule
$\lambda=0.0$ & 10.0 & 3.0 & 4.0 & 12.0 & -2.0 & -7.0 & 24.0 & -1.0 & 59.0 & 22.0 & -12.0 & 62.0 & 11.0 & 0.0 & 2.0 & 12.0 & -0.0 & -15.0\\
$\lambda=0.1$ & 6.0 & 3.0 & 7.0 & 9.0 & -2.0 & -4.0 & 18.0 & -2.0 & 47.0 & 16.0 & -11.0 & 49.0 & 7.0 & 0.0 & 2.0 & 8.0 & -0.0 & -11.0\\
$\lambda=0.2$ & 4.0 & 2.0 & 8.0 & 6.0 & -2.0 & -3.0 & 12.0 & -2.0 & 33.0 & 10.0 & -8.0 & 32.0 & 5.0 & -0.0 & 3.0 & 5.0 & -1.0 & -7.0\\

\rule{0pt}{4ex}$\lambda=0.3$ & 2.0 & 1.0 & 8.0 & 3.0 & -2.0 & -1.0 & 7.0 & -1.0 & 19.0 & 6.0 & -5.0 & 20.0 & 2.0 & -0.0 & 3.0 & 3.0 & -1.0 & -4.0\\
$\lambda=0.4$ & 1.0 & 1.0 & 5.0 & 1.0 & -1.0 & -0.0 & 3.0 & -1.0 & 9.0 & 2.0 & -3.0 & 6.0 & 1.0 & -0.0 & 2.0 & 1.0 & -0.0 & -1.0\\
$\lambda=0.5$ & 0.0 & 0.0 & 0.0 & 0.0 & 0.0 & 0.0 & 0.0 & 0.0 & 0.0 & 0.0 & 0.0 & 0.0 & 0.0 & 0.0 & 0.0 & 0.0 & 0.0 & 0.0\\
$\lambda=0.6$ & 0.0 & -0.0 & -7.0 & -1.0 & 1.0 & -0.0 & -1.0 & 0.0 & -6.0 & -2.0 & 3.0 & -15.0 & -0.0 & 0.0 & -3.0 & -0.0 & 0.0 & 1.0\\
$\lambda=0.7$ & 1.0 & -1.0 & -15.0 & -1.0 & 3.0 & -0.0 & -2.0 & 0.0 & -10.0 & -2.0 & 3.0 & -17.0 & 0.0 & 1.0 & -5.0 & 0.0 & 1.0 & -0.0\\

\rule{0pt}{4ex}$\lambda=0.8$ & 2.0 & -1.0 & -25.0 & -0.0 & 4.0 & -1.0 & -0.0 & -0.0 & -12.0 & -0.0 & 3.0 & -16.0 & 1.0 & 2.0 & -9.0 & 1.0 & 2.0 & -1.0\\
$\lambda=0.9$ & 4.0 & -1.0 & -35.0 & 1.0 & 7.0 & -2.0 & 2.0 & -1.0 & -13.0 & 3.0 & 3.0 & -6.0 & 2.0 & 3.0 & -11.0 & 3.0 & 3.0 & -2.0\\
$\lambda=1.0$ & 6.0 & 0.0 & -45.0 & 3.0 & 9.0 & -3.0 & 5.0 & -2.0 & -13.0 & 6.0 & 2.0 & -5.0 & 4.0 & 6.0 & -9.0 & 5.0 & 5.0 & -4.0\\
\bottomrule
\end{tabular}
} % scalebox

} % subtable
\subtable[$18 < p_T^{\Y1S} < 40 \gevc$] {
\scalebox{0.5}{
\begin{tabular}{lrrrrrrrrrrrr}\toprule
 & \multicolumn{12}{c}{$\Upsilon(1S)$ transverse momentum range, \gevc}\\
 & \multicolumn{6}{c}{18 --- 22} & \multicolumn{6}{c}{22 --- 40}\\
\cmidrule(r){2-7}\cmidrule(r){8-13}
 & \multicolumn{3}{c}{\sqs=7\tev} & \multicolumn{3}{c}{\sqs=8\tev} & \multicolumn{3}{c}{\sqs=7\tev} & \multicolumn{3}{c}{\sqs=8\tev}\\
\cmidrule(r){2-4}\cmidrule(r){5-7}\cmidrule(r){8-10}\cmidrule(r){11-13}
 & $N_{\chi_{b}(1P)}$ & $N_{\chi_{b}(2P)}$ & $N_{\chi_{b}(3P)}$ & $N_{\chi_{b}(1P)}$ & $N_{\chi_{b}(2P)}$ & $N_{\chi_{b}(3P)}$ & $N_{\chi_{b}(1P)}$ & $N_{\chi_{b}(2P)}$ & $N_{\chi_{b}(3P)}$ & $N_{\chi_{b}(1P)}$ & $N_{\chi_{b}(2P)}$ & $N_{\chi_{b}(3P)}$\\
\midrule
$\lambda=0.0$ & 11.0 & 1.0 & -12.0 & 9.0 & -2.0 & -6.0 & 7.0 & 2.0 & -1.0 & 3.0 & 5.0 & 12.0\\
$\lambda=0.1$ & 8.0 & 1.0 & -9.0 & 6.0 & -2.0 & -4.0 & 5.0 & 1.0 & -1.0 & -1.0 & 3.0 & 19.0\\
$\lambda=0.2$ & 5.0 & 0.0 & -7.0 & 4.0 & -2.0 & -2.0 & 3.0 & 0.0 & -1.0 & -3.0 & 7.0 & 16.0\\

\rule{0pt}{4ex}$\lambda=0.3$ & 2.0 & -0.0 & -4.0 & 2.0 & -1.0 & -0.0 & 1.0 & -0.0 & -0.0 & -4.0 & 8.0 & 17.0\\
$\lambda=0.4$ & 1.0 & -0.0 & -2.0 & 1.0 & -1.0 & 0.0 & 0.0 & -0.0 & -0.0 & -5.0 & 9.0 & 19.0\\
$\lambda=0.5$ & 0.0 & 0.0 & 0.0 & 0.0 & 0.0 & 0.0 & 0.0 & 0.0 & 0.0 & 0.0 & 0.0 & 0.0\\
$\lambda=0.6$ & -0.0 & 0.0 & 2.0 & 0.0 & 1.0 & -1.0 & 0.0 & 1.0 & 0.0 & -5.0 & 10.0 & 20.0\\
$\lambda=0.7$ & 0.0 & 1.0 & 3.0 & 1.0 & 1.0 & -3.0 & 1.0 & 2.0 & 1.0 & -3.0 & 10.0 & 21.0\\

\rule{0pt}{4ex}$\lambda=0.8$ & 1.0 & 2.0 & 5.0 & 2.0 & 3.0 & -5.0 & 2.0 & 3.0 & 2.0 & -1.0 & 10.0 & 21.0\\
$\lambda=0.9$ & 3.0 & 3.0 & 6.0 & 4.0 & 4.0 & -7.0 & 4.0 & 4.0 & 2.0 & 2.0 & 10.0 & 21.0\\
$\lambda=1.0$ & 5.0 & 4.0 & 8.0 & 6.0 & 7.0 & -8.0 & 6.0 & 7.0 & 4.0 & 6.0 & 11.0 & 21.0\\
\bottomrule
\end{tabular}
} % scalebox

} % subtable
\label{tab:syst:lambda_ups1s}
\end{table}

% \begin{table}[H]
\centering
\caption{\small Systematic uncertainties due to $\lambda$ values in
$\chi_b(2,3P) \to \TwoS \gamma$ decays.
}

\scalebox{1}{
\begin{tabular}{crrrrrr}\toprule
& \multicolumn{ 6 }{c}{\TwoS transverse momentum intervals} \\
 & & \multicolumn{2}{c}{$18 < p_T < 25 \gevc$} & & \multicolumn{2}{c}{$25 < p_T < 40 \gevc$} \\
\cmidrule{3-4}\cmidrule{6-7}
$\lambda$ / Change (\%)  & & $N_{\chibTwoP}$ & $N_{\chibThreeP}$ & & $N_{\chibTwoP}$ & $N_{\chibThreeP}$ \\
\midrule
0.0  &  & 12 & 10 &  & 12 & -4\\
0.1  &  & 7 & 6 &  & 8 & -3\\
0.2  &  & 4 & 3 &  & 6 & -7\\
0.3  &  & 1 & 1 &  & 3 & -5\\
0.4  &  & 0 & 0 &  & 1 & -3\\
0.5  &  & 0 & 0 &  & 0 & 0\\
0.6  &  & 1 & 1 &  & -2 & 11\\
0.7  &  & 4 & 4 &  & 1 & 8\\
0.8  &  & 7 & 8 &  & 3 & 12\\
0.9  &  & 11 & 13 &  & 6 & 17\\
1.0  &  & 17 & 19 &  & 10 & 22\\
\bottomrule
\end{tabular}
}

\label{tab:syst:lambda2s}
\end{table}
% \begin{table}[H]
\centering
\caption{\small Systematic uncertainties due to $m(\chiboneOneP)$ mass range
$\chi_b(1,2,3P) \to \OneS \gamma$ decays.
}
\subtable[$6 < p_T < 14 \gevcc$]{
\scalebox{0.5}{
\begin{tabular}{crrrrrrrrrrrrrrrr}\toprule
& \multicolumn{ 16 }{c}{\OneS transverse momentum intervals} \\
 & & \multicolumn{3}{c}{$6 < p_T < 8 \gevc$} & & \multicolumn{3}{c}{$8 < p_T < 10 \gevc$} & & \multicolumn{3}{c}{$10 < p_T < 12 \gevc$} & & \multicolumn{3}{c}{$12 < p_T < 14 \gevc$} \\
\cmidrule{3-5}\cmidrule{7-9}\cmidrule{11-13}\cmidrule{15-17}
$\m(\chiboneOneP)$ / Change (\%)  & & $N_{\chibOneP}$ & $N_{\chibTwoP}$ & $N_{\chibThreeP}$ & & $N_{\chibOneP}$ & $N_{\chibTwoP}$ & $N_{\chibThreeP}$ & & $N_{\chibOneP}$ & $N_{\chibTwoP}$ & $N_{\chibThreeP}$ & & $N_{\chibOneP}$ & $N_{\chibTwoP}$ & $N_{\chibThreeP}$ \\
\midrule
9.885 \gevcc ($min\left[m(\chiboneOneP)\right]$) &  & -1 & 0 & -  &  & -6 & 2 & -  &  & 0 & 5 & -  &  & -2 & 6 & 1\\
9.896 \gevcc ($max\left[m(\chiboneOneP)\right]$) &  & 4 & -1 & -  &  & 6 & -1 & -  &  & 4 & -5 & -  &  & 3 & -2 & -1\\
\bottomrule
\end{tabular}
}
}
\subtable[$14 < p_T < 30 \gevcc$]{
\scalebox{0.5}{
\begin{tabular}{crrrrrrrrrrrrrrrr}\toprule
& \multicolumn{ 16 }{c}{\OneS transverse momentum intervals} \\
 & & \multicolumn{3}{c}{$14 < p_T < 18 \gevc$} & & \multicolumn{3}{c}{$18 < p_T < 22 \gevc$} & & \multicolumn{3}{c}{$22 < p_T < 30 \gevc$} & & \multicolumn{3}{c}{$18 < p_T < 30 \gevc$} \\
\cmidrule{3-5}\cmidrule{7-9}\cmidrule{11-13}\cmidrule{15-17}
$\m(\chiboneOneP)$ / Change (\%)  & & $N_{\chibOneP}$ & $N_{\chibTwoP}$ & $N_{\chibThreeP}$ & & $N_{\chibOneP}$ & $N_{\chibTwoP}$ & $N_{\chibThreeP}$ & & $N_{\chibOneP}$ & $N_{\chibTwoP}$ & $N_{\chibThreeP}$ & & $N_{\chibOneP}$ & $N_{\chibTwoP}$ & $N_{\chibThreeP}$ \\
\midrule
9.885 \gevcc ($min\left[m(\chiboneOneP)\right]$) &  & -1 & 2 & 4 &  & 1 & 4 & -2 &  & 3 & 5 & 8 &  & 2 & 5 & 2\\
9.896 \gevcc ($max\left[m(\chiboneOneP)\right]$) &  & 2 & 0 & -3 &  & 1 & 0 & 1 &  & 0 & -2 & -5 &  & 1 & -2 & -2\\
\bottomrule
\end{tabular}
}
}
\label{tab:syst:m1p}
\end{table}
% \begin{table}[H]
\centering
\caption{\small Systematic uncertainties due to $m(\chiboneTwoP)$ mass range
$\chi_b(2,3P) \to \TwoS \gamma$ decays.
}

\scalebox{1}{
\begin{tabular}{crrrrrr}\toprule
& \multicolumn{ 6 }{c}{\TwoS transverse momentum intervals} \\
 & & \multicolumn{2}{c}{$18 < p_T < 25 \gevc$} & & \multicolumn{2}{c}{$25 < p_T < 40 \gevc$} \\
\cmidrule{3-4}\cmidrule{6-7}
$\m(\chiboneTwoP)$ / Change (\%)  & & $N_{\chibTwoP}$ & $N_{\chibThreeP}$ & & $N_{\chibTwoP}$ & $N_{\chibThreeP}$ \\
\midrule
10.245 \gevcc ($min\left[m(\chiboneTwoP)\right]$) &  & 4 & 4 &  & -2 & 19\\
10.255 \gevcc ($max\left[m(\chiboneTwoP)\right]$) &  & 3 & 2 &  & 8 & -11\\
\bottomrule
\end{tabular}
}

\label{tab:syst:m2p}
\end{table}

% \begin{table}[H]
\centering
\caption{\small Systematic uncertainties due to $m(\chiboneThreeP)$ mass range in
$\chi_b(1,2,3P) \to \OneS \gamma$ decays.
}

\scalebox{0.5}{
\begin{tabular}{crrrrrrrrrrrrrrrr}\toprule
& \multicolumn{ 16 }{c}{\OneS transverse momentum intervals} \\
 & & \multicolumn{3}{c}{$14 < p_T < 18 \gevc$} & & \multicolumn{3}{c}{$18 < p_T < 22 \gevc$} & & \multicolumn{3}{c}{$22 < p_T < 30 \gevc$} & & \multicolumn{3}{c}{$18 < p_T < 30 \gevc$} \\
\cmidrule{3-5}\cmidrule{7-9}\cmidrule{11-13}\cmidrule{15-17}
$\m(\chiboneThreeP)$ / Change (\%)  & & $N_{\chibOneP}$ & $N_{\chibTwoP}$ & $N_{\chibThreeP}$ & & $N_{\chibOneP}$ & $N_{\chibTwoP}$ & $N_{\chibThreeP}$ & & $N_{\chibOneP}$ & $N_{\chibTwoP}$ & $N_{\chibThreeP}$ & & $N_{\chibOneP}$ & $N_{\chibTwoP}$ & $N_{\chibThreeP}$ \\
\midrule
10.503 \gevcc ($min\left[m(\chiboneThreeP)\right]$) &  & 0 & -1 & -6 &  & 0 & 1 & 7 &  & 0 & 0 & 1 &  & 0 & 1 & 5\\
10.517 \gevcc ($max\left[m(\chiboneThreeP)\right]$) &  & 0 & 3 & 19 &  & 0 & -2 & -14 &  & 0 & 0 & -2 &  & 0 & -1 & -9\\
\bottomrule
\end{tabular}
}

\label{tab:syst:m3p1s}
\end{table}
% \begin{table}[H]
\centering
\caption{\small Systematic uncertainties due to $m(\chiboneThreeP)$ mass range in
$\chi_b(2,3P) \to \TwoS \gamma$ decays.
}

\scalebox{1}{
\begin{tabular}{crrrrrr}\toprule
& \multicolumn{ 6 }{c}{\TwoS transverse momentum intervals} \\
 & & \multicolumn{2}{c}{$18 < p_T < 25 \gevc$} & & \multicolumn{2}{c}{$25 < p_T < 40 \gevc$} \\
\cmidrule{3-4}\cmidrule{6-7}
$\m(\chiboneThreeP)$ / Change (\%)  & & $N_{\chibTwoP}$ & $N_{\chibThreeP}$ & & $N_{\chibTwoP}$ & $N_{\chibThreeP}$ \\
\midrule
10.503 \gevcc ($min\left[m(\chiboneThreeP)\right]$) &  & 0 & -1 &  & 0 & -4\\
10.517 \gevcc ($max\left[m(\chiboneThreeP)\right]$) &  & 0 & 4 &  & 0 & -8\\
\bottomrule
\end{tabular}
}

\label{tab:syst:m3p2s}
\end{table}


\subsection{\chib polarization}
\label{sec:syst:pol}

The prompt \chib polarization is unknown. The simulated \chib mesons are
unpolarized and all the efficiencies given in the previous sections are
therefore determined under the assumption that the $\chi_{b1}$ and the
$\chi_{b2}$ mesons are produced unpolarized. The photon and $\Upsilon$ momentum
distributions depend on the polarization of the \chib state and the same is
true for the ratio of efficiencies. The correction factors for the ratio of
efficiencies under other polarization scenarios are derived here.

The angular distribution of the $\chib \to \Upsilon \gamma$ decay is described
by the angles $\theta_{\Upsilon}$ , $\theta_{\chib}$ and $\phi$ where:
$\theta_{\Upsilon}$ is the angle between the directions of the positive muon in
the $\Upsilon$ rest frame and the $\Upsilon$ in the $\chib$ rest frame;
$\theta_{\chib}$ is the angle between the directions of the $\Upsilon$ in the
\chib rest frame and the \chib in the laboratory frame; $\phi$ is the angle
between the $\Upsilon$ decay plane in the \chib rest frame and the plane formed
by the \chib direction in the laboratory frame and the direction of the
$\Upsilon$ in the \chib rest frame. The angular distributions of the
\chib states depend on $m_{\chi_{bJ}}$ , which is the azimuthal angular
momentum quantum number of the $\chi_{bJ}$ state. For each simulated event in
the unpolarized sample, a weight is calculated from the values of described
angles in the various polarization hypotheses and the ratio of efficiencies is
deduced for each ($m_{\chi_{b1}}$, $m_{\chi_{b2}}$) polarization combination.

As an example,
Figures~\ref{sec:syst:polarization:angles_chib11p_ups1s}-\ref{sec:syst:polarization:angles_chib21p_ups1s}
show the angular distributions distribution in the $\chi_{b1,2}(1P) \to
\Upsilon(1S) \gamma$ decay for unpolarized and various polarization scenarios
for the $\chi_b$. The resulting efficiency ratios are shown in
Figure~\ref{sec:syst:polarization:eratio_chib1p}.

\begin{figure}[H]
  \setlength{\unitlength}{1mm}
  \centering
  \scalebox{0.6} {
  \begin{picture}(225,120)
  	\put(0,0){
      \includegraphics*[width=75mm, height=45mm]{polarization/angles/w1_cosphi_chib11p_ups1s}
    }
    \put(75,0){
      \includegraphics*[width=75mm, height=45mm]{polarization/angles/w1_theta_chib11p_ups1s}
    }
    \put(150,0){
      \includegraphics*[width=75mm, height=45mm]{polarization/angles/w1_thetap_chib11p_ups1s}
    }
	\put(0,60){
      \includegraphics*[width=75mm, height=45mm]{polarization/angles/w0_cosphi_chib11p_ups1s}
    }
    \put(75,60){
      \includegraphics*[width=75mm, height=45mm]{polarization/angles/w0_theta_chib11p_ups1s}
    }
    \put(150,60){
      \includegraphics*[width=75mm, height=45mm]{polarization/angles/w0_thetap_chib11p_ups1s}
    }

    \put(60,0){$\cos(\phi)$}
    \put(60,60){$\cos(\phi)$}

    \put(135,0){$\cos(\theta_{\chib})$}
    \put(135,60){$\cos(\theta_{\chib})$}

    \put(210,0){$\cos(\theta_{\Upsilon})$}
    \put(210,60){$\cos(\theta_{\Upsilon})$}


	\put(2,10){\begin{sideways}Arbitrary units\end{sideways}}
    \put(2,70){\begin{sideways}Arbitrary units\end{sideways}}

    \put(77,10){\begin{sideways}Arbitrary units\end{sideways}}
    \put(77,70){\begin{sideways}Arbitrary units\end{sideways}}

    \put(152,10){\begin{sideways}Arbitrary units\end{sideways}}
    \put(152,70){\begin{sideways}Arbitrary units\end{sideways}}

    \put(45,35){\includegraphics*[width=4mm, height=2mm]{blue}}
    \put(45,32){\includegraphics*[width=4mm, height=2mm]{red}}

    \put(45,95){\includegraphics*[width=4mm, height=2mm]{blue}}
    \put(45,92){\includegraphics*[width=4mm, height=2mm]{red}}

    \put(120,35){\includegraphics*[width=4mm, height=2mm]{blue}}
    \put(120,32){\includegraphics*[width=4mm, height=2mm]{red}}

    \put(120,95){\includegraphics*[width=4mm, height=2mm]{blue}}
    \put(120,92){\includegraphics*[width=4mm, height=2mm]{red}}

    \put(195,15){\includegraphics*[width=4mm, height=2mm]{blue}}
    \put(195,12){\includegraphics*[width=4mm, height=2mm]{red}}

    \put(195,75){\includegraphics*[width=4mm, height=2mm]{blue}}
    \put(195,72){\includegraphics*[width=4mm, height=2mm]{red}}    


    \put(50,35){unpolarized}
    \put(50,32){$|m_{\chi_{b1}}|=1$}

    \put(50,95){unpolarized}
    \put(50,92){$|m_{\chi_{b1}}|=0$}

    \put(125,35){unpolarized}
    \put(125,32){$|m_{\chi_{b1}}|=1$}

    \put(125,95){unpolarized}
    \put(125,92){$|m_{\chi_{b1}}|=0$}

    \put(200,15){unpolarized}
    \put(200,12){$|m_{\chi_{b1}}|=1$}

    \put(200,75){unpolarized}
    \put(200,72){$|m_{\chi_{b1}}|=0$}    

  \end{picture}
  }
\caption {\small
	Angular distributions of simulated events in $\boldsymbol{\chi_{b1}(1P) \to \Y1S \gamma}$
	decay. The blue curves corresponds to unpolarized events distribution and
	the red curves corresponds to specified polarized events distribution. All
	histograms are normalized by the corresponding integral. }
\label{fig:syst:polarization:angles_chib11p_ups1s}
\end{figure}


\begin{figure}[H]
  \setlength{\unitlength}{1mm}
  \centering
  \scalebox{0.6} {
  \begin{picture}(225,180)
  	\put(0,0){
      \includegraphics*[width=75mm, height=45mm]{polarization/angles/w2_cosphi_chib21p_ups1s}
    }
    \put(75,0){
      \includegraphics*[width=75mm, height=45mm]{polarization/angles/w2_theta_chib21p_ups1s}
    }
    \put(150,0){
      \includegraphics*[width=75mm, height=45mm]{polarization/angles/w2_thetap_chib21p_ups1s}
    }
	\put(0,60){
      \includegraphics*[width=75mm, height=45mm]{polarization/angles/w1_cosphi_chib21p_ups1s}
    }
    \put(75,60){
      \includegraphics*[width=75mm, height=45mm]{polarization/angles/w1_theta_chib21p_ups1s}
    }
    \put(150,60){
      \includegraphics*[width=75mm, height=45mm]{polarization/angles/w1_thetap_chib21p_ups1s}
    }
    \put(0,120){
      \includegraphics*[width=75mm, height=45mm]{polarization/angles/w0_cosphi_chib21p_ups1s}
    }
    \put(75,120){
      \includegraphics*[width=75mm, height=45mm]{polarization/angles/w0_theta_chib21p_ups1s}
    }
    \put(150,120){
      \includegraphics*[width=75mm, height=45mm]{polarization/angles/w0_thetap_chib21p_ups1s}
    }

    \put(60,0){$\cos(\phi)$}
    \put(60,60){$\cos(\phi)$}
    \put(60,120){$\cos(\phi)$}

    \put(135,0){$\cos(\theta_{\chib})$}
    \put(135,60){$\cos(\theta_{\chib})$}
    \put(135,120){$\cos(\theta_{\chib})$}

    \put(210,0){$\cos(\theta_{\Upsilon})$}
    \put(210,60){$\cos(\theta_{\Upsilon})$}
    \put(210,120){$\cos(\theta_{\Upsilon})$}


	  \put(2,10){\begin{sideways}Arbitrary units\end{sideways}}
    \put(2,70){\begin{sideways}Arbitrary units\end{sideways}}
    \put(2,130){\begin{sideways}Arbitrary units\end{sideways}}

    \put(77,10){\begin{sideways}Arbitrary units\end{sideways}}
    \put(77,70){\begin{sideways}Arbitrary units\end{sideways}}
    \put(77,130){\begin{sideways}Arbitrary units\end{sideways}}

    \put(152,10){\begin{sideways}Arbitrary units\end{sideways}}
    \put(152,70){\begin{sideways}Arbitrary units\end{sideways}}
    \put(152,130){\begin{sideways}Arbitrary units\end{sideways}}

    \put(40,37){\includegraphics*[width=4mm, height=2mm]{blue}}
    \put(40,34){\includegraphics*[width=4mm, height=2mm]{red}}

    \put(40,97){\includegraphics*[width=4mm, height=2mm]{blue}}
    \put(40,94){\includegraphics*[width=4mm, height=2mm]{red}}

    \put(40,156){\includegraphics*[width=4mm, height=2mm]{blue}}
    \put(40,154){\includegraphics*[width=4mm, height=2mm]{red}}

    \put(115,37){\includegraphics*[width=4mm, height=2mm]{blue}}
    \put(115,34){\includegraphics*[width=4mm, height=2mm]{red}}

    \put(115,97){\includegraphics*[width=4mm, height=2mm]{blue}}
    \put(115,94){\includegraphics*[width=4mm, height=2mm]{red}}

    \put(115,136){\includegraphics*[width=4mm, height=2mm]{blue}}
    \put(115,133){\includegraphics*[width=4mm, height=2mm]{red}}


    \put(190,37){\includegraphics*[width=4mm, height=2mm]{blue}}
    \put(190,34){\includegraphics*[width=4mm, height=2mm]{red}}

    \put(190,97){\includegraphics*[width=4mm, height=2mm]{blue}}
    \put(190,94){\includegraphics*[width=4mm, height=2mm]{red}}    

    \put(190,156){\includegraphics*[width=4mm, height=2mm]{blue}}
    \put(190,154){\includegraphics*[width=4mm, height=2mm]{red}}    




    \put(45,37){unpolarized}
    \put(45,34){$|m_{\chi_{b2}}|=2$}

    \put(45,97){unpolarized}
    \put(45,94){$|m_{\chi_{b2}}|=1$}

    \put(45,156){unpolarized}
    \put(45,153){$|m_{\chi_{b2}}|=0$}



    \put(120,37){unpolarized}
    \put(120,34){$|m_{\chi_{b2}}|=2$}

    \put(120,97){unpolarized}
    \put(120,94){$|m_{\chi_{b2}}|=1$}

    \put(120,136){unpolarized}
    \put(120,133){$|m_{\chi_{b2}}|=0$}

    
    \put(195,37){unpolarized}
    \put(195,34){$|m_{\chi_{b2}}|=2$}

    \put(195,97){unpolarized}
    \put(195,94){$|m_{\chi_{b2}}|=1$}    

    \put(195,156){unpolarized}
    \put(195,153){$|m_{\chi_{b2}}|=0$}        

  \end{picture}
  }
\caption {\small
  Angular distributions of simulated events in
  $\boldsymbol{\chi_{b2}(1P) \to \Y1S \gamma}$ decay.
  The blue curves corresponds to unpolarized events
  distribution and the red curves corresponds to specified polarized events
  distribution. All histograms are normalized by the corresponding integral. }
\label{fig:syst:polarization:angles_chib21p_ups1s}
\end{figure}
\begin{figure}[H]
  \setlength{\unitlength}{1mm}
  \centering
  \begin{picture}(150,180)
  	\put(0,0){
      \includegraphics*[width=75mm, height=45mm]{polarization/chib21p_ups1s_w2_ratio}
    }
    \put(0,60){
      \includegraphics*[width=75mm, height=45mm]{polarization/chib21p_ups1s_w0_ratio}
    }
    \put(75,60){
      \includegraphics*[width=75mm, height=45mm]{polarization/chib21p_ups1s_w1_ratio}
    }
    \put(0,120){
      \includegraphics*[width=75mm, height=45mm]{polarization/chib11p_ups1s_w0_ratio}
    }
    \put(75,120){
      \includegraphics*[width=75mm, height=45mm]{polarization/chib11p_ups1s_w1_ratio}
    }

    \put(55,0){$p_T^{\Y1S} \left[\gevc\right]$}
    \put(55,60){$p_T^{\Y1S} \left[\gevc\right]$}
    \put(55,120){$p_T^{\Y1S} \left[\gevc\right]$}

    \put(130,60){$p_T^{\Y1S} \left[\gevc\right]$}
    \put(130,120){$p_T^{\Y1S} \left[\gevc\right]$}


    \put(0,20){\begin{sideways}$\eps_{m_{\chi_{b2}}} / \eps_{unpol}$\end{sideways}}
    \put(0,80){\begin{sideways}$\eps_{m_{\chi_{b2}}} / \eps_{unpol}$\end{sideways}}
    \put(0,140){\begin{sideways}$\eps_{m_{\chi_{b1}}} / \eps_{unpol}$\end{sideways}}

    \put(75,80){\begin{sideways}$\eps_{m_{\chi_{b2}}} / \eps_{unpol}$\end{sideways}}
    \put(75,140){\begin{sideways}$\eps_{m_{\chi_{b1}}} / \eps_{unpol}$\end{sideways}}

    % \put(65,37){$m_2$}
    % \put(65,97){$m_0$}
    % \put(140,97){$m_1$}

    % \put(65,157){$m_0$}
    % \put(140,157){$m_1$}

    % \put(25,37){\small $\chi_{\bm{b2}}(1P) \to \Y1S \gamma$}
    % \put(25,97){\small $\chi_{\bm{b2}}(1P) \to \Y1S \gamma$}
    % \put(25,157){\small $\chi_{\bm{b1}}(1P) \to \Y1S \gamma$}

    % \put(100,97){\small $\chi_{\bm{b2}}(1P) \to \Y1S \gamma$}
    % \put(100,157){\small $\chi_{\bm{b1}}(1P) \to \Y1S \gamma$}


    \put(25,37){\small $|m_{\chi_{b2}}|=2$}
    \put(25,97){\small $|m_{\chi_{b2}}|=0$}
    \put(25,157){\small $|m_{\chi_{b1}}|=0$}

    \put(100,97){\small $|m_{\chi_{b2}}|=1$}
    \put(100,157){\small $|m_{\chi_{b1}}|=1$}


    % \put(65,97){$m_0$}
    % \put(140,97){$m_1$}

    % \put(65,157){$m_0$}
    % \put(140,157){$m_1$}

  \end{picture}
\caption {\small
Ratio between  efficiency for polarized events and the corresponding
efficiency for unpolarized events  in $\chi_{b}(1P) \to \Y1S \gamma$ decays.
The results are shown in specified intervals of \Y1S transverse momentum. }
\label{fig:syst:polarization:eratio_chib1p}
\end{figure}

\begin{figure}[H]
  \setlength{\unitlength}{1mm}
  \centering
  \begin{picture}(150,180)
    \put(0,0){
      \includegraphics*[width=75mm, height=45mm]{polarization/chib22p_ups1s_w2_ratio}
    }
    \put(0,60){
      \includegraphics*[width=75mm, height=45mm]{polarization/chib22p_ups1s_w0_ratio}
    }
    \put(75,60){
      \includegraphics*[width=75mm, height=45mm]{polarization/chib22p_ups1s_w1_ratio}
    }
    \put(0,120){
      \includegraphics*[width=75mm, height=45mm]{polarization/chib12p_ups1s_w0_ratio}
    }
    \put(75,120){
      \includegraphics*[width=75mm, height=45mm]{polarization/chib12p_ups1s_w1_ratio}
    }

    \put(55,0){$p_T^{\Y1S} \left[\gevc\right]$}
    \put(55,60){$p_T^{\Y1S} \left[\gevc\right]$}
    \put(55,120){$p_T^{\Y1S} \left[\gevc\right]$}

    \put(130,60){$p_T^{\Y1S} \left[\gevc\right]$}
    \put(130,120){$p_T^{\Y1S} \left[\gevc\right]$}


    \put(0,20){\begin{sideways}$\eps_{unpol} / \eps_{m_{\chi_{b2}}}$\end{sideways}}
    \put(0,80){\begin{sideways}$\eps_{unpol} / \eps_{m_{\chi_{b2}}}$\end{sideways}}
    \put(0,140){\begin{sideways}$\eps_{unpol} / \eps_{m_{\chi_{b1}}}$\end{sideways}}

    \put(75,80){\begin{sideways}$\eps_{unpol} / \eps_{m_{\chi_{b2}}}$\end{sideways}}
    \put(75,140){\begin{sideways}$\eps_{unpol} / \eps_{m_{\chi_{b1}}}$\end{sideways}}


    \put(25,37){\small $|m_{\chi_{b2}}|=2$}
    \put(25,97){\small $|m_{\chi_{b2}}|=0$}
    \put(25,157){\small $|m_{\chi_{b1}}|=0$}

    \put(100,97){\small $|m_{\chi_{b2}}|=1$}
    \put(100,157){\small $|m_{\chi_{b1}}|=1$}


    % \put(65,97){$m_0$}
    % \put(140,97){$m_1$}

    % \put(65,157){$m_0$}
    % \put(140,157){$m_1$}

  \end{picture}
\caption {\small
Ratio between  efficiency for polarized events and the corresponding
efficiency for unpolarized events  in $\chi_{b}(2P) \to \Y1S \gamma$ decays.
The results are shown in specified intervals of \Y1S transverse momentum. }
\label{fig:syst:polarization:eratio_chib2p}
\end{figure}






The systematic uncertainty for different polarization scenarios is estimated
as a maximum deviation of ratio between efficiency measured for unpolarized
particles and all possible polarization scenarios. The results  are
shown in Tables~\ref{tab:syst:pol:chib1p_ups1s}-~\ref{tab:syst:pol:chib3p_ups3s}.

\begin{table}[H]
\caption{\small Maximum deviation (\%) of ratio between efficiency measured for unpolarized particles and all possible polarization scenarios in $\chi_{b} \to \Upsilon(1S) \gamma$ decays}
\centering
\scalebox{1}{
\begin{tabular}{lrrrrrr}\toprule
 & \multicolumn{6}{c}{$\Upsilon(1S)$ transverse momentum intervals, \gevc}\\
 & \multicolumn{1}{c}{6 -- 8} & \multicolumn{1}{c}{8 -- 10} & \multicolumn{1}{c}{10 -- 14} & \multicolumn{1}{c}{14 -- 18} & \multicolumn{1}{c}{18 -- 22} & \multicolumn{1}{c}{22 -- 40}\\
\midrule
$\chi_b(1P) \to \Y1S \gamma$ & ${}^{+2.4}_{-4.0}$ & ${}^{+3.5}_{-5.1}$ & ${}^{+2.9}_{-3.3}$ & ${}^{+1.1}_{-1.1}$ & ${}^{+2.3}_{-1.8}$ & ${}^{+4.0}_{-2.9}$\\

\rule{0pt}{4ex}$\chi_b(2P) \to \Y1S \gamma$ & ${}^{+0.9}_{-2.0}$ & ${}^{+0.9}_{-1.5}$ & ${}^{+0.7}_{-0.8}$ & ${}^{+2.7}_{-2.8}$ & ${}^{+5.3}_{-5.8}$ & ${}^{+6.8}_{-5.5}$\\

\rule{0pt}{4ex}$\chi_b(3P) \to \Y1S \gamma$ & --- & --- & ${}^{+2.2}_{-2.4}$ & ${}^{+5.2}_{-5.3}$ & ${}^{+6.7}_{-6.9}$ & ${}^{+5.9}_{-6.3}$\\
\bottomrule
\end{tabular}
} % scalebox
\label{tab:syst:pol:ups1s}
\end{table}

\begin{table}[H]
\caption{\small Maximum deviation (\%) of ratio between efficiency measured for unpolarized particles and all possible polarization scenarios in $\chi_{b} \to \Upsilon(2S) \gamma$ decays}
\centering
\scalebox{1}{
\begin{tabular}{lrrrr}\toprule
 & \multicolumn{4}{c}{$\Upsilon(2S)$ transverse momentum intervals, \gevc}\\
 & \multicolumn{1}{c}{18 -- 22} & \multicolumn{1}{c}{18 -- 24} & \multicolumn{1}{c}{22 -- 24} & \multicolumn{1}{c}{24 -- 40}\\
\midrule
$\chi_b(2P) \to \Y2S \gamma$ & ${}^{+7.8}_{-8.7}$ & --- & ${}^{+6.1}_{-3.6}$ & ${}^{+4.6}_{-4.3}$\\

\rule{0pt}{4ex}$\chi_b(3P) \to \Y2S \gamma$ & --- & ${}^{+2.7}_{-2.6}$ & --- & ${}^{+4.2}_{-4.5}$\\
\bottomrule
\end{tabular}
} % scalebox
\label{tab:syst:pol:ups2s}
\end{table}

\begin{table}[H]
\caption{\small Maximum deviation (\%) of ratio between efficiency measured for unpolarized particles and all possible polarization scenarios in $\chi_{b} \to \Upsilon(3S) \gamma$ decays}
\centering
\scalebox{1}{
\begin{tabular}{lr}\toprule
 & \multicolumn{1}{c}{$\Upsilon(3S)$ transverse momentum intervals, \gevc}\\
 & \multicolumn{1}{c}{27 -- 40}\\
\midrule
$\chi_b(3P) \to \Y3S \gamma$ & ${}^{+7.5}_{-6.4}$\\
\bottomrule
\end{tabular}
} % scalebox
\label{tab:syst:pol:ups3s}
\end{table}


\begin{table}[H]
\centering
\caption{\small \Y1S fraction systematic uncertainties  related to $\chi_b$ fit model.}
\subtable[$6 < p_T^{\Y1S} < 10 \gevc$] {
\scalebox{0.7}{
\begin{tabular}{lrrrr}\toprule
 & \multicolumn{4}{c}{$\Upsilon(1S)$ transverse momentum intervals, \gevc}\\
 & \multicolumn{2}{c}{6 -- 8} & \multicolumn{2}{c}{8 -- 10}\\
\cmidrule(r){2-3}\cmidrule(r){4-5}
 & \sqs = 7\tev & \sqs = 8\tev & \sqs = 7\tev & \sqs = 8\tev\\
\midrule
$\chi_b(1P) \to \Y1S \gamma$ & ${}^{+1.4 \%}_{-0.8 \%}$ & ${}^{+1.2 \%}_{-0.8 \%}$ & ${}^{+2.2 \%}_{-1.2 \%}$ & ${}^{+1.8 \%}_{-1.0 \%}$\\

\rule{0pt}{4ex}$\chi_b(2P) \to \Y1S \gamma$ & ${}^{+0.1 \%}_{-0.1 \%}$ & ${}^{+0.2 \%}_{-0.2 \%}$ & ${}^{+0.2 \%}_{-0.3 \%}$ & ${}^{+0.2 \%}_{-0.2 \%}$\\

\rule{0pt}{4ex}$\chi_b(3P) \to \Y1S \gamma$ & --- & --- & --- & ---\\
\bottomrule
\end{tabular}
} % scalebox

} % subtable
\subtable[$10 < p_T^{\Y1S} < 18 \gevc$] {
\scalebox{0.7}{
\begin{tabular}{lrrrr}\toprule
 & \multicolumn{4}{c}{$\Upsilon(1S)$ transverse momentum intervals, \gevc}\\
 & \multicolumn{2}{c}{10 -- 14} & \multicolumn{2}{c}{14 -- 18}\\
\cmidrule(r){2-3}\cmidrule(r){4-5}
 & \sqs = 7\tev & \sqs = 8\tev & \sqs = 7\tev & \sqs = 8\tev\\
\midrule
$\chi_b(1P) \to \Y1S \gamma$ & ${}^{+1.9 \%}_{-1.0 \%}$ & ${}^{+1.8 \%}_{-1.4 \%}$ & ${}^{+0.9 \%}_{-0.6 \%}$ & ${}^{+0.9 \%}_{-0.6 \%}$\\

\rule{0pt}{4ex}$\chi_b(2P) \to \Y1S \gamma$ & ${}^{+0.2 \%}_{-0.2 \%}$ & ${}^{+0.3 \%}_{-0.3 \%}$ & ${}^{+0.3 \%}_{-0.2 \%}$ & ${}^{+0.2 \%}_{-0.2 \%}$\\

\rule{0pt}{4ex}$\chi_b(3P) \to \Y1S \gamma$ & ${}^{+0.5 \%}_{-0.3 \%}$ & ${}^{+0.3 \%}_{-0.3 \%}$ & ${}^{+0.5 \%}_{-0.2 \%}$ & ${}^{+0.1 \%}_{-0.1 \%}$\\
\bottomrule
\end{tabular}
} % scalebox

} % subtable
\subtable[$18 < p_T^{\Y1S} < 40 \gevc$] {
\scalebox{0.7}{
\begin{tabular}{lrrrr}\toprule
 & \multicolumn{4}{c}{$\Upsilon(1S)$ transverse momentum intervals, \gevc}\\
 & \multicolumn{2}{c}{18 -- 22} & \multicolumn{2}{c}{22 -- 40}\\
\cmidrule(r){2-3}\cmidrule(r){4-5}
 & \sqs = 7\tev & \sqs = 8\tev & \sqs = 7\tev & \sqs = 8\tev\\
\midrule
$\chi_b(1P) \to \Y1S \gamma$ & ${}^{+0.9 \%}_{-0.5 \%}$ & ${}^{+0.8 \%}_{-0.5 \%}$ & ${}^{+0.7 \%}_{-0.5 \%}$ & ${}^{+0.0 \%}_{-3.6 \%}$\\

\rule{0pt}{4ex}$\chi_b(2P) \to \Y1S \gamma$ & ${}^{+0.1 \%}_{-0.2 \%}$ & ${}^{+0.1 \%}_{-0.1 \%}$ & ${}^{+0.2 \%}_{-0.2 \%}$ & ${}^{+2.3 \%}_{-0.0 \%}$\\

\rule{0pt}{4ex}$\chi_b(3P) \to \Y1S \gamma$ & ${}^{+0.2 \%}_{-0.3 \%}$ & ${}^{+0.1 \%}_{-0.1 \%}$ & ${}^{+0.1 \%}_{-0.1 \%}$ & ${}^{+1.6 \%}_{-0.3 \%}$\\
\bottomrule
\end{tabular}
} % scalebox

} % subtable
\label{tab:frac:ups1s_chib_model}
\end{table}

\begin{table}[H]
\centering
\caption{\small \Y1S fraction systematic uncertainties related to $\Upsilon$ fit model.}
\subtable[$6 < p_T^{\Y1S} < 10 \gevc$] {
\scalebox{0.7}{
\begin{tabular}{lrrrr}\toprule
 & \multicolumn{4}{c}{$\Upsilon(1S)$ transverse momentum intervals, \gevc}\\
 & \multicolumn{2}{c}{6 -- 8} & \multicolumn{2}{c}{8 -- 10}\\
\cmidrule(r){2-3}\cmidrule(r){4-5}
 & \sqs = 7\tev & \sqs = 8\tev & \sqs = 7\tev & \sqs = 8\tev\\
\midrule
$\chi_b(1P) \to \Y1S \gamma$ & $\pm 0.155 \%$ & $\pm 0.163 \%$ & $\pm 0.162 \%$ & $\pm 0.166 \%$\\

\rule{0pt}{4ex}$\chi_b(2P) \to \Y1S \gamma$ & $\pm 0.027 \%$ & $\pm 0.021 \%$ & $\pm 0.049 \%$ & $\pm 0.043 \%$\\

\rule{0pt}{4ex}$\chi_b(3P) \to \Y1S \gamma$ & --- & --- & --- & ---\\
\bottomrule
\end{tabular}
} % scalebox

} % subtable
\subtable[$10 < p_T^{\Y1S} < 18 \gevc$] {
\scalebox{0.7}{
\begin{tabular}{lrrrr}\toprule
 & \multicolumn{4}{c}{$\Upsilon(1S)$ transverse momentum intervals, \gevc}\\
 & \multicolumn{2}{c}{10 -- 14} & \multicolumn{2}{c}{14 -- 18}\\
\cmidrule(r){2-3}\cmidrule(r){4-5}
 & \sqs = 7\tev & \sqs = 8\tev & \sqs = 7\tev & \sqs = 8\tev\\
\midrule
$\chi_b(1P) \to \Y1S \gamma$ & $\pm 0.178 \%$ & $\pm 0.196 \%$ & $\pm 0.212 \%$ & $\pm 0.206 \%$\\

\rule{0pt}{4ex}$\chi_b(2P) \to \Y1S \gamma$ & $\pm 0.044 \%$ & $\pm 0.031 \%$ & $\pm 0.048 \%$ & $\pm 0.044 \%$\\

\rule{0pt}{4ex}$\chi_b(3P) \to \Y1S \gamma$ & $\pm 0.012 \%$ & $\pm 0.009 \%$ & $\pm 0.016 \%$ & $\pm 0.011 \%$\\
\bottomrule
\end{tabular}
} % scalebox

} % subtable
\subtable[$18 < p_T^{\Y1S} < 40 \gevc$] {
\scalebox{0.7}{
\begin{tabular}{lrrrr}\toprule
 & \multicolumn{4}{c}{$\Upsilon(1S)$ transverse momentum intervals, \gevc}\\
 & \multicolumn{2}{c}{18 -- 22} & \multicolumn{2}{c}{22 -- 40}\\
\cmidrule(r){2-3}\cmidrule(r){4-5}
 & \sqs = 7\tev & \sqs = 8\tev & \sqs = 7\tev & \sqs = 8\tev\\
\midrule
$\chi_b(1P) \to \Y1S \gamma$ & $\pm 0.230 \%$ & $\pm 0.221 \%$ & $\pm 0.244 \%$ & $\pm 0.224 \%$\\

\rule{0pt}{4ex}$\chi_b(2P) \to \Y1S \gamma$ & $\pm 0.047 \%$ & $\pm 0.036 \%$ & $\pm 0.055 \%$ & $\pm 0.069 \%$\\

\rule{0pt}{4ex}$\chi_b(3P) \to \Y1S \gamma$ & $\pm 0.016 \%$ & $\pm 0.009 \%$ & $\pm 0.027 \%$ & $\pm 0.024 \%$\\
\bottomrule
\end{tabular}
} % scalebox

} % subtable
\label{tab:frac:ups1s_ups_model}
\end{table}

\begin{table}[H]
\centering
\caption{\small \Y1S fraction systematic uncertainties related to photon reconstruction efficiency.}
\subtable[$6 < p_T^{\Y1S} < 10 \gevc$] {
\scalebox{0.7}{
\begin{tabular}{lrrrr}\toprule
 & \multicolumn{4}{c}{$\Upsilon(1S)$ transverse momentum intervals, \gevc}\\
 & \multicolumn{2}{c}{6 -- 8} & \multicolumn{2}{c}{8 -- 10}\\
\cmidrule(r){2-3}\cmidrule(r){4-5}
 & \sqs = 7\tev & \sqs = 8\tev & \sqs = 7\tev & \sqs = 8\tev\\
\midrule
$\chi_b(1P) \to \Y1S \gamma$ & $\pm 0.68 \%$ & $\pm 0.71 \%$ & $\pm 0.71 \%$ & $\pm 0.73 \%$\\

\rule{0pt}{4ex}$\chi_b(2P) \to \Y1S \gamma$ & $\pm 0.12 \%$ & $\pm 0.09 \%$ & $\pm 0.22 \%$ & $\pm 0.19 \%$\\

\rule{0pt}{4ex}$\chi_b(3P) \to \Y1S \gamma$ & --- & --- & --- & ---\\
\bottomrule
\end{tabular}
} % scalebox

} % subtable
\subtable[$10 < p_T^{\Y1S} < 18 \gevc$] {
\scalebox{0.7}{
\begin{tabular}{lrrrr}\toprule
 & \multicolumn{4}{c}{$\Upsilon(1S)$ transverse momentum intervals, \gevc}\\
 & \multicolumn{2}{c}{10 -- 14} & \multicolumn{2}{c}{14 -- 18}\\
\cmidrule(r){2-3}\cmidrule(r){4-5}
 & \sqs = 7\tev & \sqs = 8\tev & \sqs = 7\tev & \sqs = 8\tev\\
\midrule
$\chi_b(1P) \to \Y1S \gamma$ & $\pm 0.78 \%$ & $\pm 0.86 \%$ & $\pm 0.93 \%$ & $\pm 0.90 \%$\\

\rule{0pt}{4ex}$\chi_b(2P) \to \Y1S \gamma$ & $\pm 0.19 \%$ & $\pm 0.14 \%$ & $\pm 0.21 \%$ & $\pm 0.19 \%$\\

\rule{0pt}{4ex}$\chi_b(3P) \to \Y1S \gamma$ & $\pm 0.05 \%$ & $\pm 0.04 \%$ & $\pm 0.07 \%$ & $\pm 0.05 \%$\\
\bottomrule
\end{tabular}
} % scalebox

} % subtable
\subtable[$18 < p_T^{\Y1S} < 40 \gevc$] {
\scalebox{0.7}{
\begin{tabular}{lrrrr}\toprule
 & \multicolumn{4}{c}{$\Upsilon(1S)$ transverse momentum intervals, \gevc}\\
 & \multicolumn{2}{c}{18 -- 22} & \multicolumn{2}{c}{22 -- 40}\\
\cmidrule(r){2-3}\cmidrule(r){4-5}
 & \sqs = 7\tev & \sqs = 8\tev & \sqs = 7\tev & \sqs = 8\tev\\
\midrule
$\chi_b(1P) \to \Y1S \gamma$ & $\pm 1.01 \%$ & $\pm 0.97 \%$ & $\pm 1.07 \%$ & $\pm 0.98 \%$\\

\rule{0pt}{4ex}$\chi_b(2P) \to \Y1S \gamma$ & $\pm 0.21 \%$ & $\pm 0.16 \%$ & $\pm 0.24 \%$ & $\pm 0.30 \%$\\

\rule{0pt}{4ex}$\chi_b(3P) \to \Y1S \gamma$ & $\pm 0.07 \%$ & $\pm 0.04 \%$ & $\pm 0.12 \%$ & $\pm 0.10 \%$\\
\bottomrule
\end{tabular}
} % scalebox

} % subtable
\label{tab:frac:ups1s_eff}
\end{table}

\begin{table}[H]
\centering
\caption{\small \Y2S fraction systematic uncertainties  related to $\chi_b$ fit model.}
\subtable[$18 < p_T^{\Y2S} < 24 \gevc$] {
\scalebox{0.7}{
\begin{tabular}{lrrrr}\toprule
 & \multicolumn{4}{c}{$\Upsilon(2S)$ transverse momentum intervals, \gevc}\\
 & \multicolumn{2}{c}{18 -- 22} & \multicolumn{2}{c}{18 -- 24}\\
\cmidrule(r){2-3}\cmidrule(r){4-5}
 & \sqs = 7\tev & \sqs = 8\tev & \sqs = 7\tev & \sqs = 8\tev\\
\midrule
$\chi_b(2P) \to \Y2S \gamma$ & ${}^{+1.9 \%}_{-0.4 \%}$ & ${}^{+1.2 \%}_{-0.0 \%}$ & --- & ---\\

\rule{0pt}{4ex}$\chi_b(3P) \to \Y2S \gamma$ & --- & --- & ${}^{+0.4 \%}_{-0.2 \%}$ & ${}^{+0.2 \%}_{-0.0 \%}$\\
\bottomrule
\end{tabular}
} % scalebox

} % subtable
\subtable[$22 < p_T^{\Y2S} < 40 \gevc$] {
\scalebox{0.7}{
\begin{tabular}{lrrrr}\toprule
 & \multicolumn{4}{c}{$\Upsilon(2S)$ transverse momentum intervals, \gevc}\\
 & \multicolumn{2}{c}{22 -- 24} & \multicolumn{2}{c}{24 -- 40}\\
\cmidrule(r){2-3}\cmidrule(r){4-5}
 & \sqs = 7\tev & \sqs = 8\tev & \sqs = 7\tev & \sqs = 8\tev\\
\midrule
$\chi_b(2P) \to \Y2S \gamma$ & ${}^{+2.3 \%}_{-0.7 \%}$ & ${}^{+1.9 \%}_{-0.6 \%}$ & ${}^{+0.9 \%}_{-0.0 \%}$ & ${}^{+1.0 \%}_{-0.1 \%}$\\

\rule{0pt}{4ex}$\chi_b(3P) \to \Y2S \gamma$ & --- & --- & ${}^{+1.0 \%}_{-0.0 \%}$ & ${}^{+0.5 \%}_{-0.5 \%}$\\
\bottomrule
\end{tabular}
} % scalebox

} % subtable
\label{tab:frac:ups2s_chib_model}
\end{table}

\begin{table}[H]
\centering
\caption{\small \Y2S fraction systematic uncertainties related to $\Upsilon$ fit model.}
\subtable[$18 < p_T^{\Y2S} < 24 \gevc$] {
\scalebox{0.7}{
\begin{tabular}{lrrrr}\toprule
 & \multicolumn{4}{c}{$\Upsilon(2S)$ transverse momentum intervals, \gevc}\\
 & \multicolumn{2}{c}{18 -- 22} & \multicolumn{2}{c}{18 -- 24}\\
\cmidrule(r){2-3}\cmidrule(r){4-5}
 & \sqs = 7\tev & \sqs = 8\tev & \sqs = 7\tev & \sqs = 8\tev\\
\midrule
$\chi_b(2P) \to \Y2S \gamma$ & $\pm 0.215 \%$ & $\pm 0.231 \%$ & --- & ---\\

\rule{0pt}{4ex}$\chi_b(3P) \to \Y2S \gamma$ & --- & --- & $\pm 0.026 \%$ & $\pm 0.023 \%$\\
\bottomrule
\end{tabular}
} % scalebox

} % subtable
\subtable[$22 < p_T^{\Y2S} < 40 \gevc$] {
\scalebox{0.7}{
\begin{tabular}{lrrrr}\toprule
 & \multicolumn{4}{c}{$\Upsilon(2S)$ transverse momentum intervals, \gevc}\\
 & \multicolumn{2}{c}{22 -- 24} & \multicolumn{2}{c}{24 -- 40}\\
\cmidrule(r){2-3}\cmidrule(r){4-5}
 & \sqs = 7\tev & \sqs = 8\tev & \sqs = 7\tev & \sqs = 8\tev\\
\midrule
$\chi_b(2P) \to \Y2S \gamma$ & $\pm 0.235 \%$ & $\pm 0.215 \%$ & $\pm 0.180 \%$ & $\pm 0.217 \%$\\

\rule{0pt}{4ex}$\chi_b(3P) \to \Y2S \gamma$ & --- & --- & $\pm 0.063 \%$ & $\pm 0.021 \%$\\
\bottomrule
\end{tabular}
} % scalebox

} % subtable
\label{tab:frac:ups2s_ups_model}
\end{table}

\begin{table}[H]
\centering
\caption{\small \Y2S fraction systematic uncertainties related to photon reconstruction efficiency.}
\subtable[$18 < p_T^{\Y2S} < 24 \gevc$] {
\scalebox{0.7}{
\begin{tabular}{lrrrr}\toprule
 & \multicolumn{4}{c}{$\Upsilon(2S)$ transverse momentum intervals, \gevc}\\
 & \multicolumn{2}{c}{18 -- 22} & \multicolumn{2}{c}{18 -- 24}\\
\cmidrule(r){2-3}\cmidrule(r){4-5}
 & \sqs = 7\tev & \sqs = 8\tev & \sqs = 7\tev & \sqs = 8\tev\\
\midrule
$\chi_b(2P) \to \Y2S \gamma$ & $\pm 0.95 \%$ & $\pm 1.01 \%$ & --- & ---\\

\rule{0pt}{4ex}$\chi_b(3P) \to \Y2S \gamma$ & --- & --- & $\pm 0.11 \%$ & $\pm 0.10 \%$\\
\bottomrule
\end{tabular}
} % scalebox

} % subtable
\subtable[$22 < p_T^{\Y2S} < 40 \gevc$] {
\scalebox{0.7}{
\begin{tabular}{lrrrr}\toprule
 & \multicolumn{4}{c}{$\Upsilon(2S)$ transverse momentum intervals, \gevc}\\
 & \multicolumn{2}{c}{22 -- 24} & \multicolumn{2}{c}{24 -- 40}\\
\cmidrule(r){2-3}\cmidrule(r){4-5}
 & \sqs = 7\tev & \sqs = 8\tev & \sqs = 7\tev & \sqs = 8\tev\\
\midrule
$\chi_b(2P) \to \Y2S \gamma$ & $\pm 1.03 \%$ & $\pm 0.94 \%$ & $\pm 0.79 \%$ & $\pm 0.95 \%$\\

\rule{0pt}{4ex}$\chi_b(3P) \to \Y2S \gamma$ & --- & --- & $\pm 0.28 \%$ & $\pm 0.09 \%$\\
\bottomrule
\end{tabular}
} % scalebox

} % subtable
\label{tab:frac:ups2s_eff}
\end{table}

\begin{table}[H]
\caption{\small \Y3S fraction systematic uncertainties  related to $\chi_b$ fit model.}
\centering
\scalebox{0.7}{
\begin{tabular}{lrr}\toprule
 & \multicolumn{2}{c}{$\Upsilon(3S)$ transverse momentum intervals, \gevc}\\
 & \multicolumn{2}{c}{27 -- 40}\\
\cmidrule(r){2-3}
 & \sqs = 7\tev & \sqs = 8\tev\\
\midrule
$\chi_b(3P) \to \Y3S \gamma$ & ${}^{+11.9 \%}_{-10.5 \%}$ & ${}^{+11.6 \%}_{-0.0 \%}$\\
\bottomrule
\end{tabular}
} % scalebox
\label{tab:frac:ups3s_chib_model}
\end{table}

\begin{table}[H]
\caption{\small \Y3S fraction systematic uncertainties related to $\Upsilon$ fit model.}
\centering
\scalebox{0.7}{
\begin{tabular}{lrr}\toprule
 & \multicolumn{2}{c}{$\Upsilon(3S)$ transverse momentum intervals, \gevc}\\
 & \multicolumn{2}{c}{27 -- 40}\\
\cmidrule(r){2-3}
 & \sqs = 7\tev & \sqs = 8\tev\\
\midrule
$\chi_b(3P) \to \Y3S \gamma$ & $\pm 0.290 \%$ & $\pm 0.295 \%$\\
\bottomrule
\end{tabular}
} % scalebox
\label{tab:frac:ups3s_ups_model}
\end{table}

\begin{table}[H]
\caption{\small \Y3S fraction systematic uncertainties related to photon reconstruction efficiency.}
\centering
\scalebox{0.7}{
\begin{tabular}{lrr}\toprule
 & \multicolumn{2}{c}{$\Upsilon(3S)$ transverse momentum intervals, \gevc}\\
 & \multicolumn{2}{c}{27 -- 40}\\
\cmidrule(r){2-3}
 & \sqs = 7\tev & \sqs = 8\tev\\
\midrule
$\chi_b(3P) \to \Y3S \gamma$ & $\pm 1.27 \%$ & $\pm 1.30 \%$\\
\bottomrule
\end{tabular}
} % scalebox
\label{tab:frac:ups3s_eff}
\end{table}




