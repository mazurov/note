\section{Results and summary}
\label{sec:results}

In summary, the fractions of $\Upsilon(1,2,3S)$ mesons originating from $\chi_b(1,2,3P)$ 
radiative decays has been measured on the full data sample collected by LHCb in 2011 and 2012 
at center of mass energies of 7 and 8 \tev respectively, as a function of the $\Upsilon$ 
transverse momentum. Results are shown in~\Cref{fig:results}
and~\Crefrange{tab:frac:ups1s_final}{tab:frac:ups3s_final}.
\Cref{fig:results_2010} shows previous \lhcb results which
are consistent with the current ones.


\begin{figure}[H]
  \setlength{\unitlength}{1mm}
  \centering
  \resizebox{\textwidth}{!}{
  \begin{picture}(150,120)
    %% =======================================================================
    \put(0,60){
      \includegraphics*[width=75mm, height=60mm]{results/ups1s}
    }
    \put(2,85){\begin{sideways}\Y1S fraction, \% \end{sideways}}
    \put(35,62){$p_T^{\Y1S} \left[\gevc\right]$}

    % \put(15,113){\scriptsize $\chi_b(1,2,3P) \to \Y1S \gamma$}
    
    \put(55,113){\scriptsize \textcolor{blue}{\sqs=7\tev}}
    \put(55,109){\scriptsize \textcolor{red}{\sqs=8\tev}}
    % \put(48,140){\scriptsize \textcolor{cyan}{\sqs=7\tev (2010)}}
    
    \put(15,113){\includegraphics*[width=2mm, height=2mm]{markers/circle-blue-open}}
    \put(17,113){\includegraphics*[width=2mm, height=2mm]{markers/circle-red-filled}}
    \put(20,114){\tiny $\chibOneP \to \Y1S \gamma$}

    \put(15,110){\includegraphics*[width=2mm, height=2mm]{markers/tri-blue-open}}
    \put(17,110){\includegraphics*[width=2mm, height=2mm]{markers/tri-red-filled}}
    \put(20,111){\tiny $\chibTwoP \to \Y1S \gamma$}

    \put(15,107){\includegraphics*[width=2mm, height=2mm]{markers/sq-blue-open}}
    \put(17,107){\includegraphics*[width=2mm, height=2mm]{markers/sq-red-filled}}
    \put(20,108){\tiny $\chibThreeP \to \Y1S \gamma$}    

    %% =======================================================================
    \put(75,60){
      \includegraphics*[width=75mm, height=60mm]{results/ups2s}
    }
    \put(77,85){\begin{sideways}\Y2S fraction, \% \end{sideways}}
    \put(110,62){$p_T^{\Y2S} \left[\gevc\right]$}

    % \put(90,113){\scriptsize $\chi_b(2,3P) \to \Y2S \gamma$}
    
    \put(130,113){\scriptsize \textcolor{blue}{\sqs=7\tev}}
    \put(130,109){\scriptsize \textcolor{red}{\sqs=8\tev}}
    
    
    \put(90,113){\includegraphics*[width=2mm, height=2mm]{markers/tri-blue-open}}
    \put(92,113){\includegraphics*[width=2mm, height=2mm]{markers/tri-red-filled}}
    \put(95,114){\tiny $\chibTwoP \to \Y2S \gamma$}

    \put(90,110){\includegraphics*[width=2mm, height=2mm]{markers/sq-blue-open}}
    \put(92,110){\includegraphics*[width=2mm, height=2mm]{markers/sq-red-filled}}
    \put(95,111){\tiny $\chibThreeP \to \Y2S \gamma$}    

    
    %% =======================================================================
    \put(0,0){
      \includegraphics*[width=75mm, height=60mm]{results/ups3s}
    }
    \put(2,25){\begin{sideways}\Y3S fraction, \% \end{sideways}}
    \put(35,2){$p_T^{\Y3S} \left[\gevc\right]$}

    % \put(15,53){\scriptsize $\chi_b(3P) \to \Y3S \gamma$}
    
    \put(55,52){\scriptsize \textcolor{blue}{\sqs=7\tev}}
    \put(55,48){\scriptsize \textcolor{red}{\sqs=8\tev}}
    
    \put(15,53){\includegraphics*[width=2mm, height=2mm]{markers/sq-blue-open}}
    \put(17,53){\includegraphics*[width=2mm, height=2mm]{markers/sq-red-filled}}
    \put(20,54){\tiny $\chibThreeP \to \Y3S \gamma$}    

   
  % \graphpaper[5](0,0)(150, 120)
  \end{picture}
  }
  \caption{\small
    Fraction of $\Upsilon$ originated from \chib decays in the specified
    $p_T^{\Upsilon}$ ranges. Outer error bars show
    statistical and systematics errors, inner error bars --- only statistical
    errors.}
  \label{fig:results}
\end{figure}
\begin{figure}[H]
  \setlength{\unitlength}{1mm}
  \centering
  \resizebox{\textwidth}{!}{
  \begin{picture}(150,60)
    %% =======================================================================
    \put(0,0){
      \includegraphics*[width=75mm, height=60mm]{results/ups1s_23p}
    }
    \put(2,25){\begin{sideways}\Y1S fraction, \% \end{sideways}}
    \put(35,2){$p_T^{\Y1S} \left[\gevc\right]$}

    % \put(15,113){\scriptsize $\chi_b(1,2,3P) \to \Y1S \gamma$}
    
    \put(55,54){\scriptsize \textcolor{blue}{\sqs=7\tev}}
    \put(55,51){\scriptsize \textcolor{red}{\sqs=8\tev}}
    % \put(48,140){\scriptsize \textcolor{cyan}{\sqs=7\tev (2010)}}
    
    \put(15,53){\includegraphics*[width=2mm, height=2mm]{markers/circle-blue-open}}
    \put(17,53){\includegraphics*[width=2mm, height=2mm]{markers/circle-red-filled}}
    \put(20,54){\tiny $\chibOneP \to \Y1S \gamma$}

    \put(15,50){\includegraphics*[width=2mm, height=2mm]{markers/tri-blue-open}}
    \put(17,50){\includegraphics*[width=2mm, height=2mm]{markers/tri-red-filled}}
    \put(20,51){\tiny $\chibTwoP \to \Y1S \gamma$}

    \put(15,47){\includegraphics*[width=2mm, height=2mm]{markers/sq-blue-open}}
    \put(17,47){\includegraphics*[width=2mm, height=2mm]{markers/sq-red-filled}}
    \put(20,48){\tiny $\chibThreeP \to \Y1S \gamma$}    

    %% =======================================================================
    \put(75,0){
      \includegraphics*[width=75mm, height=60mm]{results/ups2s_3p}
    }
    \put(77,25){\begin{sideways}\Y2S fraction, \% \end{sideways}}
    \put(110,2){$p_T^{\Y2S} \left[\gevc\right]$}

    % \put(90,113){\scriptsize $\chi_b(2,3P) \to \Y2S \gamma$}
    
    \put(130,53){\scriptsize \textcolor{blue}{\sqs=7\tev}}
    \put(130,49){\scriptsize \textcolor{red}{\sqs=8\tev}}
    
    
    \put(90,53){\includegraphics*[width=2mm, height=2mm]{markers/tri-blue-open}}
    \put(92,53){\includegraphics*[width=2mm, height=2mm]{markers/tri-red-filled}}
    \put(95,54){\tiny $\chibTwoP \to \Y2S \gamma$}

    \put(90,50){\includegraphics*[width=2mm, height=2mm]{markers/sq-blue-open}}
    \put(92,50){\includegraphics*[width=2mm, height=2mm]{markers/sq-red-filled}}
    \put(95,51){\tiny $\chibThreeP \to \Y2S \gamma$}    
   
  % \graphpaper[5](0,0)(150, 120)
  \end{picture}
  }
  \caption{\small
    Fraction of $\Upsilon(2,3S)$ originated from $\chi_b(2,3P)$ decays in the specified
    $p_T^{\Upsilon}$ ranges (enlarged parts of \Cref{fig:results}). Outer error bars show
    statistical and systematics errors, inner error bars --- only statistical
    errors.}
  \label{fig:results_23p}
\end{figure}
\begin{figure}[H]
  \setlength{\unitlength}{1mm}
  \centering
  \resizebox{0.75\textwidth}{!}{
  \begin{picture}(80,60)
    %% =======================================================================
    \put(0,0){
      \includegraphics*[width=75mm, height=60mm]{results/ups1s_1_old}
    }
    \put(2,15){\begin{sideways}\Y1S fraction, \% \end{sideways}}
    \put(40,0){$p_T^{\Y1S} \left[\gevc\right]$}

    \put(40,30){\scriptsize $\chibOneP \to \Y1S \gamma$}
    
    \put(45,26){\scriptsize \textcolor{blue}{\sqs=7\tev}}
    \put(45,22){\scriptsize \textcolor{red}{\sqs=8\tev}}
    \put(45,18){\scriptsize \sqs=7\tev (2010)}
    
    \put(40,26){\includegraphics*[width=4mm, height=2mm]{blue}}
    \put(40,22){\includegraphics*[width=4mm, height=2mm]{red}}
    \put(40,18){\includegraphics*[width=4mm, height=2mm]{black}}

    
  % \graphpaper[5](0,0)(80, 60)
  \end{picture}
  }
  \caption{\small
    Fracton of $\Y1S$ originated from \chibOneP decays with comparison to the
    previous results. Outer error bars show  statistical and systematics
    errors, inner error bars --- only statistical errors. }
  \label{fig:results_2010}
\end{figure}
%% Final tables ============ 
\begin{table}[H]
\centering
\caption{\small \Y1S fraction originating from \chib decay}
\subtable[$6 < p_T^{\Y1S} < 8 \gevc$] {
\scalebox{0.55}{
\begin{tabular}{lrr}\toprule
 & \multicolumn{2}{c}{$\Upsilon(1S)$ transverse momentum intervals, \gevc}\\
 & \multicolumn{2}{c}{6 -- 8}\\
\cmidrule(r){2-3}
 & \multicolumn{1}{c}{\sqs = 7\tev} & \multicolumn{1}{c}{\sqs = 8\tev}\\
\midrule
$\chi_b(1P) \to \Y1S \gamma$ & 23.0 $\pm$ 1.6\stat${}^{+1.1}_{-1.3}\syst^{+0.9}_{-0.6}\systpol \%$ & 22.9 $\pm$ 1.0\stat${}^{+1.1}_{-1.4}\syst^{+0.9}_{-0.6}\systpol \%$\\

\rule{0pt}{4ex}$\chi_b(2P) \to \Y1S \gamma$ & 3.8 $\pm$ 0.8\stat${}^{+0.2}_{-0.2}\syst^{+0.1}_{-0.0}\systpol \%$ & 2.9 $\pm$ 0.5\stat${}^{+0.2}_{-0.1}\syst^{+0.1}_{-0.0}\systpol \%$\\

\rule{0pt}{4ex}$\chi_b(3P) \to \Y1S \gamma$ & --- & ---\\
\bottomrule
\end{tabular}
} % scalebox

} % subtable
\subtable[$8 < p_T^{\Y1S} < 10 \gevc$] {
\scalebox{0.55}{
\begin{tabular}{lrr}\toprule
 & \multicolumn{2}{c}{$\Upsilon(1S)$ transverse momentum intervals, \gevc}\\
 & \multicolumn{2}{c}{8 -- 10}\\
\cmidrule(r){2-3}
 & \multicolumn{1}{c}{\sqs = 7\tev} & \multicolumn{1}{c}{\sqs = 8\tev}\\
\midrule
$\chi_b(1P) \to \Y1S \gamma$ & 24.7 $\pm$ 1.7\stat${}^{+1.3}_{-1.6}\syst^{+1.3}_{-0.9}\systpol \%$ & 25.2 $\pm$ 1.2\stat${}^{+1.2}_{-1.4}\syst^{+1.3}_{-0.9}\systpol \%$\\

\rule{0pt}{4ex}$\chi_b(2P) \to \Y1S \gamma$ & 6.7 $\pm$ 0.9\stat${}^{+0.3}_{-0.3}\syst^{+0.1}_{-0.1}\systpol \%$ & 6.1 $\pm$ 0.6\stat${}^{+0.3}_{-0.2}\syst^{+0.1}_{-0.1}\systpol \%$\\

\rule{0pt}{4ex}$\chi_b(3P) \to \Y1S \gamma$ & --- & ---\\
\bottomrule
\end{tabular}
} % scalebox

} % subtable
\subtable[$10 < p_T^{\Y1S} < 14 \gevc$] {
\scalebox{0.55}{
\begin{tabular}{lrr}\toprule
 & \multicolumn{2}{c}{$\Upsilon(1S)$ transverse momentum intervals, \gevc}\\
 & \multicolumn{2}{c}{10 -- 14}\\
\cmidrule(r){2-3}
 & \multicolumn{1}{c}{\sqs = 7\tev} & \multicolumn{1}{c}{\sqs = 8\tev}\\
\midrule
$\chi_b(1P) \to \Y1S \gamma$ & 26.4 $\pm$ 1.4\stat${}^{+1.1}_{-1.4}\syst^{+0.9}_{-0.8}\systpol \%$ & 29.3 $\pm$ 0.9\stat${}^{+1.3}_{-1.5}\syst^{+1.0}_{-0.9}\systpol \%$\\

\rule{0pt}{4ex}$\chi_b(2P) \to \Y1S \gamma$ & 6.3 $\pm$ 0.8\stat${}^{+0.3}_{-0.2}\syst^{+0.1}_{-0.0}\systpol \%$ & 4.3 $\pm$ 0.6\stat${}^{+0.2}_{-0.2}\syst^{+0.0}_{-0.0}\systpol \%$\\

\rule{0pt}{4ex}$\chi_b(3P) \to \Y1S \gamma$ & 2.0 $\pm$ 0.7\stat${}^{+0.2}_{-0.3}\syst^{+0.0}_{-0.0}\systpol \%$ & 1.5 $\pm$ 0.5\stat${}^{+0.2}_{-0.2}\syst^{+0.0}_{-0.0}\systpol \%$\\
\bottomrule
\end{tabular}
} % scalebox

} % subtable
\subtable[$14 < p_T^{\Y1S} < 18 \gevc$] {
\scalebox{0.55}{
\begin{tabular}{lrr}\toprule
 & \multicolumn{2}{c}{$\Upsilon(1S)$ transverse momentum intervals, \gevc}\\
 & \multicolumn{2}{c}{14 -- 18}\\
\cmidrule(r){2-3}
 & \multicolumn{1}{c}{\sqs = 7\tev} & \multicolumn{1}{c}{\sqs = 8\tev}\\
\midrule
$\chi_b(1P) \to \Y1S \gamma$ & 30.5 $\pm$ 1.6\stat${}^{+1.1}_{-1.2}\syst^{+0.3}_{-0.3}\systpol \%$ & 30.1 $\pm$ 1.1\stat${}^{+1.1}_{-1.2}\syst^{+0.3}_{-0.3}\systpol \%$\\

\rule{0pt}{4ex}$\chi_b(2P) \to \Y1S \gamma$ & 6.8 $\pm$ 1.0\stat${}^{+0.3}_{-0.4}\syst^{+0.2}_{-0.2}\systpol \%$ & 6.3 $\pm$ 0.7\stat${}^{+0.3}_{-0.3}\syst^{+0.2}_{-0.2}\systpol \%$\\

\rule{0pt}{4ex}$\chi_b(3P) \to \Y1S \gamma$ & 2.4 $\pm$ 0.9\stat${}^{+0.3}_{-0.4}\syst^{+0.1}_{-0.1}\systpol \%$ & 1.5 $\pm$ 0.6\stat${}^{+0.1}_{-0.1}\syst^{+0.1}_{-0.1}\systpol \%$\\
\bottomrule
\end{tabular}
} % scalebox

} % subtable
\subtable[$18 < p_T^{\Y1S} < 22 \gevc$] {
\scalebox{0.55}{
\begin{tabular}{lrr}\toprule
 & \multicolumn{2}{c}{$\Upsilon(1S)$ transverse momentum intervals, \gevc}\\
 & \multicolumn{2}{c}{18 -- 22}\\
\cmidrule(r){2-3}
 & \multicolumn{1}{c}{\sqs = 7\tev} & \multicolumn{1}{c}{\sqs = 8\tev}\\
\midrule
$\chi_b(1P) \to \Y1S \gamma$ & 33.2 $\pm$ 2.5\stat${}^{+1.1}_{-1.3}\syst^{+0.6}_{-0.8}\systpol \%$ & 31.0 $\pm$ 1.6\stat${}^{+1.1}_{-1.3}\syst^{+0.5}_{-0.7}\systpol \%$\\

\rule{0pt}{4ex}$\chi_b(2P) \to \Y1S \gamma$ & 6.6 $\pm$ 1.2\stat${}^{+0.3}_{-0.3}\syst^{+0.4}_{-0.4}\systpol \%$ & 4.9 $\pm$ 0.8\stat${}^{+0.2}_{-0.2}\syst^{+0.3}_{-0.3}\systpol \%$\\

\rule{0pt}{4ex}$\chi_b(3P) \to \Y1S \gamma$ & 2.0 $\pm$ 0.9\stat${}^{+0.4}_{-0.3}\syst^{+0.1}_{-0.1}\systpol \%$ & 1.2 $\pm$ 0.6\stat${}^{+0.1}_{-0.1}\syst^{+0.1}_{-0.1}\systpol \%$\\
\bottomrule
\end{tabular}
} % scalebox

} % subtable
\subtable[$22 < p_T^{\Y1S} < 40 \gevc$] {
\scalebox{0.55}{
\begin{tabular}{lrr}\toprule
 & \multicolumn{2}{c}{$\Upsilon(1S)$ transverse momentum intervals, \gevc}\\
 & \multicolumn{2}{c}{22 -- 40}\\
\cmidrule(r){2-3}
 & \multicolumn{1}{c}{\sqs = 7\tev} & \multicolumn{1}{c}{\sqs = 8\tev}\\
\midrule
$\chi_b(1P) \to \Y1S \gamma$ & 34.6 $\pm$ 2.9\stat${}^{+1.2}_{-1.3}\syst^{+1.0}_{-1.4}\systpol \%$ & 33.5 $\pm$ 2.3\stat${}^{+1.3}_{-1.9}\syst^{+1.0}_{-1.3}\systpol \%$\\

\rule{0pt}{4ex}$\chi_b(2P) \to \Y1S \gamma$ & 7.4 $\pm$ 1.3\stat${}^{+0.3}_{-0.5}\syst^{+0.4}_{-0.5}\systpol \%$ & 8.6 $\pm$ 1.1\stat${}^{+0.5}_{-0.3}\syst^{+0.5}_{-0.6}\systpol \%$\\

\rule{0pt}{4ex}$\chi_b(3P) \to \Y1S \gamma$ & 3.6 $\pm$ 1.1\stat${}^{+0.1}_{-0.3}\syst^{+0.2}_{-0.2}\systpol \%$ & 2.4 $\pm$ 0.6\stat${}^{+0.4}_{-0.2}\syst^{+0.2}_{-0.1}\systpol \%$\\
\bottomrule
\end{tabular}
} % scalebox

} % subtable
\label{tab:frac:ups1s_final}
\end{table}

\begin{table}[H]
\centering
\caption{\small \Y2S fraction originating from \chib decay}
\subtable[$18 < p_T^{\Y2S} < 22 \gevc$] {
\scalebox{0.55}{
\begin{tabular}{lrr}\toprule
 & \multicolumn{2}{c}{$\Upsilon(2S)$ transverse momentum intervals, \gevc}\\
 & \multicolumn{2}{c}{18 -- 22}\\
\cmidrule(r){2-3}
 & \multicolumn{1}{c}{\sqs = 7\tev} & \multicolumn{1}{c}{\sqs = 8\tev}\\
\midrule
$\chi_b(2P) \to \Y2S \gamma$ & 31 $\pm$ 5\stat${}^{+1.1}_{-2.0}\syst^{+2.7}_{-2.4}\systpol \%$ & 34 $\pm$ 4\stat${}^{+1.0}_{-1.6}\syst^{+2.9}_{-2.6}\systpol \%$\\

\rule{0pt}{4ex}$\chi_b(3P) \to \Y2S \gamma$ & --- & ---\\
\bottomrule
\end{tabular}
} % scalebox

} % subtable
\subtable[$18 < p_T^{\Y2S} < 24 \gevc$] {
\scalebox{0.55}{
\begin{tabular}{lrr}\toprule
 & \multicolumn{2}{c}{$\Upsilon(2S)$ transverse momentum intervals, \gevc}\\
 & \multicolumn{2}{c}{18 -- 24}\\
\cmidrule(r){2-3}
 & \multicolumn{1}{c}{\sqs = 7\tev} & \multicolumn{1}{c}{\sqs = 8\tev}\\
\midrule
$\chi_b(2P) \to \Y2S \gamma$ & --- & ---\\

\rule{0pt}{4ex}$\chi_b(3P) \to \Y2S \gamma$ & 3.5 $\pm$ 2.2\stat${}^{+0.2}_{-0.5}\syst^{+0.1}_{-0.1}\systpol \%$ & 3.3 $\pm$ 1.3\stat${}^{+0.1}_{-0.3}\syst^{+0.1}_{-0.1}\systpol \%$\\
\bottomrule
\end{tabular}
} % scalebox

} % subtable
\subtable[$22 < p_T^{\Y2S} < 24 \gevc$] {
\scalebox{0.55}{
\begin{tabular}{lrr}\toprule
 & \multicolumn{2}{c}{$\Upsilon(2S)$ transverse momentum intervals, \gevc}\\
 & \multicolumn{2}{c}{22 -- 24}\\
\cmidrule(r){2-3}
 & \multicolumn{1}{c}{\sqs = 7\tev} & \multicolumn{1}{c}{\sqs = 8\tev}\\
\midrule
$\chi_b(2P) \to \Y2S \gamma$ & 34 $\pm$ 11\stat${}^{+1.3}_{-2.6}\syst^{+1.2}_{-2.1}\systpol \%$ & 31 $\pm$ 7\stat${}^{+1.2}_{-2.1}\syst^{+1.1}_{-1.9}\systpol \%$\\

\rule{0pt}{4ex}$\chi_b(3P) \to \Y2S \gamma$ & --- & ---\\
\bottomrule
\end{tabular}
} % scalebox

} % subtable
\subtable[$24 < p_T^{\Y2S} < 40 \gevc$] {
\scalebox{0.55}{
\begin{tabular}{lrr}\toprule
 & \multicolumn{2}{c}{$\Upsilon(2S)$ transverse momentum intervals, \gevc}\\
 & \multicolumn{2}{c}{24 -- 40}\\
\cmidrule(r){2-3}
 & \multicolumn{1}{c}{\sqs = 7\tev} & \multicolumn{1}{c}{\sqs = 8\tev}\\
\midrule
$\chi_b(2P) \to \Y2S \gamma$ & 26 $\pm$ 7\stat${}^{+0.8}_{-1.4}\syst^{+1.1}_{-1.2}\systpol \%$ & 31 $\pm$ 4\stat${}^{+1.0}_{-1.4}\syst^{+1.4}_{-1.4}\systpol \%$\\

\rule{0pt}{4ex}$\chi_b(3P) \to \Y2S \gamma$ & 9.0 $\pm$ 3.3\stat${}^{+0.3}_{-1.0}\syst^{+0.4}_{-0.4}\systpol \%$ & 3.0 $\pm$ 1.6\stat${}^{+0.6}_{-0.6}\syst^{+0.1}_{-0.1}\systpol \%$\\
\bottomrule
\end{tabular}
} % scalebox

} % subtable
\label{tab:frac:ups2s_final}
\end{table}

\begin{table}[H]
\caption{\small \Y3S fraction originating from \chib decay}
\centering
\scalebox{0.55}{
\begin{tabular}{lrr}\toprule
 & \multicolumn{2}{c}{$\Upsilon(3S)$ transverse momentum intervals, \gevc}\\
 & \multicolumn{2}{c}{27 -- 40}\\
\cmidrule(r){2-3}
 & \multicolumn{1}{c}{\sqs = 7\tev} & \multicolumn{1}{c}{\sqs = 8\tev}\\
\midrule
$\chi_b(3P) \to \Y3S \gamma$ & 42 $\pm$ 12\stat${}^{+8.9}_{-11.6}\syst^{+2.7}_{-3.1}\systpol \%$ & 41 $\pm$ 8\stat${}^{+1.3}_{-8.6}\syst^{+2.6}_{-3.1}\systpol \%$\\
\bottomrule
\end{tabular}
} % scalebox
\label{tab:frac:ups3s_final}
\end{table}


The results in this study extend previous \lhcb measurements to considerably
more decays, higher transverse momentum regions and increased statistical
precision. The measurement of the 
\Y3S production fraction due to radiative \chibThreeP decays is performed for the first time.

Also, in this study the \chiboneThreeP mass was measured to be $10{,}511.3 \pm
1.7\stat \pm 2.4 \syst \mevcc$, which is in good agreement with a recent
unpublished \lhcb measurement with converted photons, where the \chiboneThreeP mass is
$10{,}515 \pm 3\stat{}^{+1.5}_{-2.1}\syst$.
