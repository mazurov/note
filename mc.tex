
\section{Simulation}
\label{sec:mc}

% =============================================================================
\subsection{Data - simulation comparison}
\label{sec:mc:datavsmc}

A comparison of the distribution of the relevant observables used in this
analysis was performed on real and simulated data, in order to assess the
reliability of Monte Carlo in computing efficiencies. It should be stressed
that, since a relative branching fraction is measured, systematic effect cancel
out at first order. As expected, there are not many differences observed between
the simulation distributions, since the generated \chibOneP, \chibTwoP and
\chibThreeP differ only by \chib mass value.

Combinatorial background has been subtracted in real data by using 
the \sPlot\  technique~\cite{Pivk:2004ty}.
The resulting signal weights are used to obtain the signal distribution for each
relevant variable. 

% These distributions are compared with simulation, as
% shown in Figures~[\ref{fig:mc:datavsmc:compare1}, \ref{fig:mc:datavsmc:compare2}].

In \Cref{fig:mc:datavsmc:compare1,fig:mc:datavsmc:compare2}
these distributions  are shown for signals in $\chib \to \Y1S \gamma$ decays
compared with the corresponding simulated distributions.

\begin{figure}[H]
  \setlength{\unitlength}{1mm}
  \centering
  \begin{picture}(150,140)
    %
    \put(0,0){
      \includegraphics*[width=50mm, height=35mm]{mc/vsdata/cl_g_1p}
    }
    \put(50,0){
      \includegraphics*[width=50mm, height=35mm]{mc/vsdata/cl_g_2p}
    }
    \put(100,0){
      \includegraphics*[width=50mm, height=35mm]{mc/vsdata/cl_g_3p}
    }

    \put(0,35){
      \includegraphics*[width=50mm, height=35mm]{mc/vsdata/c2_dtf_1p}
    }
    \put(50,35){
      \includegraphics*[width=50mm, height=35mm]{mc/vsdata/c2_dtf_2p}
    }
    \put(100,35){
      \includegraphics*[width=50mm, height=35mm]{mc/vsdata/c2_dtf_3p}
    }
    \put(0,70){
      \includegraphics*[width=50mm, height=35mm]{mc/vsdata/pt_chib_1p}
    }
    \put(50,70){
      \includegraphics*[width=50mm, height=35mm]{mc/vsdata/pt_chib_2p}
    }
    \put(100,70){
      \includegraphics*[width=50mm, height=35mm]{mc/vsdata/pt_chib_3p}
    }
    \put(0,105){
      \includegraphics*[width=50mm, height=35mm]{mc/vsdata/pt_ups_1p}
    }
    \put(50,105){
      \includegraphics*[width=50mm, height=35mm]{mc/vsdata/pt_ups_2p}
    }
    \put(100,105){
      \includegraphics*[width=50mm, height=35mm]{mc/vsdata/pt_ups_3p}
    }

    \put(15,-1){\scriptsize $\gamma$ confidence level}
    \put(65,-1){\scriptsize $\gamma$ confidence level}
    \put(115,-1){\scriptsize $\gamma$ confidence level}
    \put(10,34){\scriptsize $\chisq$ of decay tree fitter}
    \put(60,34){\scriptsize $\chisq$ of decay tree fitter}
    \put(110,34){\scriptsize $\chisq$ of decay tree fitter}
    \put(20,69){\scriptsize $p_T[\chibOneP] \left[\gevcc\right]$}
    \put(70,69){\scriptsize $p_T[\chibTwoP] \left[\gevcc\right]$}
    \put(120,69){\scriptsize $p_T[\chibThreeP] \left[\gevcc\right]$}
    \put(20,104){\scriptsize $p_T[\Y1S] \left[\gevcc\right]$}
    \put(70,104){\scriptsize $p_T[\Y1S] \left[\gevcc\right]$}
    \put(120,104){\scriptsize $p_T[\Y1S] \left[\gevcc\right]$}

    \put(0,7){\scriptsize \begin{sideways}Arbitrary units\end{sideways}}
    \put(50,7){\scriptsize \begin{sideways}Arbitrary units\end{sideways}}
    \put(100,7){\scriptsize \begin{sideways}Arbitrary units\end{sideways}}
    \put(0,42){\scriptsize \begin{sideways}Arbitrary units\end{sideways}}
    \put(50,42){\scriptsize \begin{sideways}Arbitrary units\end{sideways}}
    \put(100,42){\scriptsize \begin{sideways}Arbitrary units\end{sideways}}
    \put(0,77){\scriptsize \begin{sideways}Arbitrary units\end{sideways}}
    \put(50,77){\scriptsize \begin{sideways}Arbitrary units\end{sideways}}
    \put(100,77){\scriptsize \begin{sideways}Arbitrary units\end{sideways}}
    \put(0,112){\scriptsize \begin{sideways}Arbitrary units\end{sideways}}
    \put(50,112){\scriptsize \begin{sideways}Arbitrary units\end{sideways}}
    \put(100,112){\scriptsize \begin{sideways}Arbitrary units\end{sideways}}

    \put(35,27){\scriptsize \chibOneP}
    \put(85,27){\scriptsize \chibTwoP}
    \put(135,27){\scriptsize \chibThreeP}
    \put(35,62){\scriptsize \chibOneP}
    \put(85,62){\scriptsize \chibTwoP}
    \put(135,62){\scriptsize \chibThreeP}
    \put(35,97){\scriptsize \chibOneP}
    \put(85,97){\scriptsize \chibTwoP}
    \put(135,97){\scriptsize \chibThreeP}
    \put(35,132){\scriptsize \chibOneP}
    \put(85,132){\scriptsize \chibTwoP}
    \put(135,132){\scriptsize \chibThreeP}
 
    % \graphpaper[5](0,0)(150, 175)        
  \end{picture}
  \caption {\small 
    Data (\sqs = 8\tev) --- Monte Carlo values comparison. Square (blue) points
    with errors bars corresponds data values, open circle (red) points with
    errors bars corresponds to simulation values. }
  \label{fig:mc:datavsmc:compare1}
\end{figure}


\begin{figure}[H]
  \setlength{\unitlength}{1mm}
  \centering
  \begin{picture}(150,175)
    %
    \put(0,0){
      \includegraphics*[width=50mm, height=35mm]{mc/vsdata/y_1p}
    }
    \put(50,0){
      \includegraphics*[width=50mm, height=35mm]{mc/vsdata/y_2p}
    }
    \put(100,0){
      \includegraphics*[width=50mm, height=35mm]{mc/vsdata/y_3p}
    }

    \put(0,35){
      \includegraphics*[width=50mm, height=35mm]{mc/vsdata/min_p_mu1_p_mu2_1p}
    }
    \put(50,35){
      \includegraphics*[width=50mm, height=35mm]{mc/vsdata/min_p_mu1_p_mu2_2p}
    }
    \put(100,35){
      \includegraphics*[width=50mm, height=35mm]{mc/vsdata/min_p_mu1_p_mu2_3p}
    }
    \put(0,70){
      \includegraphics*[width=50mm, height=35mm]{mc/vsdata/min_pt_mu1_pt_mu2_1p}
    }
    \put(50,70){
      \includegraphics*[width=50mm, height=35mm]{mc/vsdata/min_pt_mu1_pt_mu2_2p}
    }
    \put(100,70){
      \includegraphics*[width=50mm, height=35mm]{mc/vsdata/min_pt_mu1_pt_mu2_3p}
    }
    \put(0,105){
      \includegraphics*[width=50mm, height=35mm]{mc/vsdata/dll_min_1p}
    }
    \put(50,105){
      \includegraphics*[width=50mm, height=35mm]{mc/vsdata/dll_min_2p}
    }
    \put(100,105){
      \includegraphics*[width=50mm, height=35mm]{mc/vsdata/dll_min_3p}
    }
    \put(0,140){
      \includegraphics*[width=50mm, height=35mm]{mc/vsdata/chi2_vx_1p}
    }
    \put(50,140){
      \includegraphics*[width=50mm, height=35mm]{mc/vsdata/chi2_vx_2p}
    }
    \put(100,140){
      \includegraphics*[width=50mm, height=35mm]{mc/vsdata/chi2_vx_3p}
    }    

    \put(20,-1){\scriptsize \Y1S rapidity}
    \put(70,-1){\scriptsize \Y1S rapidity}
    \put(120,-1){\scriptsize \Y1S rapidity}
    \put(10,34){\scriptsize $min[p(\mu^+), p(\mu^-)] [\gevc]$ }
    \put(60,34){\scriptsize $min[p(\mu^+), p(\mu^-)] [\gevc]$}
    \put(110,34){\scriptsize $min[p(\mu^+), p(\mu^-)] [\gevc]$}
    \put(10,69){\scriptsize $min[p_T(\mu^+), p_T(\mu^-)] [\gevc]$}
    \put(60,69){\scriptsize $min[p_T(\mu^+), p_T(\mu^-)] [\gevc]$}
    \put(110,69){\scriptsize $min[p_T(\mu^+), p_T(\mu^-)] [\gevc]$}
    \put(20,104){\scriptsize $\Delta\log\lum^{\mmu-\Ph}$}
    \put(70,104){\scriptsize $\Delta\log\lum^{\mmu-\Ph}$}
    \put(120,104){\scriptsize $\Delta\log\lum^{\mmu-\Ph}$}
    \put(10,139){\scriptsize $\chisq$ for decay vertex fitter}
    \put(60,139){\scriptsize $\chisq$ for decay vertex fitter}
    \put(110,139){\scriptsize $\chisq$ for decay vertex fitter}

    \put(0,7){\scriptsize \begin{sideways}Arbitrary units\end{sideways}}
    \put(50,7){\scriptsize \begin{sideways}Arbitrary units\end{sideways}}
    \put(100,7){\scriptsize \begin{sideways}Arbitrary units\end{sideways}}
    \put(0,42){\scriptsize \begin{sideways}Arbitrary units\end{sideways}}
    \put(50,42){\scriptsize \begin{sideways}Arbitrary units\end{sideways}}
    \put(100,42){\scriptsize \begin{sideways}Arbitrary units\end{sideways}}
    \put(0,77){\scriptsize \begin{sideways}Arbitrary units\end{sideways}}
    \put(50,77){\scriptsize \begin{sideways}Arbitrary units\end{sideways}}
    \put(100,77){\scriptsize \begin{sideways}Arbitrary units\end{sideways}}
    \put(0,112){\scriptsize \begin{sideways}Arbitrary units\end{sideways}}
    \put(50,112){\scriptsize \begin{sideways}Arbitrary units\end{sideways}}
    \put(100,112){\scriptsize \begin{sideways}Arbitrary units\end{sideways}}
    \put(0,147){\scriptsize \begin{sideways}Arbitrary units\end{sideways}}
    \put(50,147){\scriptsize \begin{sideways}Arbitrary units\end{sideways}}
    \put(100,147){\scriptsize \begin{sideways}Arbitrary units\end{sideways}}

    \put(35,27){\scriptsize \chibOneP}
    \put(85,27){\scriptsize \chibTwoP}
    \put(135,27){\scriptsize \chibThreeP}
    \put(35,62){\scriptsize \chibOneP}
    \put(85,62){\scriptsize \chibTwoP}
    \put(135,62){\scriptsize \chibThreeP}
    \put(35,97){\scriptsize \chibOneP}
    \put(85,97){\scriptsize \chibTwoP}
    \put(135,97){\scriptsize \chibThreeP}
    \put(35,132){\scriptsize \chibOneP}
    \put(85,132){\scriptsize \chibTwoP}
    \put(135,132){\scriptsize \chibThreeP}
    \put(35,167){\scriptsize \chibOneP}
    \put(85,167){\scriptsize \chibTwoP}
    \put(135,167){\scriptsize \chibThreeP}
 
    % \graphpaper[5](0,0)(150, 175)        
  \end{picture}
  \caption {\small 
    Data (\sqs = 8\tev) --- Monte Carlo values comparison. Square (blue) points
    with errors bars corresponds data values, open circle (red) points with
    errors bars corresponds to simulation values. }
  \label{fig:mc:datavsmc:compare2}
\end{figure}

\begin{figure}[H]
  \setlength{\unitlength}{1mm}
  \centering
  \begin{picture}(150,35)
    %
    \put(0,0){
      \includegraphics*[width=50mm, height=35mm]{mc/vsdata/pt_g_1p}
    }
    \put(50,0){
      \includegraphics*[width=50mm, height=35mm]{mc/vsdata/pt_g_2p}
    }
    \put(100,0){
      \includegraphics*[width=50mm, height=35mm]{mc/vsdata/pt_g_3p}
    }

    \put(20,-1){\scriptsize $p_T(\gamma)$ $\left[\gevc\right]$}
    \put(70,-1){\scriptsize $p_T(\gamma)$ $\left[\gevc\right]$}
    \put(120,-1){\scriptsize $p_T(\gamma)$ $\left[\gevc\right]$}


    \put(0,7){\scriptsize \begin{sideways}Arbitrary units\end{sideways}}
    \put(50,7){\scriptsize \begin{sideways}Arbitrary units\end{sideways}}
    \put(100,7){\scriptsize \begin{sideways}Arbitrary units\end{sideways}}

    \put(35,27){\scriptsize \chibOneP}
    \put(85,27){\scriptsize \chibTwoP}
    \put(135,27){\scriptsize \chibThreeP}
 
    % \graphpaper[5](0,0)(150, 175)        
  \end{picture}
  \caption {\small 
    Data (\sqs = 8\tev) - Monte Carlo values comparison. Square (blue) points
    with errors bars corresponds data values, open circle (red) points with
    errors bars corresponds to simulation values. }
  \label{fig:mc:datavsmc:gamma}
\end{figure}


The agreement is generally very good except for the distribution of photon
transverse momentum~(\Cref{fig:mc:datavsmc:gamma}). This is due to the \sPlot\ 
technique, when applied on variables that affect the background shape. In our
study the background in the fit of the invariant mass distribution 
depends on the photon transverse momentum,
hence a mismatch between data and simulation is expected. Other discrepancies
observed in the distributions for \chibThreeP decays are possibly due a poor
signal to background ratio, which translates into large systematic
uncertainties in the sWeights. 


% =============================================================================
\subsection{Selection efficiencies}
\label{sec:mc:eff}

The distributions of the invariant mass difference of truth-matched MC events in the
\chib simulation are shown in~\Cref{fig:mc:eff:nominal}. The flat left tails 
are due to photons which, although being correctly associated to the \chib decay,
are poorly reconstructed in the calorimeter (due to e.g. cracks, spillover, cross-talk, 
etc.). In principle, these tails could be modeled in our fit to signal, but in
practice they will not be distinguishable from background. Therefore, the
number of \chib events for efficiency calculations is not determined by simple
event counting but from a fit where the tails are considered as background. In
this fit the signal is described by a CrystalBall function and the background is
a product of first order polynomial and exponential functions. The number of $\Upsilon$
events is obtained by counting all truth-matched MC $\Upsilon$ events.

\begin{figure}[H]
  \setlength{\unitlength}{1mm}
  \centering
  \scalebox{0.5}{
  \begin{picture}(225,120)
    \put(0,60){
      \includegraphics*[width=75mm, height=60mm]{mc/eff/cb11_6_40}
    }
    \put(42,114){\scriptsize $\chiboneOneP \to \Y1S \gamma$}
    \put(42,109){\scriptsize $6 < p_T^{\Y1S} < 40 \gevc$}
    \put(10,73){$m_{\mumu \gamma} - m_{\mumu} + 9.4603 \left[\gevcc\right]$}
    \put(3,85){\scriptsize \begin{sideways}Candidates/(10\mevcc)\end{sideways}}    
    \put(45,103){\scriptsize N =34,330 $\pm$ 220}
    \put(45,100){\scriptsize B/N = 15.3 $\pm$ 0.4\%}
    

    \put(75,60){
      \includegraphics*[width=75mm, height=60mm]{mc/eff/cb12_6_40}
    }
    \put(117,114){\scriptsize $\chiboneTwoP \to \Y1S \gamma$}
    \put(117,109){\scriptsize $6 < p_T^{\Y1S} < 40 \gevc$}
    \put(85,73){$m_{\mumu \gamma} - m_{\mumu} + 9.4603 \left[\gevcc\right]$}
    \put(78,85){\scriptsize \begin{sideways}Candidates/(10\mevcc)\end{sideways}}    
    \put(120,103){\scriptsize N =22,210 $\pm$ 170}
    \put(120,100){\scriptsize B/N = 10.3 $\pm$ 0.4\%}
    

    \put(150,60){
      \includegraphics*[width=75mm, height=60mm]{mc/eff/cb13_6_40}
    }
    \put(192,114){\scriptsize $\chiboneThreeP \to \Y1S \gamma$}
    \put(192,109){\scriptsize $6 < p_T^{\Y1S} < 40 \gevc$}
    \put(160,73){$m_{\mumu \gamma} - m_{\mumu} + 9.4603 \left[\gevcc\right]$}
    \put(153,85){\scriptsize \begin{sideways}Candidates/(10\mevcc)\end{sideways}}    
    \put(195,103){\scriptsize N =15,110 $\pm$ 130}
    \put(195,100){\scriptsize B/N = 9.0 $\pm$ 0.4\%}
    

    \put(0,0){
      \includegraphics*[width=75mm, height=60mm]{mc/eff/cb12_18_40}
    }
    \put(42,54){\scriptsize $\chiboneTwoP \to \Y2S \gamma$}
    \put(42,49){\scriptsize $18 < p_T^{\Y1S} < 40 \gevc$}
    \put(10,13){$m_{\mumu \gamma} - m_{\mumu} + 10.02326 \left[\gevcc\right]$}
    \put(3,25){\scriptsize \begin{sideways}Candidates/(10\mevcc)\end{sideways}}    
    \put(45,43){\scriptsize N =194 $\pm$ 21}
    \put(45,40){\scriptsize B/N = 8 $\pm$ 9\%}
    

    \put(75,0){
      \includegraphics*[width=75mm, height=60mm]{mc/eff/cb13_18_40}
    }
    \put(117,54){\scriptsize $\chiboneThreeP \to \Y2S \gamma$}
    \put(117,49){\scriptsize $18 < p_T^{\Y1S} < 40 \gevc$}
    \put(85,13){$m_{\mumu \gamma} - m_{\mumu} + 10.02326 \left[\gevcc\right]$}
    \put(78,25){\scriptsize \begin{sideways}Candidates/(10\mevcc)\end{sideways}}    
    \put(120,43){\scriptsize N =672 $\pm$ 32}
    \put(120,40){\scriptsize B/N = 8.5 $\pm$ 3.1\%}
    

    \put(150,0){
      \includegraphics*[width=75mm, height=60mm]{mc/eff/cb13_27_40}
    }
    \put(192,54){\scriptsize $\chiboneThreeP \to \Y3S \gamma$}
    \put(192,49){\scriptsize $27 < p_T^{\Y1S} < 40 \gevc$}
    \put(160,13){$m_{\mumu \gamma} - m_{\mumu} + 10.355 \left[\gevcc\right]$}
    \put(153,25){\scriptsize \begin{sideways}Candidates/(10\mevcc)\end{sideways}}    
    \put(195,43){\scriptsize N =154 $\pm$ 17}
    \put(195,40){\scriptsize B/N = 73 $\pm$ 13\%}
    

     % \graphpaper[5](0,0)(225, 120)        
  \end{picture}
  }
  \caption {\small 
    Distribution of the mass difference $\mumu \gamma - \mumu$ for matched
    $\chi_{b1}(1,2,3P)$ candidates in $\chib \to \Upsilon \gamma$ decays (black
    points) together with the result of the fit (solid red curve), including
    background (dotted blue curve) contribution.The pull is defined as the
    difference  between the data and fit value divided by the data error. }
  \label{fig:mc:eff:nominal}
\end{figure}

\Cref{fig:mc:eff} shows the obtained efficiency of \chib reconstruction.
Photon is more energetic as $p_T(\Upsilon)$ increases so it is reconstructed more
efficiently. More details on these measurements are shown
in~\Crefrange{tab:mc:eff:chib1_y1}{tab:mc:eff:chib3_y3} in Appendix.


\begin{figure}[H]
  \setlength{\unitlength}{1mm}
  \centering
  \scalebox{ 0.7 }{
  \begin{picture}(225,60)
    
    %% =======================================================================
    \put(0,0){
      \includegraphics*[width=75mm, height=60mm]{mc/eff/cb_ups1s}
    }
    \put(2,25){\begin{sideways}Efficiency\end{sideways}}
    \put(35,2){$p_T^{\Y1S} \left[\gevc\right]$}
    \put(50,25){\textcolor{blue}{\chibOneP}}
    \put(50,20){\textcolor{red}{\chibTwoP}}
    \put(50,15){\textcolor{cyan}{\chibThreeP}}
    
    \put(45,25){\includegraphics*[width=3mm, height=2mm]{bsf}}
    \put(45,20){\includegraphics*[width=3mm, height=2mm]{rco}}
    \put(45,15){\includegraphics*[width=3mm, height=2mm]{ctuc}}

    \put(45,30){$\chib \to \Y1S \gamma$}
    
    %% =======================================================================
    \put(75,0){
      \includegraphics*[width=75mm, height=60mm]{mc/eff/cb_ups2s}
    }
    \put(77,25){\begin{sideways}Efficiency\end{sideways}}
    \put(110,2){$p_T^{\Y2S} \left[\gevc\right]$}
    
    \put(125,20){\textcolor{red}{\chibTwoP}}
    \put(125,15){\textcolor{cyan}{\chibThreeP}}
    
    
    \put(120,20){\includegraphics*[width=3mm, height=2mm]{rco}}
    \put(120,15){\includegraphics*[width=3mm, height=2mm]{ctuc}}

    \put(120,30){$\chib \to \Y2S \gamma$}
    
    %% =======================================================================
    \put(150,0){
      \includegraphics*[width=75mm, height=60mm]{mc/eff/cb_ups3s}
    }
    \put(152,25){\begin{sideways}Efficiency\end{sideways}}
    \put(185,2){$p_T^{\Y3S} \left[\gevc\right]$}
    
    
    \put(200,15){\textcolor{cyan}{\chibThreeP}}
    
    
    
    \put(195,15){\includegraphics*[width=3mm, height=2mm]{ctuc}}

    \put(195,30){$\chib \to \Y3S \gamma$}
  % \graphpaper[5](0,0)(75, 60)
  \end{picture}
  }
  \caption {\small
    The \chib reconstruction efficiency as function of $p_T^{\Upsilon}$
  }
  \label{fig:mc:eff}
\end{figure}
