
\section{Simulation}
\label{sec:mc}

% =============================================================================
\subsection{Data - simulation comparison}
\label{sec:mc:datavsmc}

A comparison of the distribution of the relevant observables used in this
analysis was performed on real and simulated data, in order to assess the
reliability of Monte Carlo in computing efficiencies. It should be stressed
that, since a relative branching fraction is measured, systematic effects cancel
at the first order.

Combinatorial background has been subtracted in real data by 
\sPlot  technique~\cite{Pivk:2004ty}.
The resulting signal weights are used to obtain the signal distribution for each
relevant variable. 

% These distributions are compared with simulation, as
% shown in Figures~[\ref{fig:mc:datavsmc:compare1}, \ref{fig:mc:datavsmc:compare2}].

In \Cref{fig:mc:datavsmc:compare1,fig:mc:datavsmc:compare2}
these distributions  are shown for signals in $\chib \to \Y1S \gamma$ decays
compared with the corresponding simulated distributions.

\input{pics/mc/vsdata}


The agreement is generally very good except the distribution of photon
transverse momentum~(\Cref{fig:mc:datavsmc:gamma}). Imperfect  distribution of
photon transverse momentum can be explained by sPlot behavior in case when this
technique is applied on variables that affect the background form. In our study
the background in the fit depends on the photon transverse momentum, so the
mismatch with simulation was expected. Some discrepancy observed in figures for
\chibThreeP decay can be caused by large systematic uncertainties for this
decay due to the low signal to background ratio.

% =============================================================================
\subsection{Selection efficiencies}
\label{sec:mc:eff}

The invariant mass difference distributions of matched MC-true events in the
\chib simulation are shown in~\Cref{fig:mc:eff:nominal}. The left tails are
almost flat at these plots, so in the \chib fit model for real data, described
in~\Cref{sec:chib:fit}, the events corresponding to these tails are matched as
background. Therefore, the number of \chib events for efficiency calculation
was decided to obtain from the fit that count events in tails as a background.
In this fit the signal is described by CrystalBall function and the background
is a product of first order polynomial and exponential. The number of
$\Upsilon$ events is obtained by counting all matched MC-true $\Upsilon$
events.


\input{pics/mc/nominal_fits}

\Cref{fig:mc:eff} shows the measured efficiency of \chib reconstruction.
More details on these measurements are shown
in~\Crefrange{tab:mc:eff:chib1_y1}{tab:mc:eff:chib3_y3} in Appendix.


\begin{figure}[H]
  \setlength{\unitlength}{1mm}
  \centering
  \scalebox{ 0.7 }{
  \begin{picture}(225,60)
    
    %% =======================================================================
    \put(0,0){
      \includegraphics*[width=75mm, height=60mm]{mc/eff/cb_ups1s}
    }
    \put(2,25){\begin{sideways}Efficiency\end{sideways}}
    \put(35,2){$p_T^{\Y1S} \left[\gevc\right]$}
    \put(50,25){\textcolor{blue}{\chibOneP}}
    \put(50,20){\textcolor{red}{\chibTwoP}}
    \put(50,15){\textcolor{cyan}{\chibThreeP}}
    
    \put(45,25){\includegraphics*[width=3mm, height=2mm]{bsf}}
    \put(45,20){\includegraphics*[width=3mm, height=2mm]{rco}}
    \put(45,15){\includegraphics*[width=3mm, height=2mm]{ctuc}}

    \put(45,30){$\chib \to \Y1S \gamma$}
    
    %% =======================================================================
    \put(75,0){
      \includegraphics*[width=75mm, height=60mm]{mc/eff/cb_ups2s}
    }
    \put(77,25){\begin{sideways}Efficiency\end{sideways}}
    \put(110,2){$p_T^{\Y2S} \left[\gevc\right]$}
    
    \put(125,20){\textcolor{red}{\chibTwoP}}
    \put(125,15){\textcolor{cyan}{\chibThreeP}}
    
    
    \put(120,20){\includegraphics*[width=3mm, height=2mm]{rco}}
    \put(120,15){\includegraphics*[width=3mm, height=2mm]{ctuc}}

    \put(120,30){$\chib \to \Y2S \gamma$}
    
    %% =======================================================================
    \put(150,0){
      \includegraphics*[width=75mm, height=60mm]{mc/eff/cb_ups3s}
    }
    \put(152,25){\begin{sideways}Efficiency\end{sideways}}
    \put(185,2){$p_T^{\Y3S} \left[\gevc\right]$}
    
    
    \put(200,15){\textcolor{cyan}{\chibThreeP}}
    
    
    
    \put(195,15){\includegraphics*[width=3mm, height=2mm]{ctuc}}

    \put(195,30){$\chib \to \Y3S \gamma$}
  % \graphpaper[5](0,0)(75, 60)
  \end{picture}
  }
  \caption {\small
    The \chib reconstruction efficiency as function of $p_T^{\Upsilon}$
  }
  \label{fig:mc:eff}
\end{figure}
