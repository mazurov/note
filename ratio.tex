\section{\texorpdfstring{$\chi_{b1}$}{chib1} and \texorpdfstring{$\chi_{b2}$}{chib2} yields ratio}
\label{sec:ratio}

\begin{figure}[H]
  \setlength{\unitlength}{1mm}
  \centering
  \begin{picture}(75,60)
    %
     \put(0,0){\includegraphics[width=75mm, height=60mm]{ratio/theory}}


     \put(-1,22){\begin{sideways}$\sigma({\chi_2})/\sigma({\chi_1}$)\end{sideways}}
     \put(46,30){\tiny $\chi_b$ scaled}
     \put(45,27){\tiny (this study, MC data)}

    % \graphpaper[5](0,0)(75, 60)
  \end{picture}
  \caption {\small This figure is taken from~\cite{Likhoded:2012hw} and shows
  transverse momentum distributions of the
$d\sigma\left[\chi_{2}\right]/d\sigma[\chi_{1}]$ ratio. Solid and dashed lines
stand for charmonium and bottomonium mesons. The dot-dashed line corresponds to
the rescaled bottomonium ratio:
$\sigma_{b2}/\sigma_{b1}(M_{\chi_c}/M_{\chi)b}\,p_T)$. The experimental results
for charmonium from LHCb \cite{LHCb-PAPER-2013-028} are shown with dots,
CDF~\cite{CDF:2007bra} --- with rectangles, and CMS~
\cite{CMS-PAS-BPH-11-010} --- with triangles. The scaled transverse momentum
distribution of \chib on Monte-Carlo data from this study is shown with open
circles. As it is seen, it almost matches the bottomonium curve. Thus the
results of this work have a reason to be used for fixing the fractions of
\chibone and \chibtwo yields on data. }
  \label{fig:frac:ratio}
\end{figure}

The \chibone and \chibtwo ratio is measured by the following formula in
specified transverse momentum intervals of \Y1S:
\begin{equation}
    \frac{N_{\chibtwo}^{data}}{N_{\chibone}^{data}} = \frac{\sigma(\chibtwo)}{\sigma(\chibone)}
    \frac{Br(\chibtwo\to\Upsilon\gamma)}{Br(\chibone\to\Upsilon\gamma)}\frac{\eps_{\chibtwo}}{\eps_{\chibone}}
\label{eqn:mc_ratio}
\end{equation}
, where $\sigma(\chibtwo) / \sigma(\chibone)$ is a ratio
from~\cite{Likhoded:2012hw}.

\begin{table}[H]
\caption{Branching ratios}
\centering
\begin{tabular}{l}
$Br_1[1P, 1S] = 35\% \pm 8\%$ \\
$Br_2[1P, 1S] = 22\% \pm 4\%$ \\
$Br_1[2P, 1S] = 8.5\% \pm 1.3\%$ \\
$Br_2[2P, 1S] = 7.1\% \pm 1\%$ \\
$Br_1[2P, 2S] = 21\% \pm 4\%$ \\
$Br_2[2P, 2S] = 16\% \pm 2.4\%$ \\
\end{tabular}
\label{tab:branching}
\end{table}


\input{tables/ratio/lambda}
