\section{Introduction}
\label{sec:introduction}

A significant fraction of the production cross-section of J/$\psi$ and
$\Upsilon$ states in hadron collisions is due to feed-down from heavier
quarkonium states. A study of this effect is important for the interpretation of
onia production cross section and polarization measurements in hadron
collisions. For P-wave quarkonia, measurements of \chic have been reported by
\cdf~\cite{Abulencia:2007bra}, HERA-B~\cite{Abt:2008ed}
and \lhcb~\cite{LHCb-PAPER-2011-019}, whereas \cdf~\cite{Affolder:1999wm} and 
\atlas~\cite{Aad:2011ih} have performed measurements involving $\chi_b$ states.
\lhcb has reported~\cite{LHCb-PAPER-2012-015} a preliminary measurement of
the $\chi_b$ production cross-section, and subsequent decay into \OneS $\gamma$,
relative to the \OneS production. This measurement was performed on 2011 data
in a region defined by $6 \gevc < \pt^{\Y1S} < 15 \gevc$ and rapidity range
$2.0 < y^{\Y1S} < 4.5$.
The corresponding integrated luminosity was 32.4 \invpb.

This note presents an update of the previous \lhcb\ study. Data collected in
2012 were also analyzed, allowing for cross-section measurements at \sqs=8\tev.
Using the full integrated luminosity allows for a measurement of the
differential cross-section in \pt and rapidity bins of the \OneS, and to study
the production of $\chi_b(2P)$ and $\chi_b(3P)$. A measurement of the
$\chi_b(3P)$ mass is also performed by combining data collected in 2011 and
2012.

The analysis proceeds through the reconstruction of $\Upsilon(nS)$ candidates
via their dimuon decays, and their subsequent pairing with a photon to look for
$\chi_b(mP) \to \Upsilon(nS) \gamma$ decays. Ratios of $\chi_b(mP)$ to
$\Upsilon(nS)$ production cross section can be written as

\begin{equation}
\resizebox{.9\hsize}{!}{
$
    \frac{\sigma(pp \to \chi_b(mP)
    	\to \Upsilon(nS) \gamma)}{\sigma(pp \to \Upsilon(nS))} =
    \frac{N_{\chi_b(mP)\to \Upsilon(nS) \gamma}}{N_{\Upsilon(nS)}} \times \frac{\epsilon_{\Upsilon(nS)}}{\epsilon_{\chi_b(mP)\to \Upsilon(nS) \gamma}} =
    \frac{N_{\chi_b(mP)\to \Upsilon(nS) \gamma}}{N_{\Upsilon(nS)}} \times \frac{1}{\epsilon^{reco}_{\gamma}}
$
}
\end{equation}


\noindent where
${N_{\Upsilon(nS)}}$ and ${N_{\chi_b(mP)\to \Upsilon(nS) \gamma}}$ are the
$\Upsilon(nS)$ and $\chi_b(mP)$ yields, $\epsilon_{\Upsilon(nS)}$ and
$\epsilon_{\chi_b(mP)\to \Upsilon(nS) \gamma}$ are their corresponding selection
efficiencies. The latter are the product of geometric acceptance, trigger
efficiency and reconstruction efficiency. Since the selection criteria for the
two samples differ only in the reconstruction of a photon, the efficiency ratio
can be replaced by 1/$\epsilon_{\gamma}$, the reconstruction efficiency for the
photon from the $\chi_b$ decay. Similar expressions may be used to compute
differential cross sections in $\Upsilon$ \pt and rapidity bins.