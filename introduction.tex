\section{Introduction}
\label{sec:introduction}

A significant fraction of the production cross-section of $\jpsi$ and
$\Upsilon$ states in hadron collisions is due to feed-down from heavier
quarkonium states. A study of this effect is important for the interpretation of
quarkonia production cross-section and polarization measurements in hadron
collisions. For P-wave quarkonia, measurements of \chic have been reported by
\cdf~\cite{Abulencia:2007bra}, HERA-B~\cite{Abt:2008ed}
and \lhcb~\cite{LHCb-PAPER-2011-019}, whereas \cdf~\cite{Affolder:1999wm} and 
\atlas~\cite{Aad:2011ih} have performed measurements involving $\chi_b$ states.
\lhcb has reported~\cite{LHCb-PAPER-2012-015} a measurement of
the \chibOneP production cross-section, and subsequent decay into \OneS $\gamma$,
relative to the \OneS production. This measurement was performed on 2010 data
in a region defined by $6 \gevc < \pt^{\Y1S} < 15 \gevc$ and
$2.0 < y^{\Y1S} < 4.5$. The corresponding integrated luminosity was $32.4\invpb$.

A substantial update of the previous \lhcb study is presented in this part of
the document. Data collected in 2011 and 2012 wre analyzed, resulting in a total
integrated luminosity of 3\invfb. Using the full integrated luminosity also allows
differential measurements in \pt bins of the $\Upsilon(1,2,3S)$ mesons, and to
study the production of radial excitations such as the $\chi_b(2P)$ and
$\chi_b(3P)$ mesons. A measurement of the mass of the $\chi_{b1}(3P)$, which was recently
observed at \atlas~\cite{Aad:2011ih}, D0~\cite{Abazov:2012gh} and
\lhcb~\cite{LHCb-CONF-2012-020} collaborations, is also performed in this study by
combining data collected in 2011 and 2012.

The analysis proceeds through the reconstruction of $\Upsilon(nS)$ candidates
via their dimuon decays, and their subsequent pairing with a photon to look for
$\chi_b(mP) \to \Upsilon(nS) \gamma$ decays.  The fraction of $\Upsilon(nS)$
originating from $\chi_b(mP)$ decays can generically be written as:

\begin{equation}
\resizebox{.9\hsize}{!}{
$
\frac{\sigma(pp \to \chi_b (mP) X) \times Br (\chi_b (mP) \to \Upsilon(nS) \gamma)}{\sigma(pp \to \Upsilon(nS) X)} =
\frac{N_{\chi_b (mP)\to \Upsilon(nS) \gamma}}{N_{\Upsilon(nS)}} \times \frac{\epsilon_{\Upsilon(nS)}}{\epsilon_{\chi_b (mP)\to \Upsilon(nS) \gamma}} =
\frac{N_{\chi_b (mP)\to \Upsilon(nS) \gamma}}{N_{\Upsilon(nS)}} \times \frac{1}{\epsilon^{reco}_{\gamma}}
$
}
\label{eqn:master}
\end{equation}


\noindent where
${N_{\Upsilon(nS)}}$ and ${N_{\chi_b(mP)\to \Upsilon(nS) \gamma}}$ are the
$\Upsilon(nS)$ and $\chi_b(mP)$ yields, $\epsilon_{\Upsilon(nS)}$ and
$\epsilon_{\chi_b(mP)\to \Upsilon(nS) \gamma}$ are their corresponding
selection efficiencies. The latter are the product of geometric acceptance,
trigger efficiency and reconstruction efficiency. Since the selection criteria
for the two samples differ only in the reconstruction of a photon, the
efficiency ratio can be replaced by 1/$\epsilon^{reco}_{\gamma}$, the
reconstruction efficiency for the photon from the $\chi_b$ decay.
$\epsilon^{reco}_{\gamma}$ is the ratio of $\chib$ and $\Upsilon$ true matched
events in the corresponding decay simulation.

The differential production ratios in $\Upsilon$ \pt bins 
can be computed by using a similar formula.  