\begin{figure}[H]
  \setlength{\unitlength}{1mm}
  \centering
  \begin{picture}(150,60)
    \put(0,0){
      \includegraphics*[width=75mm, height=60mm]{upsilon/f2011_6_40}
    }
    \put(0,15){\small \begin{sideways}Candidates/(6\mevcc)\end{sideways}}
    \put(25, 2){$m_{\mumu} \left[\gevcc\right]$}
    \put(45,45){\sqs = 7 \tev}

    \put(75,0){
      \includegraphics*[width=75mm, height=60mm]{upsilon/f2012_6_40}
    }
    \put(75,15){\small \begin{sideways}Candidates/(6\mevcc)\end{sideways}}
    \put(100,2){$m_{\mumu} \left[\gevcc\right]$}
    \put(120,45){\sqs = 8 \tev}

    % \graphpaper[5](0,0)(150, 60)
  \end{picture}
  \caption {\small
    Invariant mass distibution of the selected $\Upsilon \to \mumu$ candidates in
    the range $ 6 < p_T(\mumu)  < 40 \gevc$ and $2 < y^{\mumu} < 4.5 $. The three peaks
    correspond to the \Y1S, \Y2S and \Y2S signals (from left to right). The curves
    are the result of the fit described in the previous
    section~\ref{sec:upsilon:fit}.
  }
  \label{fig:upsilon:result:nominal}
\end{figure}