\begin{figure}[H]
  \setlength{\unitlength}{1mm}
  \centering
  \begin{picture}(150,60)
    %
    \put(0,0){
      \includegraphics*[width=75mm, height=60mm]{chib/ups1s/f2011_14_40}
    }

    \put(75,0){
      \includegraphics*[width=75mm, height=60mm]{chib/ups1s/f2012_14_40}
    }


    \put(3,23){\scriptsize \begin{sideways}Candidates/(20\mevcc)\end{sideways}}
    \put(10,13){$m_{\mumu \gamma} - m_{\mumu} + m_{\Y1S}^{PDG} \left[\gevcc\right]$}
    \put(40,50){\sqs=7\tev}


    \put(78,23){\scriptsize \begin{sideways}Candidates/(20\mevcc)\end{sideways}}
    \put(85,13){$m_{\mumu \gamma} - m_{\mumu} + m_{\Y1S}^{PDG} \left[\gevcc\right]$}
    \put(115,50){\sqs=8\tev}

    \put(25,40){$14 < p_T^{\Y1S} < 40 \gevc$}
    \put(100,40){$14 < p_T^{\Y1S} < 40 \gevc$}


    % \graphpaper[5](0,0)(150, 60)
  \end{picture}
  \caption {\small
    Distribution of the mass difference $\mumu \gamma - \mumu$ for selected
    \chib(1,2,3P) candidates (black points) together with the result of the fit
    (solid red curve), including background (dotted blue curve) and signals
    (dashed green and magenta curves) contributions. Green dashed curve
    corresponds to \chibone signal and magenta dashed curve to \chibtwo signal.
    The bottom insert shows the  pull distribution of the fit. The pull is
    defined as the difference  between the data and fit value divided by the
    data error. }
  \label{fig:chib:ups1s:fit:nominal}
\end{figure}
