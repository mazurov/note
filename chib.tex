\section{\texorpdfstring{$\chi_b$}{xb} signal extraction}
\label{sec:chib}

In this study, the photon in \chib decay is measured by the calorimeter system.
Another approach is to look at photons that convert to an electron-positron
(\epem) pair. Converted photons provide a better invariant mass resolution and would 
allow to separate mass peaks due to close resonances, since the 
\epm momentum 
resolution obtained from the tracking stations is better than the photon energy resolution obtained
by the calorimeter system. However, conversions should be required to happen before the magnet in
order to reconstruct the charged tracks. Furthermore, if the photon converts too
early, the \epm has more chance to radiate energy, which leads to worse
track reconstruction and worse energy resolution. Therefore, only photons
converting before the magnet and after the VELO should be used, which severely limits 
the size of the available sample and the decays which can be analyzed. In this study 
unconverted photons are used, in order to obtain a much larger data sample and 
analyze more decays in a wide range of $\Upsilon$ transverse momentum. 

%% ============================================================================
\subsection{Selection}
\label{sec:chib:selection}

To separate decays into different $\Upsilon$ channels, cuts on dimuon invariant mass are 
applied as shown in \Cref{fig:ups:mass_win} and \Cref{tab:chib:selection:window}:

\begin{figure}[H]
  \setlength{\unitlength}{1mm}
  \centering
  \begin{picture}(80,60)
    %
    \put(0,0){
      \includegraphics*[width=80mm, height=60mm]{upsilon/mass_win}
    }

     \put(0,15){\scriptsize \begin{sideways}Candidates/(12\mevcc)\end{sideways}}
     \put(30,0){$m_{\mumu} \left[\gevcc\right]$}
  \end{picture}
  \caption {\small
    Dimuon mass windows (blue bars) for \chib decay modes separation.
  }
  \label{fig:ups:mass_win}
\end{figure}

\begin{comment}
$3 \sigma_{\Y1S} < \mumu < 2.5 \sigma_{\Y1S}$
$3 \sigma_{\Y2S} < \mumu < \sigma_{\Y2S}$
$\sigma_{\Y3S} < \mumu < 3 \sigma_{\Y3S} \mevc$
\end{comment}

\begin{table}[H]
\caption{\small The cuts on dimuon mass window.}
\centering
\scalebox{0.9}{
\begin{tabular}{lr}\toprule
\multicolumn{1}{c}{Decay} & \multicolumn{1}{c}{Cut}\\
\midrule
$\chib(1,2,3P) \to \Upsilon(1S) \gamma$ & $9310 < \mumu < 9600 \mevc$\\
$\chib(2,3P) \to \Upsilon(2S)  \gamma$ & $9870 < \mumu < 10090  \mevc$\\
$\chib(3P) \to \Upsilon(3S)  \gamma $ & $10300 < \mumu < 10526  \mevc$\\
\bottomrule
\end{tabular}
}
\label{tab:chib:selection:window}
\end{table}

To avoid \Y2S and \Y3S mixing contamination, the mass ranges of the \Y2S and \Y3S are
asymmetric with respect to the nominal masses.

The selected $\Upsilon$ candidates  are combined with photon candidates to form
\chib candidates. Well reconstructed photons are selected by requiring their 
transverse momentum to be greater than 600 \mevc. To further suppress background, 
the cosine of the angle of the photon direction in the center-of-mass of the
$\mumu\gamma$ system with respect to the momentum of this system, is required to
be greater than zero. An additional loose cut on the photon confidence level is
required to be greater than 0.01. This confidence level is computed starting from the distributions 
of calorimetric variables which are sensitive to photons, by computing likelihoods under different particle 
hypotheses, and taking   
the ratio of the likelihood for a photon hypothesis divided by the sum of likelihoods for all hypotheses. 

The criteria for event selection with a reconstructed photon are summarized in
Table~\ref{tab:chib:selection:photons}:

\begin{table}[H]
\caption{\small The $\gamma$ selection criteria in  $\chib \to \Upsilon \gamma$ decays.}
\centering
\begin{tabular}{lr}\toprule
Transverse momentum of $\gamma$ & $p_T(\gamma) > 600 \mevc$ \\
Polar angle of $\gamma$ in the $\mumu\gamma$ rest frame & $\cos\theta_{\gamma} > 0$ \\
Confidence level of $\gamma$ & $cl(\gamma) > 0.01$ \\
\bottomrule
\end{tabular}
\label{tab:chib:selection:photons}
\end{table}


The $\Upsilon$ selection cuts
~(Table~\ref{tab:upsilon:selection:study:summary}), the cuts  on $\gamma$
~(Table~\ref{tab:chib:selection:photons}) and dimuon mass  
~(Table~\ref{tab:chib:selection:window}) are used to obtain \chib yields
using a fit model which is described in the next section.
    
%% ============================================================================
\subsection{Fit model}
\label{sec:chib:fit}

This section describes the common properties of the fit model that is used for
obtaining yields in each of the \chib decays. The results of the fits are
given in the following sections.

The \chib signal yields are obtained by fitting event candidates
in the distribution of invariant mass difference $m(\mumu\gamma) - m(\mumu)$.
In this case the bias and resolution effects of the $\Upsilon$ mass from reconstruction
are canceled at first order. For clearness, the PDG mass of the corresponding $\Upsilon$
particle is added to the mass difference value in each plot.

The $\chi_b(jP)$ (j=1,2,3) signals are the sum of three contributions, due to $\chi_{b0}(jP)$,
$\chi_{b1}(jP)$, $\chi_{b2}(jP)$. The $\chi_{b0}$ meson is difficult to detect
because it has a low radiative branching ratio in comparison with the other two
mesons. So the $\chi_{b0}$ states were excluded from this study and the fit
model.

To determine the $\chi_b$ signal yields, an unbinned maximum likelihood fit to
$m(\mumu\gamma) - m(\mumu)$ has been performed. The signal has been modeled with
a sum of right-sided Crystal Ball (CB)  functions. The background is parameterized with a
product of an exponential function and a linear combination of basic Bernstein
polynomials~\cite{Phillips:2003} with non-negative coefficients $c_{i}^2$:

\begin{equation}
\label{eq:bernstein}
{\mathscr B}_{n}(x) = e^{-\tau x} \times \sum_{i=0}^{n} c_{i}^2 {\mathscr B}_{n}^{i}(x)
\end{equation}
Such combination results in a smooth and non-negative function that can be used
as a PDF.

The right-sided CrystalBall function can be written in the following form:

\begin{equation}
CB(x) = 
\begin{cases}
\frac{1}{\sqrt{2\pi\sigma}}{(\frac{n_R}{|\alpha|})}^{n}\exp(-\frac{|\alpha|^2}{2}){(\frac{n}{|\alpha|}-|\alpha|+\frac{x-\mu}{\sigma})}^{-n} & \text{, if $\frac{x-\mu}{\sigma} > \alpha$}\\
\frac{1}{\sqrt{2\pi\sigma}}\exp(-\frac{{(x-\mu)}^2}{2\sigma^2}) & \text{, otherwise}
\end{cases}
\label{eq:cb}
\end{equation}




As already mentioned, the CB is similar to a gaussian distribution, but has an asymmetric tail. This
function has five parameters: $\mu$, $\sigma$, $\alpha$ and $n$, where
parameters $\mu$ and $\sigma$ have the same meaning as for gaussian. Parameters
$\alpha$ and $n$ describe the tail behavior: $\alpha$ controls the tail start
and $n$ corresponds to the decreasing power of the tail. 

The number of CrystalBall functions and the order of the polynomial depend on
the decay under study and are described in the section corresponding to that
specific decay.


The $\alpha$ and $n$
parameters of CB are fixed to the values obtained from simulation~(\Cref{sec:mc})
and are shown in~\Cref{tab:chib:fit:tail}:


\begin{table}[H]
\caption{\small   The $\alpha$ and $n$ parameters of CB functions.}
\centering
\begin{tabular}{lrr}
\toprule
Signal & $\alpha$ & $n$ \\
\midrule
$\chi_{b1,2}(1,2P)$ & 1.1 & 5 \\
$\chi_{b1,2}(3P)$ & 1.25 & 5 \\
\bottomrule
\end{tabular}
\label{tab:chib:fit:tail}
\end{table}

Due to the small mass difference between $\chi_{b2}(jP)$ and $\chi_{b1}(jP)$
(j=1,2,3) states and the insufficient detector resolution, it is not possible
to fit the $\chi_{b1}$ and $\chi_{b2}$ states by two independent CB functions.
Thus, the mean, width and yield values of  $\chi_{b1}$ and $\chi_{b2}$ signals
are linked together by the following constraints:

\begin{equation}
  \begin{aligned}
\mu_{\chi_{b2}(jP)} = \mu_{\chi_{b1}(jP)} + \Delta m_{\chi_{b1,2}(jP)}^{PDG} \text{, j = (1,2)}\\
\mu_{\chi_{b2}(3P)} = \mu_{\chi_{b1}(3P)} + \Delta m_{\chi_{b1,2}(3P)}^{theory} \\
\sigma_{\chi_{b2}} = k \sigma_{\chi_{b1}}\\
N_{\chi_{b2}} = \frac{(1-\lambda)}{\lambda} N_{\chi_{b1}}
  \end{aligned}
\end{equation}

\noindent where $\Delta m_{\chi_{b1,2}(jP)}^{PDG}$ is the corresponding PDG
mass difference which is fixed in the fit; $\Delta
m_{\chi_{b1,2}(3P)}^{theory}$ is fixed to the theoretical predicted mass
difference in 12\mevcc~\cite{Motyka:1997di}. The $\lambda$ parameter depends on
the $p_T(\Upsilon)$ range and is fixed to the value that is based on the
theoretical prediction discussed in Section~\ref{sec:ratio}. The parameter $k$
is the ratio between the resolution of $\chi_{b1}$ and $\chi_{b2}$ signals.
This parameter is obtained from simulation, and is fixed to $1.05$ for
$\chi_{b1,2}(1P)$ signals and fixed to 1 for $\chi_{b1,2}(2,3P)$ signals.

The width of each CB function ($\sigma$) is fixed to the value obtained from
simulation~(Section~\ref{sec:mc}) in each transverse momentum bin, in order to
improve fit convergence and reduce uncertainties.

As already mentioned in the introduction (\Cref{sec:introduction}), the \chibThreeP  was
recently observed, but the mass of this meson was not precisely measured. 
\Cref{sec:chib:ups3s:fit} presents a determination of the \chibThreeP mass, which 
was consequently fixed to the measured value of 10.510\gevcc in these studies. 

%% ============================================================================
\subsection{\texorpdfstring{\chib}{xb} yields in
\texorpdfstring{$\chib \to \Y1S \gamma$}{xb -> Y\(1S\) gamma } decays}
\label{sec:chib:ups1s:fit}

If $\chi_{b0}$ decays are neglected, the \Y1S can be produced in radiative decays
of six \chib particles: $\chi_{bi}(jP) \to \Y1S \gamma$ (i=1,2; j=1,2,3). 
Each \chib signal is parameterized with a CB function.
The mass of \chiboneOneP ($\mu_{\chiboneOneP}$) is taken as free parameter in the fit, and  other
parameters are constrained by:

\begin{equation}
  \begin{aligned}
\mu_{\chiboneTwoP} = \mu_{\chiboneOneP} + \Delta m_{\chi_{b1}(2P)}^{PDG} \\
\mu_{\chiboneThreeP} = \mu_{\chiboneOneP} + \Delta m_{\chi_{b1}(3P)}, \\
  \end{aligned}
\end{equation}
\noindent where $\Delta m_{\chi_{b1}(2P)}^{PDG}$ is the difference between the PDG masses of
\chiboneTwoP and \chiboneOneP. The $\Delta  m_{\chi_{b1}(3P)}$ parameter is the difference
between the masses of \chiboneThreeP and \chiboneOneP, where the mass of
\chiboneThreeP was taken from the measurement performed in this 
thesis~(Section~\ref{sec:chib:ups3s:fit}). The parameters  $\Delta
m_{\chi_{b1}(2P)}^{PDG}$ and $\Delta  m_{\chi_{b1}(3P)}$ are fixed in the fit.
The  \chiboneOneP peak width depends on $p_T^{\Y1S}$ interval and in each bin is fixed to
the value obtained from simulation (Section~\ref{sec:mc}) scaled by 1.16. 
The scaling parameter is obtained by comparing data and simulation fits
in high $p_T^{\Y1S}$ ranges with $\lambda=0.5$.

The order of the background polynomial in~\Cref{eq:bernstein} depends on the
$p_{T}^{\OneS}$ interval and is given in~\Cref{tab:chib:ups1s:fit:order}.

\begin{table}[H]
  \caption{
    \small The order of background polynomial for the $\chib \to \Y1S$ fit model
    }
    \centering
   \begin{tabular}{cc}\toprule
    $p_{T}^{\OneS}$ interval, \gevc & Polynomial order ($n$)\\
    \midrule
    6 --- 8 & 5 \\
    8 --- 12 & 4 \\
    12 --- 30 & 2 \\
    \bottomrule
  \end{tabular}
\label{tab:chib:ups1s:fit:order}
\end{table}


The fit was performed in the mass interval from  9.77 \gevcc to 10.89 \gevcc.
Figure~\ref{fig:chib:ups1s:fit:nominal} shows the mass distribution along with
the pull distribution in the transverse momentum range $14 < p_T^{\OneS} < 40
\gevc$. In this range the fit has the lowest relative error of signal yields. 
Table~\ref{tab:chib:ups1s:nominal} details the corresponding fit parameters.

The pull is the residual divided by the error:
\begin{equation}
\label{eq:pull}
Pull = \frac{N_{data} - N_{model}}{\sqrt{N_{data}}},
\end{equation}
\noindent where $N_{model}$ is the expected number of events in a bin from
the fit function and $\sqrt{N_{data}}$ is the statistical uncertainty on the
number of events in a bin. Pull values  for good fits are normally
distributed around zero, with a standard deviation of 1. 


\begin{figure}[H]
  \setlength{\unitlength}{1mm}
  \centering
  \begin{picture}(150,60)
    %
    \put(0,0){
      \includegraphics*[width=75mm, height=60mm]{chib/ups1s/f2011_14_40}
    }

    \put(75,0){
      \includegraphics*[width=75mm, height=60mm]{chib/ups1s/f2012_14_40}
    }


    \put(3,23){\scriptsize \begin{sideways}Candidates/(20\mevcc)\end{sideways}}
    \put(10,13){$m_{\mumu \gamma} - m_{\mumu} + m_{\Y1S}^{PDG} \left[\gevcc\right]$}
    \put(40,50){\sqs=7\tev}


    \put(78,23){\scriptsize \begin{sideways}Candidates/(20\mevcc)\end{sideways}}
    \put(85,13){$m_{\mumu \gamma} - m_{\mumu} + m_{\Y1S}^{PDG} \left[\gevcc\right]$}
    \put(115,50){\sqs=8\tev}

    \put(25,40){$14 < p_T^{\Y1S} < 40 \gevc$}
    \put(100,40){$14 < p_T^{\Y1S} < 40 \gevc$}


    % \graphpaper[5](0,0)(150, 60)
  \end{picture}
  \caption {\small
    Distribution of the mass difference $\mumu \gamma - \mumu$ for selected
    \chib(1,2,3P) candidates (black points) together with the result of the fit
    (solid red curve), including background (dotted blue curve) and signals
    (dashed green and magenta curves) contributions. Green dashed curve
    corresponds to \chibone signal and magenta dashed curve to \chibtwo signal.
    The bottom insert shows the  pull distribution of the fit. The pull is
    defined as the difference  between the data and fit value divided by the
    data error. }
  \label{fig:chib:ups1s:fit:nominal}
\end{figure}

\begin{table}[H]
\caption{\small Data fit parameters for $\chi_{b1,2}(1,2,3P) \to \Y1S \gamma$ decays}
\centering
\resizebox{.75\textwidth}{!}{
\begin{tabular}{lrr}\toprule
 & \multicolumn{2}{c}{$\Upsilon(1S)$ transverse momentum intervals, \gevc}\\
 & \multicolumn{2}{c}{14 -- 40}\\
\cmidrule(r){2-3}
 & \multicolumn{1}{c}{\sqs = 7\tev} & \multicolumn{1}{c}{\sqs = 8\tev}\\
\midrule
$N_{\chibOneP}$ & 2100 $\pm$ 80 & 5090 $\pm$ 130\\
$N_{\chibTwoP}$ & 440 $\pm$ 50 & 1000 $\pm$ 80\\
$N_{\chibThreeP}$ & 150 $\pm$ 40 & 220 $\pm$ 60\\

\rule{0pt}{4ex}Background & 8820 $\pm$ 130 & 23,910 $\pm$ 210\\

\rule{0pt}{4ex}$\sigma_{\chiboneOneP}, \mevcc$ & 22.0 & 22.5\\
$\sigma_{\chi_{b1}(2P)} / \sigma_{\chi_{b1}(1P)}$ & 1.5 & 1.5\\
$\sigma_{\chi_{b1}(3P)} / \sigma_{\chi_{b1}(1P)}$ & 1.86 & 1.86\\

\rule{0pt}{4ex}$\tau$ & -2.5 $\pm$ 0.5 & -3.19 $\pm$ 0.31\\
$c_0$ & -0.10 $\pm$ 0.12 & 0.06 $\pm$ 0.06\\
$c_1$ & 1.35 $\pm$ 0.04 & 0.27 $\pm$ 0.04\\

\rule{0pt}{4ex}$\chi^2 / n.d.f$ & 1.13 & 1.63\\
\bottomrule
\end{tabular}
} % scalebox
\label{tab:chib:ups1s:fits}
\end{table}


\Cref{tab:chib:ups1s:nominal} shows that the measured \chiboneOneP mass
nicely agrees with the PDG value $9892.78 \pm 0.26 \pm 0.31 \mevcc$. In
the following, this mass was fixed to 9.887 \gevcc which is the value 
measured on the combined 2011 and 2012 datasets in the range $6<p_T^{\Y1S}<40\gevc$.

In this study is observed that the background to signal ratio\footnote{ The
ratio B/S is estimated  from the equation $\sigma(S) = 1/\sqrt{S} \times
\sqrt{1 + B/S}$. } for \chibThreeP signal  is about 1.6 times
larger for 2012 data then for 2011 data. For \chibOneP and \chibTwoP signals
this difference is compatible with statistical uncertainty. Multiplicity cuts
like cuts on SPD, cuts on number of long tracks were examined for resolve
this issue. The desired difference is observed only with one long track.



\Cref{fig:chib:ups1s:yields} illustrates the number of signal events as
a function of $\Y1S$ transverse momentum. The yields
normalized by bin size and luminosity are shown in
\Cref{fig:chib:ups1s:yields_scaled}. The \chibOneP and \chibThreeP yields
are smoothly decreasing functions of $p_T^{\Y1S}$, as expected. Differences between 7 and 8\tev
data, due to different production cross sections, can be seen for the
$\chi_b(1P)$ state, while they are washed out by statistical fluctuations for
the other states. The unexpected difference for $\chibTwoP$ yields in the lowest
$p_T$ bin is due to insufficient fit in this region (see~\Cref{fig:chib:ups1s:fits2011,fig:chib:ups1s:fits2012}).
\Cref{tab:chib:ups1s:fits} in Appendix summarizes the
obtained results.

\begin{figure}[H]
  \setlength{\unitlength}{1mm}
  \centering
  \begin{picture}(150,120)
    \put(0,0){
      \includegraphics*[width=75mm, height=60mm]{chib/ups1s/N3P}
    }
    \put(0,60){
      \includegraphics*[width=75mm, height=60mm]{chib/ups1s/N1P}
    }
    \put(75,60){
      \includegraphics*[width=75mm, height=60mm]{chib/ups1s/N2P}
    }

    \put(2,25){\begin{sideways}Events\end{sideways}}
    \put(35,2){$p_T^{\Y1S} \left[\gevc\right]$}
    \put(55,50){$\chibThreeP$}

    \put(2,85){\begin{sideways}Events\end{sideways}}
    \put(35,62){$p_T^{\Y1S} \left[\gevc\right]$}
    \put(55,110){$\chibOneP$}

    \put(77,85){\begin{sideways}Events\end{sideways}}
    \put(110,62){$p_T^{\Y1S} \left[\gevc\right]$}
    \put(130,110){$\chibTwoP$}


    \put(50,45){\textcolor{blue}{\sqs=7\tev}}
    \put(50,40){\textcolor{red}{\sqs=8\tev}}
    \put(44,45){
      \includegraphics*[width=4mm, height=2mm]{blue}
    }
    \put(44,40){
      \includegraphics*[width=4mm, height=2mm]{red}
    }

    \put(50,105){\textcolor{blue}{\sqs=7\tev}}
    \put(50,100){\textcolor{red}{\sqs=8\tev}}
    \put(44,105){
      \includegraphics*[width=4mm, height=2mm]{blue}
    }
    \put(44,100){
      \includegraphics*[width=4mm, height=2mm]{red}
    }

    \put(125,105){\textcolor{blue}{\sqs=7\tev}}
    \put(125,100){\textcolor{red}{\sqs=8\tev}}
    \put(119,105){
      \includegraphics*[width=4mm, height=2mm]{blue}
    }
    \put(119,100){
      \includegraphics*[width=4mm, height=2mm]{red}
    }


  % \graphpaper[5](0,0)(75, 60)
  \end{picture}
  \caption {\small
    Distribution of \chib yields in $\chib \to \Y1S \gamma$ decay
    in specified $p_T^{\Y1S}$ ranges.
  }
  \label{fig:chib:ups1s:yields}
\end{figure}


\begin{figure}[H]
  \setlength{\unitlength}{1mm}
  \centering
  \begin{picture}(150,120)
    \put(0,0){
      \includegraphics*[width=75mm, height=60mm]{chib/ups1s/N3P_scaledbylum}
    }
    \put(0,60){
      \includegraphics*[width=75mm, height=60mm]{chib/ups1s/N1P_scaledbylum}
    }
    \put(75,60){
      \includegraphics*[width=75mm, height=60mm]{chib/ups1s/N2P_scaledbylum}
    }

    \put(2,25){\begin{sideways}Events\end{sideways}}
    \put(35,2){$p_T^{\Y1S} \left[\gevc\right]$}
    \put(55,50){$\chibThreeP$}

    \put(2,85){\begin{sideways}Events\end{sideways}}
    \put(35,62){$p_T^{\Y1S} \left[\gevc\right]$}
    \put(55,110){$\chibOneP$}

    \put(77,85){\begin{sideways}Events\end{sideways}}
    \put(110,62){$p_T^{\Y1S} \left[\gevc\right]$}
    \put(130,110){$\chibTwoP$}


    \put(50,45){\textcolor{blue}{\sqs=7\tev}}
    \put(50,40){\textcolor{red}{\sqs=8\tev}}
    \put(44,45){
      \includegraphics*[width=4mm, height=2mm]{blue}
    }
    \put(44,40){
      \includegraphics*[width=4mm, height=2mm]{red}
    }

    \put(50,105){\textcolor{blue}{\sqs=7\tev}}
    \put(50,100){\textcolor{red}{\sqs=8\tev}}
    \put(44,105){
      \includegraphics*[width=4mm, height=2mm]{blue}
    }
    \put(44,100){
      \includegraphics*[width=4mm, height=2mm]{red}
    }

    \put(125,105){\textcolor{blue}{\sqs=7\tev}}
    \put(125,100){\textcolor{red}{\sqs=8\tev}}
    \put(119,105){
      \includegraphics*[width=4mm, height=2mm]{blue}
    }
    \put(119,100){
      \includegraphics*[width=4mm, height=2mm]{red}
    }


  % \graphpaper[5](0,0)(75, 60)
  \end{picture}
  \caption {\small
    Distribution of \chib yields in $\chib \to \Y1S \gamma$ decay
    in specified $p_T^{\Y1S}$ ranges.
    The distribution normalized by bin size and luminosity value.
  }
  \label{fig:chib:ups1s:yields_scaled}
\end{figure}

Even though a correction on the momentum scale was applied on data, a smooth variation of the
\chiboneOneP mass is observed as a function of transverse momentum (see
Figure~\ref{fig:chib:ups1s:mean}). This effect can be explained by the unknown
ratio between the number of \chiboneOneP and \chiboneTwoP candidates.
Figure~\ref{fig:chib-1s:m1p} shows how the measured mass depends on this ratio
($\lambda$ parameter). A systematic uncertainty is assigned to this effect.

\begin{figure}[H]
  \setlength{\unitlength}{1mm}
  \centering
  \begin{picture}(75,60)
    \put(0,0){
      \includegraphics*[width=75mm, height=60mm]{chib/ups1s/mean_b1_1p}
    }
  
    \put(0,12){\begin{sideways}\chiboneOneP mass $\left[\gevcc\right]$\end{sideways}}
    \put(35,2){$p_T^{\Y1S} \left[\gevc\right]$}
  

    \put(50,30){\textcolor{blue}{\sqs=7\tev}}
    \put(50,25){\textcolor{red}{\sqs=8\tev}}
    \put(44,30){
      \includegraphics*[width=4mm, height=2mm]{blue}
    }
    \put(44,25){
      \includegraphics*[width=4mm, height=2mm]{red}
    }

  % \graphpaper[5](0,0)(75, 60)
  \end{picture}
  \caption {\small
    The \chiboneOneP mass reconstructed at $\chib \to \Y1S \gamma$ decays.
  }
  \label{fig:chib:ups1s:mean}
\end{figure}
\begin{figure}[H]
  \setlength{\unitlength}{1mm}
  \centering
  \begin{picture}(80,60)
    %
    \put(0,0){
      \includegraphics*[width=80mm, height=60mm]{chib/ups1s/m1p_lambda}
    }

     \put(0,15){\scriptsize \begin{sideways}Mass of \chiboneOneP (\gevcc)\end{sideways}}
     \put(75,0){$\lambda$}

    \put(15,50){\includegraphics*[width=4mm, height=2mm]{blue}}
    \put(15,46){\includegraphics*[width=4mm, height=2mm]{red}}

    \put(20,50){\scriptsize \textcolor{blue}{\sqs=7\tev}}
    \put(20,46){\scriptsize \textcolor{red}{\sqs=8\tev}}

  \end{picture}
  \caption {\small
     Mass of \chiboneOneP as function of ratio between number of \chiboneOneP
     and \chiboneTwoP candidates ($\lambda$). Measurements performes in
     $p_T^{\Y1S}$ interval from 18 to 22 \gevc. }
  \label{fig:chib-1s:m1p}
\end{figure}


\subsubsection{Model based on the study with converted photons}
This section shows results of the fits, where the ratio between
$\chiboneOneP$ and $\chiboneOneP$ yields is constrained with the ratio obtained
in the study with converted photons~\cite{Lespinasse:1664279} (very preliminary
results). \Cref{tab:chib:converted_ratio} shows the ratio measured in that study as function
of $p_T^{\Y1S}$.

\begin{table}[H]
\caption{\small Ratio of $\chiboneOneP$ and $\chibtwoOneP$ as function
of $p_T^{\Y1S}$}
\centering
% \resizebox{.75\textwidth}{!}{
\begin{tabular}{lccc}\toprule
 & \multicolumn{3}{c}{$\Upsilon(1S)$ transverse momentum intervals, \gevc}\\
 & 5 -- 10 & 10 -- 15 & 15 -- 20\\
\midrule
$ r = N_{\chibtwoTwoP} / N_{\chiboneTwoP}$ & 0.66 $\pm$ 0.15 & 0.51 $\pm$ 0.15 & 0.40 $\pm$ 0.12\\
$ \lambda = 1 / (1 + r)$ & 0.60 $\pm$ 0.05 & 0.66 $\pm$ 0.06 & 0.71 $\pm$ 0.06\\
\bottomrule
\end{tabular}
% } % scalebox
\label{tab:chib:converted_ratio}
\end{table}

\Cref{fig:chib:ups1s:conv} and \Cref{tab:chib:ups1s:conv_fits} present fits  and 
fits parameters where the gaussian external constraint is applied on  the value of $\lambda$ parameter.
The mean and sigma parameters of the gaussian are taken from \Cref{tab:chib:converted_ratio}.

\begin{figure}[H]
  \setlength{\unitlength}{1mm}
  \centering
  \resizebox{\textwidth}{!}{
    \begin{picture}(150,60)
    \put(0,0){
      \includegraphics*[width=75mm, height=60mm]{chib/ups1s/conv/f2011_14_18}
    }
    \put(75,0){
      \includegraphics*[width=75mm, height=60mm]{chib/ups1s/conv/f2012_14_18}
    }

     \put(25,50){$14 < p_T^{\Y1S} < 18 \gevc$}
     \put(100,50){$14 < p_T^{\Y1S} < 18 \gevc$}

     \put(25,40){\sqs=7\tev}
     \put(100,40){\sqs=8\tev}     

     \put(3,22){\scriptsize \begin{sideways}Candidates/(40\mevcc)\end{sideways}}
     \put(78,22){\scriptsize \begin{sideways}Candidates/(40\mevcc)\end{sideways}}

     \put(10,13){$m_{\mumu \gamma} - m_{\mumu} + m_{\Y1S}^{PDG} \left[\gevcc\right]$}
     \put(85,13){$m_{\mumu \gamma} - m_{\mumu} + m_{\Y1S}^{PDG} \left[\gevcc\right]$}

     \put(8,2){\scriptsize \begin{sideways}Pull\end{sideways}}
     \put(83,2){\scriptsize \begin{sideways}Pull\end{sideways}}

    \end{picture}
   }




  \caption {\small 
    Distribution of the mass difference $m(\mumu \gamma) - m(\mumu)$ for selected
    \chib candidates (black points) together with the result of the fit 
    (solid red curve), including the background (dotted blue curve) and the signal 
    (dashed green and magenta curves) contributions. Green dashed curve corresponds
    to \chibone signal and magenta dashed curve to \chibtwo signal. Plots
    show the distribution in specified intervals of \Y1S transverse momentum.
   }
\label{fig:chib:ups1s:conv}
\end{figure}
\begin{table}[H]
\centering
\caption{\small Data fit parameters for $\chi_{b1,2}(1,2,3P) \to \Y1S \gamma$ decays}
\subtable[$8 < p_T^{\Y1S} < 18 \gevc$] {
\scalebox{0.6}{

\begin{tabular}{lrrrrrr}\toprule
 & \multicolumn{6}{c}{$\Upsilon(1S)$ transverse momentum intervals, \gevc}\\
 & \multicolumn{2}{c}{8 -- 10} & \multicolumn{2}{c}{10 -- 14} & \multicolumn{2}{c}{14 -- 18}\\
\cmidrule(r){2-3}\cmidrule(r){4-5}\cmidrule(r){6-7}
 & \multicolumn{1}{c}{\sqs = 7\tev} & \multicolumn{1}{c}{\sqs = 8\tev} & \multicolumn{1}{c}{\sqs = 7\tev} & \multicolumn{1}{c}{\sqs = 8\tev} & \multicolumn{1}{c}{\sqs = 7\tev} & \multicolumn{1}{c}{\sqs = 8\tev}\\
\midrule
$N_{\chibOneP}$ & 2450 $\pm$ 140 & 5700 $\pm$ 220 & 3000 $\pm$ 110 & 7250 $\pm$ 180 & 1160 $\pm$ 60 & 2810 $\pm$ 100\\
$N_{\chibTwoP}$ & 900 $\pm$ 100 & 1790 $\pm$ 170 & 640 $\pm$ 80 & 1040 $\pm$ 130 & 250 $\pm$ 40 & 580 $\pm$ 60\\
$N_{\chibThreeP}$ & --- & --- & 270 $\pm$ 80 & 470 $\pm$ 130 & 85 $\pm$ 30 & 130 $\pm$ 50\\

\rule{0pt}{4ex}Background & 54,000 $\pm$ 280 & 143,000 $\pm$ 500 & 35,140 $\pm$ 250 & 93,300 $\pm$ 400 & 6970 $\pm$ 110 & 18,810 $\pm$ 180\\

\rule{0pt}{4ex}$\mu_{\chiboneOneP}, \mevcc$ & 9884.9 $\pm$ 1.6 & 9885.7 $\pm$ 1.4 & 9886.2 $\pm$ 1.4 & 9891.8 $\pm$ 1.1 & 9892.1 $\pm$ 1.8 & 9891.0 $\pm$ 1.4\\
$\sigma_{\chiboneOneP}, \mevc$ & 22.0 & 22.5 & 21.0 & 21.5 & 20.0 & 20.5\\

\rule{0pt}{4ex}$\chi^2 / n.d.f$ & 1.12 & 0.95 & 1.54 & 0.99 & 0.95 & 0.88\\

\rule{0pt}{4ex}$\lambda_{\chiboneOneP}$ & 0.58 $\pm$ 0.05 & 0.56 $\pm$ 0.05 & 0.58 $\pm$ 0.05 & 0.71 $\pm$ 0.04 & 0.70 $\pm$ 0.06 & 0.66 $\pm$ 0.05\\
\bottomrule
\end{tabular}
} % scalebox/resizebox

} % subtable
\subtable[$18 < p_T^{\Y1S} < 22 \gevc$] {
\scalebox{0.6}{

\begin{tabular}{lrr}\toprule
 & \multicolumn{2}{c}{$\Upsilon(1S)$ transverse momentum intervals, \gevc}\\
 & \multicolumn{2}{c}{18 -- 22}\\
\cmidrule(r){2-3}
 & \multicolumn{1}{c}{\sqs = 7\tev} & \multicolumn{1}{c}{\sqs = 8\tev}\\
\midrule
$N_{\chibOneP}$ & 445 $\pm$ 32 & 1120 $\pm$ 50\\
$N_{\chibTwoP}$ & 84 $\pm$ 16 & 158 $\pm$ 25\\
$N_{\chibThreeP}$ & 28 $\pm$ 10 & 33 $\pm$ 16\\

\rule{0pt}{4ex}Background & 1530 $\pm$ 50 & 4070 $\pm$ 80\\

\rule{0pt}{4ex}$\mu_{\chiboneOneP}, \mevcc$ & 9892.9 $\pm$ 2.1 & 9894.1 $\pm$ 1.6\\
$\sigma_{\chiboneOneP}, \mevc$ & 18.0 & 18.5\\

\rule{0pt}{4ex}$\chi^2 / n.d.f$ & 0.71 & 1.13\\

\rule{0pt}{4ex}$\lambda_{\chiboneOneP}$ & 0.69 $\pm$ 0.06 & 0.67 $\pm$ 0.06\\
\bottomrule
\end{tabular}
} % scalebox/resizebox

} % subtable
\label{tab:chib:ups1s:conv_fits}
\end{table}


\Cref{fig:chib:ups1s:conv_mean} shows that in $p_T^{\Y1S} > 14 \gevc$ range the
mass of $\chiboneOneP$ is consistent with the PDG value. In the same ranges the
width of $\chiboneOneP$ signal is in agreement with the simulation results
(\Cref{tab:mc:chib3p_ups3s:fits}). In the lower transverse momentum ranges the
width is fixed to the width obtained in simulation, since the fit is not stable in that
regions. In these ranges the observed $\chiboneOneP$ mass is around 6 \mevc lower than the
PDG value.

\begin{figure}[H]
  \setlength{\unitlength}{1mm}
  \centering
  \begin{picture}(75,60)
    \put(0,0){
      \includegraphics*[width=75mm, height=60mm]{chib/ups1s/conv/mean_b1_1p}
    }
  
    \put(0,12){\begin{sideways}\chiboneOneP mass $\left[\gevcc\right]$\end{sideways}}
    \put(35,2){$p_T^{\Y1S} \left[\gevc\right]$}
  

    \put(50,30){\textcolor{blue}{\sqs=7\tev}}
    \put(50,25){\textcolor{red}{\sqs=8\tev}}
    \put(44,30){
      \includegraphics*[width=4mm, height=2mm]{blue}
    }
    \put(44,25){
      \includegraphics*[width=4mm, height=2mm]{red}
    }

  % \graphpaper[5](0,0)(75, 60)
  \end{picture}
  \caption {\small
    Distribution of the \chiboneOneP mass in $\chibOneP \to \Y1S \gamma$ decay
    in specified $p_T^{\Y1S}$ ranges. The dashed line indicates the PDG value.
  }
  \label{fig:chib:ups1s:conv_mean}
\end{figure}

The systematic uncertainty of $\chibOneP$ yield is around 10\% in lower
$p_T^{\Y1S}$ bins. The source of this large uncertainty is due to the signal
width which in the nominal fit is fixed to the simulation width scaled by 1.16. 
The results in \Cref{tab:chib:ups1s:conv_fits} show that the scaling is not required. 
The second systematic uncertainty source is the different $\lambda$ value.

%% ===========================================================================
\subsection{\texorpdfstring{\chib}{xb} yields in
    \texorpdfstring{$\chib \to \Y2S \gamma$}{xb --> Y(2S) gamma} decays}
\label{sec:chib:ups2s:fit}

The fit was performed in the mass interval from 10.16 \gevcc to 11.04 \gevcc.
The  \chiboneTwoP peak width depends on $p_T^{\Y2S}$ interval and in each bin is fixed to
the value obtained from simulation (Section~\ref{sec:mc}) without any scaling.
The \chiboneThreeP peak width is fixed to \chiboneTwoP peak width scaled by
1.65, as observed on simulation. 


A third order polynomial for the background (described in \Cref{eq:bernstein}) is used 
for all intervals of \Y2S transverse momentum.


\Cref{fig:chib:ups2s:nominal} shows the mass distribution in the transverse
momentum range $18 < p_T^{\TwoS} < 40 \gevc$. \Cref{tab:chib:ups2s:nominal}
details the corresponding fit parameters.

\begin{figure}[H]
  \setlength{\unitlength}{1mm}
  \centering
  \begin{picture}(150,60)
    %
    \put(0,0){
      \includegraphics*[width=75mm, height=60mm]{chib/ups2s/f2011_18_40}
    }

    \put(75,0){
      \includegraphics*[width=75mm, height=60mm]{chib/ups2s/f2012_18_40}
    }

    \put(25,40){$18 < p_T^{\Y2S} < 40 \gevc$}
    \put(100,40){$18 < p_T^{\Y2S} < 40 \gevc$}

    \put(3,23){\scriptsize \begin{sideways}Candidates/(20\mevcc)\end{sideways}}
    \put(9,13){$m_{\mumu \gamma} - m_{\mumu} + m_{\Y2S}^{PDG} \left[\gevcc\right]$}
    \put(40,50){\sqs=7\tev}

    \put(78,23){\scriptsize \begin{sideways}Candidates/(20\mevcc)\end{sideways}}
    \put(84,13){$m_{\mumu \gamma} - m_{\mumu} + m_{\Y2S}^{PDG} \left[\gevcc\right]$}
    \put(115,50){\sqs=8\tev}



    % \graphpaper[5](0,0)(150, 60)
  \end{picture}
  \caption {\small
    Distribution of the mass difference $m(\mumu\gamma) - m(\mumu)$ for selected
    \chib(2,3P) candidates (black points) together with the result of the fit
    (solid red curve), including the background (dotted blue curve) and the
    signal (dashed green and magenta curves) contributions. Green dashed curve
    corresponds to \chibone signal and magenta dashed curve to \chibtwo signal.
    The bottom insert shows the  pull distribution of the fit. The pull is
    defined as the difference  between the data and fit value divided by the
    data error. }
  \label{fig:chib:ups2s:nominal}
\end{figure}
\begin{table}[H]
\caption{\small Data fit parameters for $\chi_{b1,2}(2,3P) \to \Y2S \gamma$ decays}
\centering
\resizebox{.75\textwidth}{!}{
\begin{tabular}{lrr}\toprule
 & \multicolumn{2}{c}{$\Upsilon(2S)$ transverse momentum intervals, \gevc}\\
 & \multicolumn{2}{c}{18 -- 40}\\
\cmidrule(r){2-3}
 & \multicolumn{1}{c}{\sqs = 7\tev} & \multicolumn{1}{c}{\sqs = 8\tev}\\
\midrule
$N_{\chibTwoP}$ & 236 $\pm$ 29 & 650 $\pm$ 50\\
$N_{\chibThreeP}$ & 50 $\pm$ 17 & 78 $\pm$ 26\\

\rule{0pt}{4ex}Background & 1830 $\pm$ 50 & 4600 $\pm$ 80\\

\rule{0pt}{4ex}$\mu_{\chiboneTwoP}, \mevcc$ & 10,250.0 & 10,250.0\\
$\sigma_{\chiboneTwoP}, \mevcc$ & 13.0 & 13.3\\
$\sigma_{\chi_{b1}(3P)} / \sigma_{\chi_{b1}(2P)}$ & 1.65 & 1.65\\

\rule{0pt}{4ex}$\tau$ & -7.5 $\pm$ 0.8 & -7.7 $\pm$ 0.5\\
$c_0$ & 0.432 $\pm$ 0.026 & 0.435 $\pm$ 0.016\\
$c_1$ & -2.08 $\pm$ 0.09 & -2.12 $\pm$ 0.05\\
$c_2$ & 0.79 $\pm$ 0.34 & 0.78 $\pm$ 0.16\\

\rule{0pt}{4ex}$\chi^2 / n.d.f$ & 0.95 & 1.32\\
\bottomrule
\end{tabular}
} % scalebox
\label{tab:chib:ups2s:nominal}
\end{table}

\Cref{tab:chib:ups2s:nominal} shows that the measured \chiboneTwoP mass is
about 5 \mevcc less than the PDG value $10255.46  \pm 0.22 \pm 0.50 \mevcc$.
The same difference is also observed in the smaller $p_T^{\Y2S}$ ranges
(Figure~\ref{fig:chib:ups2s:mean}).
In the following analysis this mass was fixed to 10.250\gevcc, which was measured
in the $18 < p_T^{\Y2S} < 40 \gevc$ interval on the sum of 2011 and 2012 datasets, and the
systematic uncertainty on the results due to this assumption has been
determined.

\begin{figure}[H]
  \setlength{\unitlength}{1mm}
  \centering
  \begin{picture}(75,60)
    \put(0,0){
      \includegraphics*[width=75mm, height=60mm]{chib/ups2s/mean_b1_2p}
    }
  
    \put(0,12){\begin{sideways}\chiboneTwoP mass $\left[\gevcc\right]$\end{sideways}}
    \put(35,2){$p_T^{\Y2S} \left[\gevc\right]$}
  

    \put(50,50){\textcolor{blue}{\sqs=7\tev}}
    \put(50,45){\textcolor{red}{\sqs=8\tev}}
    \put(44,50){
      \includegraphics*[width=4mm, height=2mm]{blue}
    }
    \put(44,45){
      \includegraphics*[width=4mm, height=2mm]{red}
    }

  % \graphpaper[5](0,0)(75, 60)
  \end{picture}
  \caption {\small
    The \chiboneTwoP mass reconstructed at $\chib \to \Y2S \gamma$ decays
  }
  \label{fig:chib:ups2s:mean}
\end{figure}

\Cref{fig:chib:ups2s:yields} shows the number of signal events as a function
of $p_T^{\Y2S}$. \Cref{tab:chib:ups2s:fits} in Appendix summarizes the
obtained results.

\begin{figure}[H]
  \setlength{\unitlength}{1mm}
  \centering
  \begin{picture}(150,60)
    \put(0,0){
      \includegraphics*[width=75mm, height=60mm]{chib/ups2s/N2P}
    }
    \put(75,0){
      \includegraphics*[width=75mm, height=60mm]{chib/ups2s/N3P}
    }


    \put(2,25){\begin{sideways}Events\end{sideways}}
    \put(35,2){$p_T^{\Y2S} \left[\gevc\right]$}
    \put(55,50){$\chibTwoP$}

    \put(77,25){\begin{sideways}Events\end{sideways}}
    \put(110,2){$p_T^{\Y2S} \left[\gevc\right]$}
    \put(130,50){$\chibThreeP$}


    \put(50,45){\textcolor{blue}{\sqs=7\tev}}
    \put(50,40){\textcolor{red}{\sqs=8\tev}}
    \put(45,45){
      \includegraphics*[width=3mm, height=2mm]{bsf}
    }
    \put(45,40){
      \includegraphics*[width=3mm, height=2mm]{rco}
    }

    \put(125,45){\textcolor{blue}{\sqs=7\tev}}
    \put(125,40){\textcolor{red}{\sqs=8\tev}}
    \put(120,45){
      \includegraphics*[width=3mm, height=2mm]{bsf}
    }
    \put(120,40){
      \includegraphics*[width=3mm, height=2mm]{rco}
    }

  % \graphpaper[5](0,0)(75, 60)
  \end{picture}
  \caption {\small
    Distribution of \chib yields in $\chib \to \Y2S \gamma$ decay in specified
    $p_T^{\Y2S}$ ranges.
  }
  \label{fig:chib:ups2s:yields}
\end{figure}

\begin{figure}[H]
  \setlength{\unitlength}{1mm}
  \centering
  \begin{picture}(150,60)
    \put(0,0){
      \includegraphics*[width=75mm, height=60mm]{chib/ups2s/N2P_scaledbylum}
    }
    \put(75,0){
      \includegraphics*[width=75mm, height=60mm]{chib/ups2s/N3P_scaledbylum}
    }


    \put(2,25){\begin{sideways}Arbitrary units\end{sideways}}
    \put(35,2){$p_T^{\Y2S} \left[\gevc\right]$}
    \put(55,50){$\chibTwoP$}

    \put(77,25){\begin{sideways}Arbitrary units\end{sideways}}
    \put(110,2){$p_T^{\Y2S} \left[\gevc\right]$}
    \put(130,50){$\chibThreeP$}


    \put(50,45){\textcolor{blue}{\sqs=7\tev}}
    \put(50,40){\textcolor{red}{\sqs=8\tev}}
    \put(45,45){
      \includegraphics*[width=3mm, height=2mm]{bsf}
    }
    \put(45,40){
      \includegraphics*[width=3mm, height=2mm]{rco}
    }

    \put(125,45){\textcolor{blue}{\sqs=7\tev}}
    \put(125,40){\textcolor{red}{\sqs=8\tev}}
    \put(120,45){
      \includegraphics*[width=3mm, height=2mm]{bsf}
    }
    \put(120,40){
      \includegraphics*[width=3mm, height=2mm]{rco}
    }

  % \graphpaper[5](0,0)(75, 60)
  \end{picture}
  \caption {\small
    Distribution of \chib yields in $\chib \to \Y2S \gamma$ decay in specified
    $p_T^{\Y2S}$ ranges. The distribution  normalized  by bin size and
    luminosity value. }
  \label{fig:chib:ups2s:yields_scaled}
\end{figure}


\Cref{fig:chib:ups2s:yields_scaled} shows the yields normalized by the bin
size and the luminosity. Both \chibTwoP and \chibThreeP yields are smoothly
decreasing functions of $p_T^{\Y2S}$, as expected.
%% ===========================================================================
\subsection{\texorpdfstring{\chib}{xb} yields in
    \texorpdfstring{$\chib \to \Y3S \gamma$}{xb --> Y(3S) gamma} decays}
\label{sec:chib:ups3s:fit}

The fit was performed in the mass interval from  10.440 to 10.760 \gevcc. 
A second order polynomial for the background (described in Equation 5.3) is used.
Due to the large fluctuations in the background, the
parameters of this component are fixed. 
A systematic uncertainty is assigned to this effect (\Cref{tab:syst:floatingbg_ups3s}).


\Cref{fig:chib:ups3s:nominal} shows the mass distribution in the
transverse momentum range $24<p_T^{\Y3S}<40\gevc$.
\Cref{tab:chib:ups3s:nominal} details the corresponding fit parameters.

\begin{figure}[H]
  \setlength{\unitlength}{1mm}
  \centering
  \begin{picture}(150,60)
    %
    \put(0,0){
      \includegraphics*[width=75mm, height=60mm]{chib/ups3s/f2011_27_40}
    }
    \put(75,0){
      \includegraphics*[width=75mm, height=60mm]{chib/ups3s/f2012_27_40}
    }


    \put(3,23){\scriptsize \begin{sideways}Candidates/(20\mevcc)\end{sideways}}
    \put(9,13){$m_{\mumu \gamma} - m_{\mumu} + m_{\Y3S}^{PDG} \left[\gevcc\right]$}

    \put(78,23){\scriptsize \begin{sideways}Candidates/(20\mevcc)\end{sideways}}
    \put(84,13){$m_{\mumu \gamma} - m_{\mumu} + m_{\Y3S}^{PDG} \left[\gevcc\right]$}

    \put(40,53){\sqs=7\tev}
    \put(115,53){\sqs=8\tev}

    \put(30,47){$27 < p_T^{\Y3S} < 40 \gevc$}
    \put(105,47){$27 < p_T^{\Y3S} < 40 \gevc$}



    % \graphpaper[5](0,0)(150, 60)
  \end{picture}
  \caption {\small
    Distribution of the mass difference $\mumu \gamma - \mumu$ for selected
    \chib(3P) candidates (black points) together with the result of the fit
    (solid red curve), including background (dotted blue curve) and signals
    (dashed green and magenta curves) contributions. Green dashed curve corresponds
    to \chibone signal and magenta dashed curve to \chibtwo signal.
    The bottom insert shows the  pull distribution of the fit. The pull is
    defined as the difference  between the data and fit value divided by the
    data error.
  }
  \label{fig:chib:ups3s:nominal}
\end{figure}
\begin{table}[H]
\caption{\small Data fit parameters for $\chi_{b1,2}(3P) \to \Y3S \gamma$ decays}
\centering
\scalebox{1}{
\begin{tabular}{lrr}\toprule
 & \multicolumn{2}{c}{$\Upsilon(3S)$ transverse momentum intervals, \gevc}\\
 & \multicolumn{2}{c}{27 -- 40}\\
\cmidrule(r){2-3}
 & \multicolumn{1}{c}{\sqs = 7\tev} & \multicolumn{1}{c}{\sqs = 8\tev}\\
\midrule
$N_{\chibThreeP}$ & 32 $\pm$ 8 & 81 $\pm$ 13\\

\rule{0pt}{4ex}Background & 96 $\pm$ 11 & 274 $\pm$ 19\\

\rule{0pt}{4ex}$\mu_{\chiboneThreeP}, \mevcc$ & 10,517 $\pm$ 4 & 10,505.5 $\pm$ 2.4\\
$\sigma_{\chiboneThreeP}, \mevcc$ & 10.0 & 12.0\\

\rule{0pt}{4ex}$\tau$ & -14.39 & -14.39\\
$c_0$ & 0.6155 & 0.6155\\
$c_1$ & 0.561 & 0.561\\

\rule{0pt}{4ex}$\chi^2 / n.d.f$ & 0.34 & 0.99\\
\bottomrule
\end{tabular}
} % scalebox
\label{tab:chib:ups3s:nominal}
\end{table}


A good resolution on the \chiboneThreeP mass is observed, so this decay can be
used for \chiboneThreeP mass estimation. \Cref{fig:chib-3s:m3p} shows how the
measured \chiboneThreeP mass depends on the \chiboneThreeP and \chibtwoThreeP
yields ratio ($\lambda$ parameter), which is unknown~(\Cref{sec:ratio}).
The \chiboneThreeP mass is measured to be $10{,}510 \pm
1.6\stat \pm 6\syst \mevc$, where the combined 2011 and 2012 datasets are used, 
the central value has been obtained by setting $\lambda = 0.5$ in the fit and the 
systematic error takes into account the uncertainties on  
the $\lambda$ parameter and the mass difference between \chiboneThreeP and \chibtwoThreeP. 
This result is in agreement with a recent unpublished
\lhcb study with converted photons, where the \chiboneThreeP mass is
$10{,}510 \pm 3\stat{}^{+4.4}_{-3.4}\syst$. This result is also compatible within~$\sim{}1.5\sigma$ with 
the $\chi_{b1,2}(3P)$ mass barycenter reported by \atlas~\cite{Aad:2011ih} ($10{,}530 \pm 5\stat
\pm 17\syst \mevcc$) and D0~\cite{Abazov:2012gh}  ($10{,}551 \pm 14\syst \pm 17\stat$)


\begin{figure}[H]
  \setlength{\unitlength}{1mm}
  \centering
  \begin{picture}(80,60)
    %
    \put(0,0){
      \includegraphics*[width=80mm, height=60mm]{chib/ups3s/m3p_lambda}
    }

     \put(0,15){\scriptsize \begin{sideways}Mass of \chiboneThreeP (\gevcc)\end{sideways}}
     \put(75,0){$\lambda$}

    \put(15,54){\includegraphics*[width=4mm, height=2mm]{blue}}
    \put(15,50){\includegraphics*[width=4mm, height=2mm]{red}}

    \put(20,54){\scriptsize \textcolor{blue}{$\Delta{m_{\chi_{b1,2}(3P)}} = 13\mevcc$}}
    \put(20,50){\scriptsize \textcolor{red}{$\Delta{m_{\chi_{b1,2}(3P)}} = 10\mevcc$}}
    % \put(48,140){\scriptsize \textcolor{cyan}{\sqs=7\tev (2010)}}
    
    % \put(15,169){\includegraphics*[width=4mm, height=2mm]{blue}}
    % \put(15,165){\includegraphics*[width=4mm, height=2mm]{red}}     

     % \graphpaper[5](0,0)(80, 60)
  \end{picture}
  \caption {\small
     Mass of \chiboneThreeP as function of $\lambda$ and mass difference
     between \chibtwoThreeP and \chibtwoThreeP states
     ($\Delta{m_{\chi_{b1,2}(3P)}}$). Measurements were performed on
     merged 2011 and 2012 datasets.
   }
  \label{fig:chib-3s:m3p}
\end{figure}

% \begin{figure}[H]
  \setlength{\unitlength}{1mm}
  \centering
  \begin{picture}(150,60)
    %
    \put(0,0){
      \includegraphics*[width=75mm, height=60mm]{chib/ups3s/f2011_27_40}
    }
    \put(75,0){
      \includegraphics*[width=75mm, height=60mm]{chib/ups3s/f2012_27_40}
    }


    \put(3,23){\scriptsize \begin{sideways}Candidates/(20\mevcc)\end{sideways}}
    \put(9,13){$m_{\mumu \gamma} - m_{\mumu} + m_{\Y3S}^{PDG} \left[\gevcc\right]$}

    \put(78,23){\scriptsize \begin{sideways}Candidates/(20\mevcc)\end{sideways}}
    \put(84,13){$m_{\mumu \gamma} - m_{\mumu} + m_{\Y3S}^{PDG} \left[\gevcc\right]$}

    \put(40,53){\sqs=7\tev}
    \put(115,53){\sqs=8\tev}

    \put(30,47){$27 < p_T^{\Y3S} < 40 \gevc$}
    \put(105,47){$27 < p_T^{\Y3S} < 40 \gevc$}



    % \graphpaper[5](0,0)(150, 60)
  \end{picture}
  \caption {\small
    Distribution of the mass difference $\mumu \gamma - \mumu$ for selected
    \chib(3P) candidates (black points) together with the result of the fit
    (solid red curve), including background (dotted blue curve) and signals
    (dashed green and magenta curves) contributions. Green dashed curve corresponds
    to \chibone signal and magenta dashed curve to \chibtwo signal.
    The bottom insert shows the  pull distribution of the fit. The pull is
    defined as the difference  between the data and fit value divided by the
    data error.
  }
  \label{fig:chib:ups3s:nominal}
\end{figure}



