\section{\texorpdfstring{$\chi_b$}{chib} signal extraction}
\label{sec:chib}

%% ============================================================================
\subsection{Selection}
\label{sec:chib:selection}

The selected $\Upsilon$ candidates  are combined with photon candidates to form
\chib candidates. Well reconstructed photons are selected by requiring the
transverse momentum greater a 600 \mevc. To further suppress the background
the cosine of the angle of the photon direction in the center of mass of the
$\mumu\gamma$ system with respect to the momentum of this system, is required to
be greater than zero. An additional loose cut on the photon confidence level is
required to be greater than 0.01.

The criteria for event selection with a reconstructed photon is summarized in
Table~\ref{tab:chib:selection:photons}:

\input{tables/chib/selection_photons}

To separate decays by different $\Upsilon$ channels the cut on dimuon mass is
applied as shown in Table~\ref{tab:chib:selection:window}:

% \begin{table}[H]
% \caption{\small The cuts on dimuon mass window.}
% \centering
% \scalebox{0.9}{
% \begin{tabular}{lrr}\toprule
% \multicolumn{1}{c}{Decay} & \multicolumn{1}{c}{Cut} & \multicolumn{1}{c}{Description}\\
% \midrule
% $\chib(1,2,3P) \to \Upsilon(1S)$ & $9310 < \mumu < 9600 \mevc$ & $3 \sigma_{\Y1S} < \mumu < 2.5 \sigma_{\Y1S} \mevc$\\
% $\chib(2,3P) \to \Upsilon(2S)$ & $9870 < \mumu < 10090  \mevc$ &  $3 \sigma_{\Y2S} < \mumu < \sigma_{\Y2S} \mevc$\\
% $\chib(3P) \to \Upsilon(3S)$ & $10300 < \mumu < 10526  \mevc$ &   $\sigma_{\Y3S} < \mumu < 3 \sigma_{\Y3S} \mevc$\\
% \bottomrule
% \end{tabular}
% }
% \label{tab:chib:selection:window}
% \end{table}


\begin{table}[H]
\caption{\small The cuts on dimuon mass window.}
\centering
\scalebox{0.9}{
\begin{tabular}{lr}\toprule
\multicolumn{1}{c}{Decay} & \multicolumn{1}{c}{Cut}\\
\midrule
$\chib(1,2,3P) \to \Upsilon(1S)$ & $9310 < \mumu < 9600 \mevc$\\
$\chib(2,3P) \to \Upsilon(2S)$ & $9870 < \mumu < 10090  \mevc$\\
$\chib(3P) \to \Upsilon(3S)$ & $10300 < \mumu < 10526  \mevc$\\
\bottomrule
\end{tabular}
}
\label{tab:chib:selection:window}
\end{table}

The $\Upsilon$ selection cuts
~(Table~\ref{tab:upsilon:selection:study:summary}), cuts  on $\gamma$
~(Table~\ref{tab:chib:selection:photons}) and limit on dimuon mass
~(Table~\ref{tab:chib:selection:window}) are used for obtaining \chib yields
using a fit model which is described in the next section.


%% ============================================================================
\subsection{Fit model}
\label{sec:chib:fit}

This section describes the common properties of the fit model that is used for
obtaining yields in each of the \chib decays. The resuls of the fits are
given in the following sections.

The efficient way to obtain \chib signal yields is by fitting event candidates
in the distribution of invariant mass difference $m(\mumu\gamma) - m(\mumu)$. In
this case the bias and resolution effects from $\Upsilon$ reconstruction are
suppressed. For clearness, the PDG mass of the corresponding $\Upsilon$
particles is added to the mass difference value in each plot.

The $\chi_b(jP)$ (j=1,2,3) signals are composed of three parts: $\chi_{b0}(jP)$,
$\chi_{b1}(jP)$, $\chi_{b2}(jP)$. The $\chi_{b0}$ signals is hard to detect
because it has a low radiative branching ratio in comparison with the other two
parts. So the $\chi_{b0}$ states were excluded from this study and the fit
model.

To determine the $\chi_b$ signal yields, an unbinned maximum likelihood fit to
$m(\mumu\gamma) - m(\mumu)$ has been performed. The signal has been modeled with
a sum of Crystal Ball (CB)  functions. The background is parameterized with a
product of an exponential function and a linear combination of basic Bernstein
polynomials~\cite{Phillips:2003} with non-negative coefficients $c_{i}^2$:

\begin{equation}
\label{eq:bernstein}
{\mathscr B}_{n}(x) = e^{-\tau x} \times \sum_{i=0}^{n} c_{i}^2 {\mathscr B}_{n}^{i}(x)
\end{equation}
Such combination results in smooth and non-negative function that can be used as
PDF.


The number of CB functions and the order of the polynomial depend on the decay
under study and are described in the section corresponding to that decay. The
definition of CB function is similar to Eq.~\ref{eq:cb}. The $\alpha$ and $n$
parameters of CB are fixed to the values obtained from simulation~\ref{sec:mc}
and are shown in Table~\ref{tab:chib:fit:tail}:

\begin{table}[H]
\caption{\small   The $\alpha$ and $n$ parameters of CB functions.}
\centering
\begin{tabular}{lrr}
\toprule
Signal & $\alpha$ & $n$ \\
\midrule
$\chi_{b1,2}(1,2P)$ & -1.1 & 5 \\
$\chi_{b1,2}(3P)$ & -1.25 & 5 \\
\bottomrule
\end{tabular}
\label{tab:chib:fit:tail}
\end{table}

Due to the current insufficient detector energy resolution and small mass
difference between $\chi_{b2}(jP)$ and $\chi_{b1}(jP)$ (j=1,2,3) it is too hard
to separate $\chi_{b1}$ and $\chi_{b2}$ states by two CB functions. Thus mean,
width and yield values of  $\chi_{b1}$ and $\chi_{b2}$ signals are linked
together by the following rules:

\begin{equation}
  \begin{aligned}
\mu_{\chi_{b2}} = \mu_{\chi_{b1}} + \Delta m_{\chi_{b1b2}}^{PDG} \\
N_{\chi_{b2}} = \frac{(1-\lambda)}{\lambda} N_{\chi_{b1}} \\
\sigma_{\chi_{b2}} = k \sigma_{\chi_{b1}}
  \end{aligned}
\end{equation}

, where $\Delta m_{b2b1}^{PDG}$ is the corresponding PDG mass difference which
is fixed in the fit. The $\lambda$ parameter depends on $p_T(\Upsilon)$ range
and is fixed to the value that is based on the theoretical prediction discussed
in Section~\ref{sec:ratio}. The $k$ parameter is the ratio between the widths of
$\chi_{b1}$ and $\chi_{b2}$ signals. The $k$ values are fixed to $1.05$ for
$\chi_b(1P)$  and to 1 for $\chi_b(1P)$ and $\chi_b(2P)$ signals. To reduce
errors the width of each CB function ($\sigma$) is fixed to the value obtained
from simulation~(Section~\ref{sec:mc}).

%% ============================================================================
\subsection{The \texorpdfstring{\chib}{xb} yields in
	\texorpdfstring{$\chib \to \Y1S \gamma$}{chib -> Y(1S) gamma} decays}
\label{sec:chib:ups1s:fit}

The \Y1S could be produced in radiative decays of six \chib particles:
$\chi_{bi}(jP) \to \Y1S \gamma$ (i=1,2; j=1,2,3). So the sum of six CB functions
is used to determine \chib signals in these decays. The mass of \chiboneOneP
($\mu_{\chiboneOneP}$) is a free parameter and  other parameters are linked by
the following rules:

\begin{equation}
  \begin{aligned}
\mu_{\chiboneTwoP} = \mu_{\chiboneOneP} + \Delta m_{\chi_{b1}(2P)}^{PDG} \\
\mu_{\chiboneThreeP} = \mu_{\chiboneOneP} + \Delta m_{\chi_{b1}(3P)} \\
  \end{aligned}
\end{equation}
, where $\Delta m_{\chi_{b1}(2P)}^{PDG}$ is a difference between PDG masses of
\chiboneTwoP and \chiboneOneP. The $\Delta  m_{\chi_{b1}(3P)}$ is a difference
between masses of \chiboneThreeP and \chiboneOneP, where the mass of
\chiboneThreeP was taken from measurement conducted in this
study~(Section~\ref{sec:chib:ups3s:fit}). The parameters  $\Delta
m_{\chi_{b1}(2P)}^{PDG}$ and $\Delta  m_{\chi_{b1}(3P)}$ are fixed in the fit.

The order of the background polynomial in~Eq.~\ref{eq:bernstein} depends on the
$p_{T}^{\OneS}$ interval and is given in Table~\ref{tab:chib:ups1s:fit:order}.

\input{tables/chib/ups1s_fit_order}

The fit was performed in the mass interval from  9.77 \gevcc to 10.89 \gevcc.
Figure~\ref{fig:chib:ups1s:nominal} shows the mass distribution in transverse
momentum range $14 < p_T^{\OneS} < 40 \gevc$. In this range the fit has
the lowest relative error of signal yields. Table~\ref{tab:chib:ups1s:nominal}
details the corresponding fit parameters.

\begin{table}[H]
\caption{\small Data fit parameters for $\chi_{b1,2}(1,2,3P) \to \Y1S \gamma$ decays}
\centering
\resizebox{.75\textwidth}{!}{

\begin{tabular}{lrr}\toprule
 & \multicolumn{2}{c}{$\Upsilon(1S)$ transverse momentum intervals, \gevc}\\
 & \multicolumn{2}{c}{14 -- 40}\\
\cmidrule(r){2-3}
 & \multicolumn{1}{c}{\sqs = 7\tev} & \multicolumn{1}{c}{\sqs = 8\tev}\\
\midrule
$N_{\chibOneP}$ & 1910 $\pm$ 70 & 4610 $\pm$ 110\\
$N_{\chibTwoP}$ & 390 $\pm$ 40 & 900 $\pm$ 70\\
$N_{\chibThreeP}$ & 133 $\pm$ 31 & 200 $\pm$ 50\\

\rule{0pt}{4ex}Background & 9080 $\pm$ 120 & 24,500 $\pm$ 200\\

\rule{0pt}{4ex}$\mu_{\chiboneOneP}, \mevcc$ & 9892.5 $\pm$ 1.0 & 9893.2 $\pm$ 0.7\\
$\sigma_{\chiboneOneP}, \mevc$ & 19.0 & 19.5\\
$\sigma_{\chi_{b1}(2P)} / \sigma_{\chi_{b1}(1P)}$ & 1.5 & 1.5\\
$\sigma_{\chi_{b1}(3P)} / \sigma_{\chi_{b1}(1P)}$ & 1.86 & 1.86\\

\rule{0pt}{4ex}$\tau$ & -2.5 $\pm$ 0.5 & -3.10 $\pm$ 0.29\\
$c_0$ & -0.11 $\pm$ 0.11 & 0.04 $\pm$ 0.06\\
$c_1$ & 1.36 $\pm$ 0.04 & 0.25 $\pm$ 0.04\\

\rule{0pt}{4ex}$\chi^2 / n.d.f$ & 1.15 & 1.63\\
\bottomrule
\end{tabular}
} % scalebox/resizebox
\label{tab:chib:ups1s:nominal}
\end{table}
\begin{table}[H]
\caption{\small Data fit parameters for $\chi_{b1,2}(1,2,3P) \to \Y1S \gamma$ decays}
\centering
\resizebox{.75\textwidth}{!}{

\begin{tabular}{lrr}\toprule
 & \multicolumn{2}{c}{$\Upsilon(1S)$ transverse momentum intervals, \gevc}\\
 & \multicolumn{2}{c}{14 -- 40}\\
\cmidrule(r){2-3}
 & \multicolumn{1}{c}{\sqs = 7\tev} & \multicolumn{1}{c}{\sqs = 8\tev}\\
\midrule
$N_{\chibOneP}$ & 1910 $\pm$ 70 & 4610 $\pm$ 110\\
$N_{\chibTwoP}$ & 390 $\pm$ 40 & 900 $\pm$ 70\\
$N_{\chibThreeP}$ & 133 $\pm$ 31 & 200 $\pm$ 50\\

\rule{0pt}{4ex}Background & 9080 $\pm$ 120 & 24,500 $\pm$ 200\\

\rule{0pt}{4ex}$\mu_{\chiboneOneP}, \mevcc$ & 9892.5 $\pm$ 1.0 & 9893.2 $\pm$ 0.7\\
$\sigma_{\chiboneOneP}, \mevc$ & 19.0 & 19.5\\
$\sigma_{\chi_{b1}(2P)} / \sigma_{\chi_{b1}(1P)}$ & 1.5 & 1.5\\
$\sigma_{\chi_{b1}(3P)} / \sigma_{\chi_{b1}(1P)}$ & 1.86 & 1.86\\

\rule{0pt}{4ex}$\tau$ & -2.5 $\pm$ 0.5 & -3.10 $\pm$ 0.29\\
$c_0$ & -0.11 $\pm$ 0.11 & 0.04 $\pm$ 0.06\\
$c_1$ & 1.36 $\pm$ 0.04 & 0.25 $\pm$ 0.04\\

\rule{0pt}{4ex}$\chi^2 / n.d.f$ & 1.15 & 1.63\\
\bottomrule
\end{tabular}
} % scalebox/resizebox
\label{tab:chib:ups1s:nominal}
\end{table}

Table~~\ref{tab:chib:ups1s:nominal} show that the measured \chiboneOneP mass
has a  nice agreement with the PDG value $9892.78 \pm 0.26 \pm 0.31 \mevcc$.
In the further analysis this mass was fixed to 9.892 \gevcc.

% \input{chib/1s/pics/mass}

Figure~\ref{fig:chib:ups1s:yields} illustrates the number of signal events as
function of $\Y1S$ transverse momentum.
Table~\ref{tab:chib:ups1s:fits} in Appendix summarizes the obtained results.

\begin{figure}[H]
  \setlength{\unitlength}{1mm}
  \centering
  \begin{picture}(150,120)
    \put(0,0){
      \includegraphics*[width=75mm, height=60mm]{chib/ups1s/n3p_1s}
    }
    \put(0,60){
      \includegraphics*[width=75mm, height=60mm]{chib/ups1s/n1p_1s}
    }
    \put(75,60){
      \includegraphics*[width=75mm, height=60mm]{chib/ups1s/n2p_1s}
    }

    \put(2,25){\begin{sideways}Events\end{sideways}}
    \put(35,2){$p_T^{\Y1S} \left[\gevc\right]$}
    \put(55,50){$\chibThreeP$}

    \put(2,85){\begin{sideways}Events\end{sideways}}
    \put(35,62){$p_T^{\Y1S} \left[\gevc\right]$}
    \put(55,110){$\chibOneP$}

    \put(77,85){\begin{sideways}Events\end{sideways}}
    \put(110,62){$p_T^{\Y1S} \left[\gevc\right]$}
    \put(130,110){$\chibTwoP$}


    \put(50,45){\textcolor{blue}{\sqs=7\tev}}
    \put(50,40){\textcolor{red}{\sqs=8\tev}}
    \put(45,45){
      \includegraphics*[width=3mm, height=2mm]{bsf}
    }
    \put(45,40){
      \includegraphics*[width=3mm, height=2mm]{rco}
    }

    \put(50,105){\textcolor{blue}{\sqs=7\tev}}
    \put(50,100){\textcolor{red}{\sqs=8\tev}}
    \put(45,105){
      \includegraphics*[width=3mm, height=2mm]{bsf}
    }
    \put(45,100){
      \includegraphics*[width=3mm, height=2mm]{rco}
    }

    \put(125,105){\textcolor{blue}{\sqs=7\tev}}
    \put(125,100){\textcolor{red}{\sqs=8\tev}}
    \put(120,105){
      \includegraphics*[width=3mm, height=2mm]{bsf}
    }
    \put(120,100){
      \includegraphics*[width=3mm, height=2mm]{rco}
    }


  % \graphpaper[5](0,0)(75, 60)
  \end{picture}
  \caption {\small
    The \chib yields reconstructed from $\chib \to \Y1S \gamma$ decays scaled
    by bin size.
  }
  \label{fig:chib:ups1s:yields}
\end{figure}

Figure ~\ref{fig:chib:ups1s:yields} shows that \chibOneP and
\chibThreeP yields are the expected smooth decreasing functions of $p_T^{\Y1S}$.
 and fluctuations are observed on the plot for \chibTwoP signal.

\input{pics/chib/ups1s_mean}


%% ===========================================================================
\subsection{The \texorpdfstring{\chib}{chib} yields in
	\texorpdfstring{$\chib \to \Y2S \gamma$}{chib --> Y(2S) gamma} decays}
\label{sec:chib:ups2s:fit}

The fit was performed in the mass interval from 10.16 \gevcc to 11.04 \gevcc.
The  \chiboneTwoP peak width depends on $p_T^{\Y2S}$ interval and is fixed to
the value obtained from simulation~\ref{sec:mc} without any scaling . The
\chiboneThreeP peak width is fixed to \chiboneTwoP peak width scaled by 1.17.


 The order of background polynomial~\ref{eq:bernstein} is 3 in all intervals of
 \Y2S transverse momentum.


Figure~\ref{fig:chib:ups2s:nominal} shows the mass distribution in transverse
momentum range $18 < p_T^{\OneS} < 40 \gevc$. Table~\ref{tab:chib:ups2s:nominal}
details the corresponding fit parameters.

\begin{table}[H]
\caption{\small Data fit parameters for $\chi_{b1,2}(2,3P) \to \Y2S \gamma$ decays}
\centering
\resizebox{0.75\textwidth}{!}{

\begin{tabular}{lrr}\toprule
 & \multicolumn{2}{c}{$\Upsilon(2S)$ transverse momentum intervals, \gevc}\\
 & \multicolumn{2}{c}{18 -- 40}\\
\cmidrule(r){2-3}
 & \multicolumn{1}{c}{\sqs = 7\tev} & \multicolumn{1}{c}{\sqs = 8\tev}\\
\midrule
$N_{\chibTwoP}$ & 265 $\pm$ 30 & 660 $\pm$ 50\\
$N_{\chibThreeP}$ & 48 $\pm$ 17 & 73 $\pm$ 26\\

\rule{0pt}{4ex}Background & 2100 $\pm$ 60 & 5320 $\pm$ 90\\

\rule{0pt}{4ex}$\mu_{\chiboneTwoP}, \mevcc$ & 10,251.5 $\pm$ 2.0 & 10,251.9 $\pm$ 1.1\\
$\sigma_{\chiboneTwoP}, \mevc$ & 12.0 & 12.0\\
$\sigma_{\chi_{b1}(3P)} / \sigma_{\chi_{b1}(2P)}$ & 1.65 & 1.65\\

\rule{0pt}{4ex}$\tau$ & -7.7 $\pm$ 0.6 & -7.8 $\pm$ 0.4\\
$c_0$ & 0.440 $\pm$ 0.020 & 0.440 $\pm$ 0.014\\
$c_1$ & -2.11 $\pm$ 0.07 & -2.13 $\pm$ 0.05\\
$c_2$ & 0.79 $\pm$ 0.18 & 0.79 $\pm$ 0.16\\

\rule{0pt}{4ex}$\chi^2 / n.d.f$ & 0.96 & 1.1\\
\bottomrule
\end{tabular}
} % scalebox/resizebox
\label{tab:chib:ups2s:nominal}
\end{table}
\begin{table}[H]
\caption{\small Data fit parameters for $\chi_{b1,2}(2,3P) \to \Y2S \gamma$ decays}
\centering
\resizebox{0.75\textwidth}{!}{

\begin{tabular}{lrr}\toprule
 & \multicolumn{2}{c}{$\Upsilon(2S)$ transverse momentum intervals, \gevc}\\
 & \multicolumn{2}{c}{18 -- 40}\\
\cmidrule(r){2-3}
 & \multicolumn{1}{c}{\sqs = 7\tev} & \multicolumn{1}{c}{\sqs = 8\tev}\\
\midrule
$N_{\chibTwoP}$ & 265 $\pm$ 30 & 660 $\pm$ 50\\
$N_{\chibThreeP}$ & 48 $\pm$ 17 & 73 $\pm$ 26\\

\rule{0pt}{4ex}Background & 2100 $\pm$ 60 & 5320 $\pm$ 90\\

\rule{0pt}{4ex}$\mu_{\chiboneTwoP}, \mevcc$ & 10,251.5 $\pm$ 2.0 & 10,251.9 $\pm$ 1.1\\
$\sigma_{\chiboneTwoP}, \mevc$ & 12.0 & 12.0\\
$\sigma_{\chi_{b1}(3P)} / \sigma_{\chi_{b1}(2P)}$ & 1.65 & 1.65\\

\rule{0pt}{4ex}$\tau$ & -7.7 $\pm$ 0.6 & -7.8 $\pm$ 0.4\\
$c_0$ & 0.440 $\pm$ 0.020 & 0.440 $\pm$ 0.014\\
$c_1$ & -2.11 $\pm$ 0.07 & -2.13 $\pm$ 0.05\\
$c_2$ & 0.79 $\pm$ 0.18 & 0.79 $\pm$ 0.16\\

\rule{0pt}{4ex}$\chi^2 / n.d.f$ & 0.96 & 1.1\\
\bottomrule
\end{tabular}
} % scalebox/resizebox
\label{tab:chib:ups2s:nominal}
\end{table}

Table~\ref{tab:chib:ups2s:nominal} shows that the measured \chiboneTwoP mass is
about 5 \mevcc less than the PDG value $10255.46  \pm 0.22 \pm 0.50 \mevcc$. In
the further analysis this mass was fixed to 10.250 \gevcc, which
was measured in $18 < p_T^{\Y2S} < 40 \gevc$ interval in merged 2011 and
2012 data.

Figure~\ref{fig:chib:ups2s:yields} shows the number of signal events as function
of $p_T^{\Y2S}$. Table~\ref{tab:chib:ups2s:fits} in Appendix summarizes the
obtained results.

\input{pics/chib/ups2s_yields}

Figure ~\ref{fig:chib:ups2s:yields} shows that both \chibTwoP and
\chibThreeP yields are smooth decreasing functions of $p_T^{\Y2S}$, as it was
expected.

\input{pics/chib/ups2s_mean}

%% ===========================================================================
\subsection{The \texorpdfstring{\chib}{chib} yields in
	\texorpdfstring{$\chib \to \Y3S \gamma$}{chib --> Y(3S) gamma} decays}
\label{sec:chib:ups3s:fit}

The fit was performed in the mass interval from  10.440 to 10.760 \gevcc. The
\chiboneThreeP peak width is a free parameter in the fit.

Figure~\ref{fig:chib:ups3s:nominal} shows the mass distribution in transverse
momentum range $27<p_T^{\Y3S}<40\gevc$. Table~\ref{tab:chib:ups3s:nominal}
details the corresponding fit parameters.

\begin{table}[H]
\caption{\small Data fit parameters for $\chi_{b1,2}(3P) \to \Y3S \gamma$ decays}
\centering
\resizebox{0.7\textwidth}{!}{

\begin{tabular}{lrr}\toprule
 & \multicolumn{2}{c}{$\Upsilon(3S)$ transverse momentum intervals, \gevc}\\
 & \multicolumn{2}{c}{24 -- 40}\\
\cmidrule(r){2-3}
 & \multicolumn{1}{c}{\sqs = 7\tev} & \multicolumn{1}{c}{\sqs = 8\tev}\\
\midrule
$N_{\chibThreeP}$ & 66 $\pm$ 12 & 150 $\pm$ 20\\

\rule{0pt}{4ex}Background & 219 $\pm$ 17 & 591 $\pm$ 29\\

\rule{0pt}{4ex}$\mu_{\chiboneThreeP}, \mevcc$ & 10,513.6 $\pm$ 2.7 & 10,508.7 $\pm$ 1.8\\
$\sigma_{\chiboneThreeP}, \mevc$ & 10.0 & 10.0\\

\rule{0pt}{4ex}$\tau$ & -0.8 $\pm$ 1.7 & -2.5 $\pm$ 1.0\\
$c_0$ & 0.62 $\pm$ 0.11 & 0.62 $\pm$ 0.06\\
$c_1$ & -0.46 $\pm$ 0.14 & -0.40 $\pm$ 0.09\\

\rule{0pt}{4ex}$\chi^2 / n.d.f$ & 1.08 & 1.41\\
\bottomrule
\end{tabular}
} % scalebox/resizebox
\label{tab:chib:ups3s:nominal}
\end{table}
\begin{table}[H]
\caption{\small Data fit parameters for $\chi_{b1,2}(3P) \to \Y3S \gamma$ decays}
\centering
\resizebox{0.7\textwidth}{!}{

\begin{tabular}{lrr}\toprule
 & \multicolumn{2}{c}{$\Upsilon(3S)$ transverse momentum intervals, \gevc}\\
 & \multicolumn{2}{c}{24 -- 40}\\
\cmidrule(r){2-3}
 & \multicolumn{1}{c}{\sqs = 7\tev} & \multicolumn{1}{c}{\sqs = 8\tev}\\
\midrule
$N_{\chibThreeP}$ & 66 $\pm$ 12 & 150 $\pm$ 20\\

\rule{0pt}{4ex}Background & 219 $\pm$ 17 & 591 $\pm$ 29\\

\rule{0pt}{4ex}$\mu_{\chiboneThreeP}, \mevcc$ & 10,513.6 $\pm$ 2.7 & 10,508.7 $\pm$ 1.8\\
$\sigma_{\chiboneThreeP}, \mevc$ & 10.0 & 10.0\\

\rule{0pt}{4ex}$\tau$ & -0.8 $\pm$ 1.7 & -2.5 $\pm$ 1.0\\
$c_0$ & 0.62 $\pm$ 0.11 & 0.62 $\pm$ 0.06\\
$c_1$ & -0.46 $\pm$ 0.14 & -0.40 $\pm$ 0.09\\

\rule{0pt}{4ex}$\chi^2 / n.d.f$ & 1.08 & 1.41\\
\bottomrule
\end{tabular}
} % scalebox/resizebox
\label{tab:chib:ups3s:nominal}
\end{table}

The fine resolution of \chiboneThreeP is observed, so this decay can be used
for precise measurement of \chiboneThreeP mass. In this study the mass 
was fixed to 10.507 \gevcc which was measured on merged 2011 and 2012 data.

